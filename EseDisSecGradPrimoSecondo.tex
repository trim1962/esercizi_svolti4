\tcbstartrecording
\section{Disequazioni di primo e secondo grado fratte}
\begin{exercise}
	\tipo{AP}	Risolvere la seguente disequazione $\dfrac{6x^2+13x+6}{2x+1}\geq0$
	\tcblower
	Risolvere la seguente disequazione $\dfrac{6x^2+13x+6}{2x+1}\geq0$
	\begin{align*}
	2x+1>&0\\
	2x>&-1\\
	x>&-\dfrac{1}{2}\\
	x_{1,2}=&\xunodue{6}{+13}{+6}=\dfrac{-13\pm\sqrt{169-154}}{12}\\
	=&\dfrac{-13\pm\sqrt{25}}{12}=\dfrac{-13\pm 5}{12}\\
	=&\begin{cases}
	x_1=-\dfrac{18}{12}=-\dfrac{3}{2}\\
	x_2=-\dfrac{8}{12}=-\dfrac{2}{3}
	\end{cases}\\
	\end{align*}
	\begin{center}
		\includestandalone{grafici/disfrazAP1}
	\end{center}
L'esercizio chiede quando la frazione è maggiore o uguale a zero quindi la risposta è 
\begin{align*}
-\dfrac{3}{2}<x&< -\dfrac{2}{3}\\ x\geq&-\dfrac{1}{2}\\
\end{align*}
\end{exercise}
\begin{exercise}
	\tipo{DP}	Risolvere la seguente disequazione $\dfrac{10-2x^2-x}{2-4x}< 0$
	\tcblower
	Risolvere la seguente disequazione $\dfrac{10-2x^2-x}{2-4x}< 0$
	\begin{align*}
		2-4x>&0\\
		-4x>&-2\\
		x<&\dfrac{1}{2}\\
		x_{1,2}=&\xunodue{-2}{-1}{+10}=\dfrac{1\pm\sqrt{1+80}}{-4}\\
		=&\dfrac{1\pm\sqrt{81}}{-4}=\dfrac{1\pm 9}{-4}\\
		=&\begin{cases}
			x_1=\dfrac{-8}{-4}=2\\
			x_2=\dfrac{10}{-4}=-\dfrac{5}{2}
		\end{cases}\\
	\end{align*}
	\begin{center}
		\includestandalone{grafici/disfrazDP1}
	\end{center}
	L'esercizio chiede quando la frazione è minore di zero quindi la risposta è 
	\begin{align*}
		\dfrac{1}{2}<x&<2\\ x<&-\dfrac{5}{2}\\
	\end{align*}
\end{exercise}
\begin{exercise}
	\tipo{PF}	Risolvere la seguente disequazione $\dfrac{1-3x}{2x-3x^2-2}\leq 0$
	\tcblower
	Risolvere la seguente disequazione $\dfrac{1-3x}{2x-3x^2-2}\leq 0$
	\begin{align*}
	x_{1,2}=&\xunodue{-3}{+2}{-2}=\dfrac{-2\pm\sqrt{4-24}}{-6}\\
	=&\dfrac{-2\pm\sqrt{-20}}{-6}=\dfrac{1\pm 9}{-4}\\
\qquad\text{nessuna soluzione}\\
		1-3x&\geq0\\
	-3x\geq&-1\\
	x\leq&\dfrac{1}{3}\\
	\end{align*}
	\begin{center}
		\includestandalone{grafici/disfrazPF1}
	\end{center}
	L'esercizio chiede quando la frazione è minore o uguale zero quindi la risposta è 
	\begin{align*}
	 x\leq&\dfrac{1}{3}\\
	\end{align*}
\end{exercise}
\begin{exercise}
	\tipo{AP}	Risolvere la seguente disequazione $\dfrac{x+2-15x^2}{3x+1}\leq0$
	\tcblower
	Risolvere la seguente disequazione $\dfrac{x+2-15x^2}{3x+1}\leq0$
	\begin{align*}
	3x+1>&0\\
	2x>&-1\\
	x>&-\dfrac{1}{3}\\
	x_{1,2}=&\xunodue{-15}{+1}{+2}=\dfrac{-1\pm\sqrt{1+120}}{-30}\\
	=&\dfrac{-1\pm\sqrt{121}}{-30}=\dfrac{-1\pm 11}{-30}\\
	=&\begin{cases}
	x_1=-\dfrac{10}{30}=-\dfrac{1}{3}\\
	x_2=\dfrac{-12}{-30}=\dfrac{2}{5}
	\end{cases}\\
	\end{align*}
	\begin{center}
		\includestandalone{grafici/disfrazAP2}
	\end{center}
	L'esercizio chiede quando la frazione è minore a zero quindi la risposta è 
	\begin{align*}
	x<&-\dfrac{1}{3}\\
	-\dfrac{1}{3}<x&\leq \dfrac{2}{5}\\ 
	\end{align*}
\end{exercise}