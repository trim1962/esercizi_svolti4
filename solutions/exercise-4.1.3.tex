	Risolvere la seguente disequazione $\dfrac{6x^2-5x-4}{-1-x-x^2}< 0$
\begin{align*}
-1-x-x^2>&0\\
-1-x-x^2=&0\\
x_{1,2}=&\xunodue{-1}{-1}{-4}=\dfrac{1\pm\sqrt{1-4}}{-2}\\
=&\dfrac{1\pm\sqrt{-3}}{-2}
\qquad\text{nessuna soluzione}\\
6x^2-5x-4>&0\\
6x^2-5x-4=&0\\
x_{1,2}=&\xunodue{6}{-5}{-4}=\dfrac{5\pm\sqrt{25+96}}{12}\\
=&\dfrac{5\pm\sqrt{121}}{12}\\
=&\dfrac{5\pm 11}{12}=\begin{cases}
x_1=\dfrac{16}{12}=\dfrac{4}{3}\\
x_2=-\dfrac{6}{12}=-\dfrac{1}{2}
\end{cases}\\
\end{align*}
\begin{center}
	\includestandalone{quarto/grafici/disfrazAF1}
\end{center}
L'esercizio chiede quando la frazione è minore di zero quindi la riposta è
\begin{align*}
x\leq& -\dfrac{1}{2}\\  \dfrac{4}{3}<&x\\
\end{align*}
