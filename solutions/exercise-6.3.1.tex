	Trovare il $C.E.$ della funzione $y=\sqrt{\dfrac{x+3}{x+5}}$
	
	La funzione è irrazionale pari fratta quindi:
	\begin{enumerate}
		\item Il radicando deve essere positivo o uguale a zero.
		\item Il denominatore non può essere zero.
	\end{enumerate}
	\begin{align*}
	\intertext{La prima condizione impone che:}
	\dfrac{x+3}{x+5}\geq&0
	\intertext{che equivale a:}
	x+3\geq&0\\
	x\geq&-3
	\intertext{La seconda condizione impone che}
	x+5>&0\\
	x>&-5
	\end{align*}
	Otteniamo il grafico
	\begin{center}
		\includestandalone{quarto/grafici/disfrazIrPaFr1}
	\end{center}
	Quindi il $C.E.$ è
	\begin{align*}
	x<&5\\
	-3\leq& x
	\end{align*}
