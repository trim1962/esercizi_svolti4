	Trovare il Dominio\index{Dominio} e la Positività\index{Positività} della funzione $y=\dfrac{x^2-x-6}{2x^2-7x-4}$
	
	La funzione è razionale fratta\index{Razionale!Fratta} quindi
	\begin{enumerate}
		\item Il denominatore non può essere zero.
	\end{enumerate}
	\begin{align*}
		\intertext{Quindi per la condizione}
		2x^2-7x-4=&0\\
		x_{1,2}=&\xunodue{2}{-7}{-4}\\
		=&\dfrac{7\pm\sqrt{49+32}}{4}\\
		=&\dfrac{7\pm \sqrt{81}}{4}\\
		=&\dfrac{7\pm 9}{4}\\
		=&\begin{cases}
			x_1=\dfrac{16}{4}=4\\
			x_2=\dfrac{-2}{4}=-\dfrac{1}{2}
		\end{cases}\\
	\end{align*}
	Quindi il Dominio\index{Dominio} e la Positività\index{Positività} è
	\begin{align*}
		\forall x\in\R-\lbrace 4,&-\dfrac{1}{2}\rbrace\\
	\end{align*}
	Studiamo la Positività\index{Positività}
	
	Bisogna risolvere la disequazione
	\begin{align*}
		\dfrac{x^2-x-6}{2x^2-7x-4}\geq&0\\
		\intertext{Spezzo la frazione e ottengo:}
		x^2-x-6\geq&0\\
		\intertext{Che diventa}
		x^2-x-6=&0\\
		x_{1,2}=&\xunodue{1}{-1}{-6}\\
		=&\dfrac{1\pm\sqrt{1+24}}{6}\\
		=&\dfrac{1\pm \sqrt{25}}{6}\\
		=&\begin{cases}
			x_1=\dfrac{6}{2}=3\\
			x_2=-\dfrac{4}{2}=-2
		\end{cases}\\
		\intertext{Resta da risolvere}
		2x^2-7x-4>&0
		\intertext{Che diventa}
		2x^2-7x-4=&0
		\intertext{Per il calcoli precedenti}
		=&\begin{cases}
			x_1=\dfrac{16}{4}=4\\
			x_2=-\dfrac{2}{4}=-\dfrac{1}{2}
		\end{cases}\\
	\end{align*}
	Otteniamo il grafico
	\begin{center}
	\includestandalone{grafici/disfrazAA_E2.tex}
	\end{center}
	La funzione è positiva per
	\begin{gather*}
		-2<x\\
		-\dfrac{1}{3}\leqslant x<3\\
		4\leq x
	\end{gather*}
