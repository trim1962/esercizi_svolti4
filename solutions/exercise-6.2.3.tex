	Trovare il Dominio\index{Dominio} e la Positività\index{Positività} della funzione $y=\dfrac{\sqrt{x+5}}{x+3}$
	
	La funzione è irrazionale pari fratta\index{Irrazionale!Pari!Fratta} quindi:
	\begin{enumerate}
		\item Il radicando deve essere positivo o uguale a zero.
		\item Il denominatore non può essere zero.
	\end{enumerate}
	\begin{align*}
	\intertext{La prima condizione impone che:}
	x+5\geq&0
	\intertext{che equivale a:}
	x+5\geq&0\\
	x\geq&-5
	\intertext{La seconda condizione impone che}
	x+3=&0\\
	x=&-3
	\end{align*}
	Otteniamo il grafico
	\begin{center}
		\includestandalone{grafici/disfrazIrPaFr2}
	\end{center}
	Quindi il Dominio\index{Dominio} e la Positività\index{Positività} è
	\begin{align*}
	-5\leq x<-3\\
	-3<& x
	\end{align*}
	o in maniera equivalente
	\begin{align*}
	-5\leq& x\\
     x\neq&-3
	\end{align*}
