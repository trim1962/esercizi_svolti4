	Determinare concavità,asse, fuoco, vertice, direttrice della parabola $y=2x^2+3x+1$
	\begin{description}
		\item[Intersezione asse $y$] La parabola e l'asse $y$ hanno un punto in comune, per cui l'asse e la curva passano contemporaneamente per lo stesso punto. Risolvo il sistema \[\begin{cases}
		y=2x^2+3x+1\\
		x=0
		\end{cases}\] Sostituisco $x$ e ottengo $A\left(0\;\text{;}1\right)$
			\item[Intersezione asse $x$] La parabola e l'asse $x$ hanno un punto in comune, per cui l'asse e la curva passano contemporaneamente per lo stesso punto. Risolvo il sistema \[\begin{cases}
		y=2x^2+3x+1\\
		y=0
		\end{cases}\] Ottengo l'equazione \[2x^2+3x+1=0\] che risolvo
		\[x_{1,2}=\dfrac{-b\pm\sqrt{b^2-4ac}}{2a}=\dfrac{-3\pm\sqrt{9-8}}{4}=\begin{cases}
		x_1=-\frac{1}{2}\\
		x_2=-1
		\end{cases} \] Le soluzioni ci danno le ascisse dei  punti $B\left(-1\;\text{;}0\right)$ e $C\left(-\dfrac{1}{2}\;\text{;}0\right)$
	\end{description}
