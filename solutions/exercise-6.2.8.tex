	Trovare il Dominio\index{Dominio} e la Positività\index{Positività}  della funzione  $y=\dfrac{\sqrt{x^2-9}}{x^2-x+1}$
	
	La funzione è irrazionale pari fratta\index{Irrazionale!Pari!Fratta} quindi:
	\begin{enumerate}
		\item Il radicando deve essere positivo o nullo.
		\item Il denominatore non può essere zero.
	\end{enumerate}
	\begin{align*}
		\intertext{La prima condizione equivale a:}
		x^2-9\geq &0\\
		\intertext{quindi}
		x^2-9=&0\\
		x^2-9=&x^2+0x-9=0\\
		x_{1,2}=&\xunodue{1}{0}{-9}\\
		=&\dfrac{0\pm\sqrt{0-4(1)(-9)}}{2}\\
		=&\begin{cases}
			x_1=\dfrac{6}{2}=3\\
			x_2=\dfrac{-6}{2}=-3
		\end{cases}\\
		\intertext{Quindi per la seconda condizione}
		x^2-x+1=&0\\
	x_{1,2}=&\xunodue{1}{-1}{1}\\
	=&\dfrac{1\pm\sqrt{1-4}}{2}\\
	=&\dfrac{1\pm\sqrt{-3}}{2}\\
	\end{align*}
	Otteniamo il grafico
	\begin{center}
		\includestandalone{grafici/disfrazIrPaFr6}
	\end{center}
	Quindi il Dominio\index{Dominio}  è
	\begin{align*}
		-3\leq x&&3\leq x\\
	\end{align*}
 Positività\index{Positività}

 Bisogna risolvere la disequazione
 \begin{align*}
 \dfrac{\sqrt{x^2-9}}{x^2-x+1}\geq&0\\
 \intertext{Essendo fratta}
 \sqrt{x^2-9}\geq&0\\
 \intertext{La radice dove esiste è sempre positiva}
  x^2-x+1>&0\\
  \intertext{Corrisponde a}
  x^2-x+1=&0\\
  \intertext{Ma questa equazione non ha soluzioni. Quindi dato che $a$ è positiva il denominatore è positivo.}
  \intertext{Numeratore positivo, denominatore positivo la funzione è sempre positiva}
 \end{align*}
