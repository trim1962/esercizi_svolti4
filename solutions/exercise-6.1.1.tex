	Trovare il $C.E.$ della funzione $y=\dfrac{3x^2+2+3x}{7x-2-5x^2} $
	
	La funzione è razionale fratta\index{Razionale!Fratta} quindi
		\begin{enumerate}
		\item Il denominatore non può essere zero.
	\end{enumerate}
\begin{align*}
\intertext{Quindi per la condizione}
7x-2-5x^2=&0\\
x_{1,2}=&\xunodue{-5}{7}{-2}\\
=&\dfrac{-7\pm\sqrt{49-40}}{-10}\\
=&\dfrac{-7\pm \sqrt{9}}{-10}\\
=&\dfrac{-7\pm 3}{-10}\\
=&\begin{cases}
x_1=\dfrac{-10}{-10}=1\\
x_2=\dfrac{-4}{-10}=\dfrac{2}{5}
\end{cases}\\
\end{align*}
Quindi il $C.E.$ è
\begin{align*}
\forall x\in\R-\lbrace 1,&\dfrac{2}{5}\rbrace\\
\end{align*}
Studiamo la positività

Bisogna risolvere la disequazione
\begin{align*}
	\dfrac{3x^2+2+3x}{7x-2-5x^2}\geq&0\\
	\intertext{Spezzo la frazione e ottengo:}
	3x^2+2+3x\geq&0\\
	\intertext{Che diventa}
	3x^2+2+3x=&0\\
	x_{1,2}=&\xunodue{3}{3}{2}\\
	=&\dfrac{-3\pm\sqrt{9-24}}{6}\\
	=&\dfrac{-3\pm \sqrt{-15}}{6}\\
	\intertext{Radice negativa non ho soluzioni}
	\intertext{Resta da risolvere}
	7x-2-5x^2>&0
	\intertext{Che diventa}
	7x-2-5x^2=&0
	\intertext{Per il calcoli precedenti}
	=&\begin{cases}
		x_1=1\\
		x_2=\dfrac{2}{5}
	\end{cases}\\
\end{align*}
Otteniamo il grafico
\begin{center}
\includestandalone{quarto/grafici/disfrazFA_E1.tex}
\end{center}
La funzione è positiva per \[\dfrac{2}{5}<x<1\]
