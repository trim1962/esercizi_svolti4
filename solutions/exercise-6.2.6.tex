	Trovare il Dominio\index{Dominio} e la Positività\index{Positività} della funzione $y=\sqrt{\dfrac{(1-x^{2})\cdot (x+4)}{(x-2)}}$%\tipo{AP}
	
	La funzione è irrazionale pari fratta\index{Irrazionale!Pari!Fratta} quindi:
	\begin{enumerate}
	\item Il radicando deve essere positivo o nullo.
	\item Il denominatore non può essere zero.
	\end{enumerate}
	\begin{align*}
	\intertext{La prima condizione equivale a:}
	\frac{(1-x^{2})\cdot (x+4)}{(x-2)}\geq &0
	\intertext{quindi}
	1-x^2\geq&0\\
x_{1,2}=&\xunodue{-1}{0}{1}\\
=&\dfrac{0\pm\sqrt{0-4(-1)(-1)}}{-2}\\
=&\begin{cases}
x_1=\dfrac{2}{-1}=-1\\
x_2=\dfrac{-2}{-2}=+1
\end{cases}\\
x+4\geq&0\\
\intertext{Quindi per la seconda condizione}
x-2>&0
	\end{align*}
		Otteniamo il grafico
	\begin{center}
		\includestandalone{grafici/disfrazIrPaFr4}
	\end{center}
	Quindi il Dominio\index{Dominio} e la Positività\index{Positività} è
	\begin{align*}
	-4\leq x&\leq-1\\1\leq x&<2
	\end{align*}
	
