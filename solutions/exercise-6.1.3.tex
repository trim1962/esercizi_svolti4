	Trovare il $C.E.$ della funzione $y=\dfrac{1-2x}{8x-x^2-15}$
	
	La funzione è razionale fratta\index{Razionale!Fratta} quindi
	\begin{enumerate}
		\item Il denominatore non può essere zero.
	\end{enumerate}
	\begin{align*}
		\intertext{Quindi per la condizione}
		8x-x^2-15=&0\\
		x_{1,2}=&\xunodue{-1}{+8}{-15}\\
		=&\dfrac{-8\pm\sqrt{4}}{-2}\\
		=&\dfrac{-8\pm \sqrt{4}}{-2}\\
		=&\dfrac{-8\pm 2}{-2}\\
		=&\begin{cases}
			x_1=\dfrac{-10}{-2}=5\\
			x_2=\dfrac{-6}{-2}=3
		\end{cases}\\
	\end{align*}
	Quindi il $C.E.$ è
	\begin{align*}
		\forall x\in\R-\lbrace 5,&3\rbrace\\
	\end{align*}
	Studiamo la positività
	
	Bisogna risolvere la disequazione
	\begin{align*}
		\dfrac{1-2x}{8x-x^2-15}\geq&0\\
		\intertext{Spezzo la frazione e ottengo:}
		1-2x\geq&0\\
		\intertext{Dato che è di primo grado si separa}
		-2x\geq&-1\\
		x\leq\dfrac{1}{2}
		\intertext{Resta da risolvere}
		8x-x^2-15>&0
		\intertext{Che diventa}
		8x-x^2-15=&0
		\intertext{Per il calcoli precedenti}
		=&\begin{cases}
			x_1=\dfrac{-10}{-2}=5\\
			x_2=\dfrac{-6}{-2}=3
		\end{cases}\\
	\end{align*}
	Otteniamo il grafico
	\begin{center}
	\includestandalone{quarto/grafici/disfrazAP_E3.tex}
	\end{center}
	La funzione è positiva per
	\begin{gather*}
		-\dfrac{1}{2}\leqslant x<3\\
		5\leq x
	\end{gather*}
