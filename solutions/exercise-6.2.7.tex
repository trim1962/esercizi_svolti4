	Trovare il Dominio\index{Dominio} e la Positività\index{Positività}  della funzione  $y=\dfrac{\sqrt{1-x^2}}{x+1}$. %\tipo{AP}
	
	La funzione è irrazionale pari fratta\index{Irrazionale!Pari!Fratta} quindi:
	\begin{enumerate}
		\item Il radicando deve essere positivo o nullo.
		\item Il denominatore non può essere zero.
	\end{enumerate}
	\begin{align*}
	\intertext{La prima condizione equivale a:}
	1-x^2\geq &0
	\intertext{quindi}
		1-x^2= &0\\
		1-x^2= &-x^2+0x+1=0\\
	x_{1,2}=&\xunodue{-1}{0}{1}\\
	=&\dfrac{0\pm\sqrt{0-4(-1)(1)}}{2}\\
	=&\begin{cases}
	x_1=\dfrac{2}{-2}=-1\\
	x_2=\dfrac{-2}{-2}=+1
	\end{cases}\\
	\intertext{Quindi per la seconda condizione}
	x+1=&0\\
	x=&-1\\
	\end{align*}
	Otteniamo il grafico
	\begin{center}
		\includestandalone{grafici/disfrazIrPaFr5}
	\end{center}
	Quindi il Dominio\index{Dominio} e la Positività\index{Positività} è
	\begin{align*}
	-1< x&\leq 1\\
	\end{align*}
