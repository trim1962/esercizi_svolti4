	Trovare il $C.E.$  della funzione  $y=\dfrac{\sqrt{x^2-1}}{x+1}$ e quando è positiva. \tipo{AP}
	
	La funzione è irrazionale pari fratta\index{Irrazionale!Pari!Fratta} quindi:
	\begin{enumerate}
		\item Il radicando deve essere positivo o nullo.
		\item Il denominatore non può essere zero.
	\end{enumerate}
	\begin{align*}
	\intertext{La prima condizione equivale a:}
	x^2-1\geq &0
	\intertext{quindi}
	x_{1,2}=&\xunodue{1}{0}{-1}\\
	=&\dfrac{0\pm\sqrt{0-4(1)(-1)}}{2}\\
	=&\begin{cases}
	x_1=\dfrac{2}{-1}=-1\\
	x_2=\dfrac{-2}{-2}=+1
	\end{cases}\\
	\intertext{Quindi per la seconda condizione}
	x+1=&0\\
	x=&-1\\
	\end{align*}
	Otteniamo il grafico
	\begin{center}
		\includestandalone{grafici/disfrazIrPaFr5}
	\end{center}
	Quindi il $C.E.$ è
	\begin{align*}
	-1< x&\leq 1\\
	\end{align*}
	
