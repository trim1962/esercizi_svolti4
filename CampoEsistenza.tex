\chapter{Campo di esistenza e positività}
\tcbstartrecording
\section{Funzioni razionali}
%1
\begin{exercise}
	Trovare il $C.E.$ della funzione $y=\dfrac{3x^2+2+3x}{7x-2-5x^2} $
	\tcblower
	Trovare il $C.E.$ della funzione $y=\dfrac{3x^2+2+3x}{7x-2-5x^2} $
	
	La funzione è razionale fratta\index{Razionale!Fratta} quindi
		\begin{enumerate}
		\item Il denominatore non può essere zero.
	\end{enumerate}
\begin{align*}
\intertext{Quindi per la condizione}
7x-2-5x^2=&0\\
x_{1,2}=&\xunodue{-5}{7}{-2}\\
=&\dfrac{-7\pm\sqrt{49-40}}{-10}\\
=&\dfrac{-7\pm \sqrt{9}}{-10}\\
=&\dfrac{-7\pm 3}{-10}\\
=&\begin{cases}
x_1=\dfrac{-10}{-10}=1\\
x_2=\dfrac{-4}{-10}=\dfrac{2}{5}
\end{cases}\\
\end{align*}
Quindi il $C.E.$ è 
\begin{align*}
\forall x\in\R-\lbrace 1,&\dfrac{2}{5}\rbrace\\
\end{align*}
Studiamo la positività

Bisogna risolvere la disequazione
\begin{align*}
	\dfrac{3x^2+2+3x}{7x-2-5x^2}\geq&0\\
	\intertext{Spezzo la frazione e ottengo:}
	3x^2+2+3x\geq&0\\
	\intertext{Che diventa}
	3x^2+2+3x=&0\\
	x_{1,2}=&\xunodue{3}{3}{2}\\
	=&\dfrac{-3\pm\sqrt{9-24}}{6}\\
	=&\dfrac{-3\pm \sqrt{-15}}{6}\\
	\intertext{Radice negativa non ho soluzioni}
	\intertext{Resta da risolvere}
	7x-2-5x^2>&0
	\intertext{Che diventa}
	7x-2-5x^2=&0
	\intertext{Per il calcoli precedenti}
	=&\begin{cases}
		x_1=1\\
		x_2=\dfrac{2}{5}
	\end{cases}\\
\end{align*}
Otteniamo il grafico 
\begin{center}
\includestandalone{grafici/disfrazFA_E1.tex}
\end{center}
La funzione è positiva per \[\dfrac{2}{5}<x<1\]
\end{exercise}
%2
\begin{exercise}
	Trovare il $C.E.$ della funzione $y=\dfrac{x^2-x-6}{2x^2-7x-4}$
	\tcblower
	Trovare il $C.E.$ della funzione $y=\dfrac{x^2-x-6}{2x^2-7x-4}$
	
	La funzione è razionale fratta\index{Razionale!Fratta} quindi
	\begin{enumerate}
		\item Il denominatore non può essere zero.
	\end{enumerate}
	\begin{align*}
		\intertext{Quindi per la condizione}
		2x^2-7x-4=&0\\
		x_{1,2}=&\xunodue{2}{-7}{-4}\\
		=&\dfrac{7\pm\sqrt{49+32}}{4}\\
		=&\dfrac{7\pm \sqrt{81}}{4}\\
		=&\dfrac{7\pm 9}{4}\\
		=&\begin{cases}
			x_1=\dfrac{16}{4}=4\\
			x_2=\dfrac{-2}{4}=-\dfrac{1}{2}
		\end{cases}\\
	\end{align*}
	Quindi il $C.E.$ è 
	\begin{align*}
		\forall x\in\R-\lbrace 4,&-\dfrac{1}{2}\rbrace\\
	\end{align*}
	Studiamo la positività
	
	Bisogna risolvere la disequazione
	\begin{align*}
		\dfrac{x^2-x-6}{2x^2-7x-4}\geq&0\\
		\intertext{Spezzo la frazione e ottengo:}
		x^2-x-6\geq&0\\
		\intertext{Che diventa}
		x^2-x-6=&0\\
		x_{1,2}=&\xunodue{1}{-1}{-6}\\
		=&\dfrac{1\pm\sqrt{1+24}}{6}\\
		=&\dfrac{1\pm \sqrt{25}}{6}\\
		=&\begin{cases}
			x_1=\dfrac{6}{2}=3\\
			x_2=-\dfrac{4}{2}=-2
		\end{cases}\\
		\intertext{Resta da risolvere}
		2x^2-7x-4>&0
		\intertext{Che diventa}
		2x^2-7x-4=&0
		\intertext{Per il calcoli precedenti}
		=&\begin{cases}
			x_1=\dfrac{16}{4}=4\\
			x_2=-\dfrac{2}{4}=-\dfrac{1}{2}
		\end{cases}\\
	\end{align*}
	Otteniamo il grafico 
	\begin{center}
	\includestandalone{grafici/disfrazAA_E2.tex}
	\end{center}
	La funzione è positiva per 
	\begin{gather*}
		-2<x\\
		-\dfrac{1}{3}\leqslant x<3\\
		4\leq x
	\end{gather*}
\end{exercise}
%3
\begin{exercise}
	Trovare il $C.E.$ della funzione $y=\dfrac{1-2x}{8x-x^2-15}$
	\tcblower
	Trovare il $C.E.$ della funzione $y=\dfrac{1-2x}{8x-x^2-15}$
	
	La funzione è razionale fratta\index{Razionale!Fratta} quindi
	\begin{enumerate}
		\item Il denominatore non può essere zero.
	\end{enumerate}
	\begin{align*}
		\intertext{Quindi per la condizione}
		8x-x^2-15=&0\\
		x_{1,2}=&\xunodue{-1}{+8}{-15}\\
		=&\dfrac{-8\pm\sqrt{4}}{-2}\\
		=&\dfrac{-8\pm \sqrt{4}}{-2}\\
		=&\dfrac{-8\pm 2}{-2}\\
		=&\begin{cases}
			x_1=\dfrac{-10}{-2}=5\\
			x_2=\dfrac{-6}{-2}=3
		\end{cases}\\
	\end{align*}
	Quindi il $C.E.$ è 
	\begin{align*}
		\forall x\in\R-\lbrace 5,&3\rbrace\\
	\end{align*}
	Studiamo la positività
	
	Bisogna risolvere la disequazione
	\begin{align*}
		\dfrac{1-2x}{8x-x^2-15}\geq&0\\
		\intertext{Spezzo la frazione e ottengo:}
		1-2x\geq&0\\
		\intertext{Dato che è di primo grado si separa}
		-2x\geq&-1\\
		x\leq\dfrac{1}{2}
		\intertext{Resta da risolvere}
		8x-x^2-15>&0
		\intertext{Che diventa}
		8x-x^2-15=&0
		\intertext{Per il calcoli precedenti}
		=&\begin{cases}
			x_1=\dfrac{-10}{-2}=5\\
			x_2=\dfrac{-6}{-2}=3
		\end{cases}\\
	\end{align*}
	Otteniamo il grafico 
	\begin{center}
	\includestandalone{grafici/disfrazAP_E3.tex}
	\end{center}
	La funzione è positiva per 
	\begin{gather*}
		-\dfrac{1}{2}\leqslant x<3\\
		5\leq x
	\end{gather*}
\end{exercise}
\section{Funzioni irrazionali}
\begin{exercise}
	Trovare il $C.E.$ della funzione $y=\sqrt{\dfrac{x+5}{x+3}}$%\tipo{PP}
	\tcblower
	Trovare il $C.E.$ della funzione $y=\sqrt{\dfrac{x+5}{x+3}}$
	
	La funzione è irrazionale pari fratta\index{Irrazionale!Pari!Fratta} quindi:
	\begin{enumerate}
		\item Il radicando deve essere positivo o uguale a zero.
		\item Il denominatore non può essere zero.
	\end{enumerate}
\begin{align*}
\intertext{La prima condizione impone che:}
\dfrac{x+5}{x+3}\geq&0
\intertext{che equivale a:}
x+5\geq&0\\
x\geq&-5
\intertext{La seconda condizione impone che}
x+3>&0\\
x>&-3
\end{align*}
Otteniamo il grafico 
\begin{center}
	\includestandalone{grafici/disfrazIrPaFr1}
\end{center}
Quindi il $C.E.$ è 
\begin{align*}
x\leq&5\\
-3<& x
\end{align*}
\end{exercise}
\begin{exercise}
	Trovare il $C.E.$ della funzione $y=\sqrt[5]{\dfrac{x+5}{x+3}}$
	\tcblower
	Trovare il $C.E.$ della funzione $y=\sqrt[5]{\dfrac{x+5}{x+3}}$
	
	La funzione è irrazionale dispari fratta\index{Irrazionale!Dispari!Fratta} quindi:
	\begin{enumerate}
		\item Il radicando può essere negativo o positivo o nullo.
		\item Il denominatore non può essere zero.
	\end{enumerate}
	\begin{align*}
	\intertext{La seconda condizione impone che}
	x+3=&0\\
	x=&-3
	\end{align*}
	Quindi il $C.E.$ è 
	\begin{align*}
	\forall x\in\R-\lbrace-3&\rbrace\\
	\end{align*}
\end{exercise}
\begin{exercise}
	Trovare il $C.E.$ della funzione $y=\dfrac{\sqrt{x+5}}{x+3}$
	\tcblower
	Trovare il $C.E.$ della funzione $y=\dfrac{\sqrt{x+5}}{x+3}$
	
	La funzione è irrazionale pari fratta\index{Irrazionale!Pari!Fratta} quindi:
	\begin{enumerate}
		\item Il radicando deve essere positivo o uguale a zero.
		\item Il denominatore non può essere zero.
	\end{enumerate}
	\begin{align*}
	\intertext{La prima condizione impone che:}
	x+5\geq&0
	\intertext{che equivale a:}
	x+5\geq&0\\
	x\geq&-5
	\intertext{La seconda condizione impone che}
	x+3=&0\\
	x=&-3
	\end{align*}
	Otteniamo il grafico 
	\begin{center}
		\includestandalone{grafici/disfrazIrPaFr2}
	\end{center}
	Quindi il $C.E.$ è 
	\begin{align*}
	-5\leq x<-3\\
	-3<& x
	\end{align*}
	o in maniera equivalente
	\begin{align*}
	-5\leq& x\\
     x\neq&-3
	\end{align*}
\end{exercise}
\begin{exercise}
	Trovare il $C.E.$ della funzione $y=\dfrac{x+5}{\sqrt{x+3}}$
	\tcblower
		Trovare il $C.E.$ della funzione $y=\dfrac{x+5}{\sqrt{x+3}}$\tipo{P}
	
	La funzione è irrazionale pari fratta\index{Irrazionale!Pari!Fratta} quindi:
	\begin{enumerate}
		\item Il radicando deve essere positivo.
		\item Il denominatore non può essere zero.
	\end{enumerate}
	\begin{align*}
	\intertext{Le condizioni equivalgono a}
	x+3>&0\\
	x>&-3
	\end{align*}
	Otteniamo il grafico 
	\begin{center}
		\includestandalone{grafici/disfrazIrPaFr3}
	\end{center}
	Quindi il $C.E.$ è 
	\begin{align*}
	-3<& x
	\end{align*}
	
\end{exercise}
\begin{exercise}
	Trovare il $C.E.$ della funzione $y=\dfrac{x+5}{\sqrt[3]{x+3}}$
	\tcblower
	Trovare il $C.E.$ della funzione $y=\dfrac{x+5}{\sqrt[3]{x+3}}$\tipo{P}
	
	La funzione è irrazionale dispari fratta\index{Irrazionale!Dispari!Fratta} quindi:
	\begin{enumerate}
	\item Il radicando deve essere positivo o nullo.
	\item Il denominatore non può essere zero.
\end{enumerate}
	\begin{align*}
	\intertext{Le condizioni equivalgono a}
1-x^2\geq&0\\
x+4
	\end{align*}
	Quindi il $C.E.$ è 
\begin{align*}
\forall x\in\R-\lbrace-3&\rbrace\\
\end{align*}
	
\end{exercise}

\begin{exercise}
	Trovare il $C.E.$ della funzione $y=\sqrt{\frac{(1-x^{2})\cdot (x+4)}{(x-2)}}$
	\tcblower
	Trovare il $C.E.$ della funzione $y=\sqrt{\dfrac{(1-x^{2})\cdot (x+4)}{(x-2)}}$\tipo{AP}
	
	La funzione è irrazionale pari fratta\index{Irrazionale!Pari!Fratta} quindi:
	\begin{enumerate}
	\item Il radicando deve essere positivo o nullo.
	\item Il denominatore non può essere zero.
	\end{enumerate}
	\begin{align*}
	\intertext{La prima condizione equivale a:}
	\frac{(1-x^{2})\cdot (x+4)}{(x-2)}\geq &0
	\intertext{quindi}
	1-x^2\geq&0\\
x_{1,2}=&\xunodue{-1}{0}{1}\\
=&\dfrac{0\pm\sqrt{0-4(-1)(-1)}}{-2}\\
=&\begin{cases}
x_1=\dfrac{2}{-1}=-1\\
x_2=\dfrac{-2}{-2}=+1
\end{cases}\\
x+4\geq&0\\
\intertext{Quindi per la seconda condizione}
x-2>&0
	\end{align*}
		Otteniamo il grafico 
	\begin{center}
		\includestandalone{grafici/disfrazIrPaFr4}
	\end{center}
	Quindi il $C.E.$ è 
	\begin{align*}
	-4\leq x&\leq-1\\1\leq x&<2
	\end{align*}
	
\end{exercise}

\begin{exercise}
	Trovare il $C.E.$  della funzione  $y=\dfrac{\sqrt{x^2-1}}{x+1}$ e quando è positiva.
	\tcblower
	Trovare il $C.E.$  della funzione  $y=\dfrac{\sqrt{x^2-1}}{x+1}$ e quando è positiva. \tipo{AP}
	
	La funzione è irrazionale pari fratta\index{Irrazionale!Pari!Fratta} quindi:
	\begin{enumerate}
		\item Il radicando deve essere positivo o nullo.
		\item Il denominatore non può essere zero.
	\end{enumerate}
	\begin{align*}
	\intertext{La prima condizione equivale a:}
	x^2-1\geq &0
	\intertext{quindi}
	x_{1,2}=&\xunodue{1}{0}{-1}\\
	=&\dfrac{0\pm\sqrt{0-4(1)(-1)}}{2}\\
	=&\begin{cases}
	x_1=\dfrac{2}{-1}=-1\\
	x_2=\dfrac{-2}{-2}=+1
	\end{cases}\\
	\intertext{Quindi per la seconda condizione}
	x+1=&0\\
	x=&-1\\
	\end{align*}
	Otteniamo il grafico 
	\begin{center}
		\includestandalone{grafici/disfrazIrPaFr5}
	\end{center}
	Quindi il $C.E.$ è 
	\begin{align*}
	-1< x&\leq 1\\
	\end{align*}
	
\end{exercise}