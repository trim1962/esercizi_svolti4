% !TEX encoding = UTF-8 Unicode
% !TEX TS-program = pdflatex

\documentclass[a4paper,oneside]{book}%
\usepackage{base}
%\geometry{top=2cm,bottom=2cm,left=2cm,right=2cm}
\usepackage[big]{layaureo}
\usepackage{grafica}
\usepackage{matematica}
\usepackage{tabelle}
\DeclareCaptionFormat{grafico}{\textbf{Grafico \thefigure}#2#3}
\DeclareCaptionFormat{esempio}{\textbf{Esempio \thefigure}#2#3}
\usepackage{imakeidx}
\makeindex
\makeindex[name=dissec,title=Disequazioni secondo grado]
\makeindex[options=-s ../Mod_base/oldclaudio.sti]
\usepackage{date}
\usepackage{pagina}
\usepackage{unita_misura}
\usepackage{indice}
\usepackage{utili}
\usepackage{copyright}
\usepackage{CDloghi}
\usepackage{stand_class}

\newcommand{\HRule}{\rule{\linewidth}{0.5mm}}
\usepackage{placeins} 
\makeatletter
\renewcommand\frontmatter{%
	\cleardoublepage
	\@mainmatterfalse
	\pagenumbering{arabic}}
\renewcommand\mainmatter{%
	\cleardoublepage
	\@mainmattertrue}
\makeatother
\newcommand{\tipo}[1]{\textbf{#1}\index[dissec]{#1}}

\usepackage{qrcode}
%%%%%%%%%%%%%%%%%%%%%%%%%%%%%%%%
%%%lunghezza arrotondamenti%%%%%
\newcommand{\lungarrotandamento}{4}
%%%%%%%%%%%%%%%%%%%%%%%%%%%%%%%%%%
\includeonly{%
quarto/disequazioni_primogrado,
quarto/disequazioni_secondogrado,
quarto/tabelle_disequazioni
}
\usepackage[grumpy,mark,markifdirty,raisemark=0.95\paperheight]{gitinfo2}
\usepackage[toc,page]{appendix}

\renewcommand{\appendixtocname}{Appendice}

\renewcommand{\appendixpagename}{Appendice}

\usepackage[italian]{varioref}
\usepackage{hyperxmp}
\usepackage[pdfpagelabels]{hyperref}
\usepackage[italian]{cleveref}
\usepackage{tcolorboxgest}
\title{Esercizi svolti quarto}
\author{Claudio Duchi}
\date{\datetime}
\hypersetup{%
	pdfencoding=auto,
	urlcolor={blue},
	pdftitle={Esercizi svolti},
	pdfsubject={terzo},
	pdfstartview={FitH},
	pdfpagemode={UseOutlines},
	pdflicenseurl={http://creativecommons.org/licenses/by-nc-nd/3.0/},
	pdflang={it},
	pdfmetalang={it},
	pdfkeywords={Parabola,Disequazioni},
	pdfcopyright={Copyright (C) 2019, Claudio Duchi},
	pdfcontacturl={http://breviariomatematico.altervista.org},
	pdfcontactpostcode={},
	pdfcontactphone={},
	pdfcontactemail={claduc},
	pdfcontactcountry={Italy},
	pdfcontactcity={Perugia},
	pdfcontactaddress={},
	pdfcaptionwriter={Claudio Duchi},
	pdfauthortitle={},%
	pdfauthor={Claudio Duchi},
	linkcolor={blue},
	colorlinks=true,
	citecolor={red},
	breaklinks,
	bookmarksopen,
	verbose,
	baseurl={http://breviariomatematico.altervista.org}
}
\listfiles
\begin{document}
	\frontmatter
	\begin{titlepage}
		\begin{center}
				\Lgrandedue\\[1cm]
			\textsc{\LARGE Claudio Duchi}\\[1.2cm]
			\HRule \\[0.4cm]
			{ \huge \bfseries ESERCIZI SVOLTI DI MATEMATICA}\\[0.4cm]
			{\LARGE \textsc{QUARTO MANUTENZIONE}}
			\HRule \\[1.2cm]
			\polylogo[5.5]{18}		
		{\large $-$\DTMnow$-$}	
	\end{center}
{\centering
Release:\gitReln\ (\gitAbbrevHash)\ Autore:\gitAuthorName\ 
\gitCommitterDate \\
}
	\end{titlepage}
	\setcounter{page}{2} 
	\CDcopyright
	\tableofcontents 
	%\addcontentsline{toc}{chapter}{\listtablename}
	%\listoftables
	\addcontentsline{toc}{chapter}{\listfigurename}
	\listoffigures
	\renewcommand\lstlistlistingname{Esempi e contro esempi}
	\addcontentsline{toc}{chapter}{\lstlistlistingname}
	\addcontentsline{toc}{section}{Esempi}
	\lstlistoflistings%{}
	{
		%	https://tex.stackexchange.com/questions/318486/number-freestyle-causes-an-overlay-in-the-list-of-tcolorboxes/318512#318512
		\makeatletter
		\renewcommand{\l@tcolorbox}{\@dottedtocline{1}{0pt}{3em}}
		\makeatother
		\tcblistof[\section*]{thm}{Esempi}
		\addcontentsline{toc}{section}{Contro esempi}
		\tcblistof[\section*]{cthm}{Contro esempi}
	}
	
	\mainmatter%
	   \tcbstartrecording
   \chapter{Parabola}
\section{Parabola Nota}
 \begin{exercise}
	Determinare concavità,asse, fuoco, vertice, direttrice della parabola $y=2x^2+4x+2$\index{Parabola!concavità}\index{Parabola!fuoco}\index{Parabola!vertice}\index{Parabola!direttrice}\index{Parabola!asse}
	\tcblower
	Determinare concavità,asse, fuoco, vertice, direttrice della parabola $y=2x^2+4x+2$
	\begin{description}
		\item[Concavità] $a>0$ concavità rivolta verso l'alto.\index{Parabola!concavità}
		\item[Delta] $\Delta=b^2-4ac=16-16=0$\index{Parabola!delta}
		\item[Asse] $x=-\dfrac{b}{2a}=-\dfrac{4}{4}=-1$\index{Parabola!asse}
		\item[Fuoco] $F\left(-\dfrac{b}{2\cdot a}\;\text{;}\dfrac{1-\Delta}{4\cdot a}\right)=F\left(-1\;\text{;}\dfrac{1}{8}\right)=F\left(\dfrac{1}{6}\;\text{;} 1\right)$\index{Parabola!fuoco}
		\item[Vertice] $V\left(-\dfrac{b}{2\cdot a}\;\text{;}-\dfrac{\Delta}{4\cdot a}\right)=V\left(-1\;\text{;}0\right)$\index{Parabola!vertice}
		\item[Direttrice] $y=-\dfrac{1+\Delta}{4\cdot a}=-\dfrac{1}{8}$\index{Parabola!direttrice}
	\end{description}
\end{exercise}
\begin{exercise}[no solution]
	Determinare concavità,asse, fuoco, vertice, direttrice della parabola $y=2x^2+3x+1$
\end{exercise}

 \begin{exercise}
	Determinare concavità,asse, fuoco, vertice, direttrice della parabola $y=3x^2-x$\index{Parabola!concavità}\index{Parabola!fuoco}\index{Parabola!vertice}\index{Parabola!direttrice}\index{Parabola!asse}
	\tcblower
	Determinare concavità,asse, fuoco, vertice, direttrice della parabola $y=3x^2-x$
	\begin{description}
		\item[Concavità] $a>0$ concavità rivolta verso l'alto.\index{Parabola!concavità}
		\item[Delta] $\Delta=b^2-4ac=1+0=1$\index{Parabola!delta}
		\item[Asse] $x=-\dfrac{b}{2a}=\dfrac{1}{6}$\index{Parabola!asse}
		\item[Fuoco] $F\left(-\dfrac{b}{2\cdot a}\;\text{;}\dfrac{1-\Delta}{4\cdot a}\right)=F\left(\dfrac{1}{6}\;\text{;}0\right)$\index{Parabola!fuoco}
		\item[Vertice] $V\left(-\dfrac{b}{2\cdot a}\;\text{;}-\dfrac{\Delta}{4\cdot a}\right)=V\left(\dfrac{1}{6}\;\text{;}-\dfrac{1}{12}\right)$\index{Parabola!vertice}
		\item[Direttrice] $y=-\dfrac{1+\Delta}{4\cdot a}=-\dfrac{1}{6}$\index{Parabola!direttrice}
	\end{description}
\end{exercise}
\begin{exercise}[no solution]
	Determinare concavità,asse, fuoco, vertice, direttrice della parabola $y=3x^2-3x-10$
\end{exercise}
\begin{exercise}[no solution]
	Determinare concavità,asse, fuoco, vertice, direttrice della parabola $y=\frac{2}{3}x^2-1$
\end{exercise}

 \begin{exercise}
	Determinare concavità,asse, fuoco, vertice, direttrice della parabola $y=-10x^2-x+2$\index{Parabola!concavità}\index{Parabola!fuoco}\index{Parabola!vertice}\index{Parabola!direttrice}\index{Parabola!asse}
	\tcblower
	Determinare concavità,asse, fuoco, vertice, direttrice della parabola $y=-10x^2-x+2$
	\begin{description}
		\item[Concavità] $a<0$ concavità rivolta verso il basso.\index{Parabola!concavità}
		\item[Delta] $\Delta=b^2-4ac=1+80=81$\index{Parabola!delta}
		\item[Asse] $x=-\dfrac{b}{2a}=-\dfrac{1}{20}=-1$\index{Parabola!asse}
		\item[Fuoco] $F\left(-\dfrac{b}{2\cdot a}\;\text{;}\dfrac{1-\Delta}{4\cdot a}\right)=F\left(-\dfrac{1}{20}\;\text{;}2\right)$\index{Parabola!fuoco}
		\item[Vertice] $V\left(-\dfrac{b}{2\cdot a}\;\text{;}-\dfrac{\Delta}{4\cdot a}\right)=V\left(-\dfrac{1}{20}\;\text{;}\dfrac{81}{40}\right)$\index{Parabola!vertice}
		\item[Direttrice] $y=-\dfrac{1+\Delta}{4\cdot a}=\dfrac{41}{20}$\index{Parabola!direttrice}
	\end{description}
\end{exercise}
\begin{exercise}
	Determinare concavità,asse, fuoco, vertice, direttrice della parabola $y=3x^2+1$\index{Parabola!concavità}\index{Parabola!fuoco}\index{Parabola!vertice}\index{Parabola!direttrice}\index{Parabola!asse}
	\tcblower
	Determinare concavità,asse, fuoco, vertice, direttrice della parabola $y=3x^2+1$
	\begin{description}
		\item[Concavità] $a>0$ concavità rivolta verso l'alto.\index{Parabola!concavità}
		\item[Delta] $\Delta=b^2-4ac=0-12=-12$\index{Parabola!delta}
		\item[Asse] $x=-\dfrac{b}{2a}=\dfrac{0}{6}=0$\index{Parabola!asse}
		\item[Fuoco] $F\left(-\dfrac{b}{2\cdot a}\;\text{;}\dfrac{1-\Delta}{4\cdot a}\right)=F\left(0\;\text{;}\dfrac{13}{12}\right)$\index{Parabola!fuoco}
		\item[Vertice] $V\left(-\dfrac{b}{2\cdot a}\;\text{;}-\dfrac{\Delta}{4\cdot a}\right)=V\left(0\;\text{;}1\right)$\index{Parabola!vertice}
		\item[Direttrice] $y=-\dfrac{1+\Delta}{4\cdot a}=-\dfrac{11}{12}$\index{Parabola!direttrice}
	\end{description}
\end{exercise}
\begin{exercise}[no solution]
	Determinare concavità,asse, fuoco, vertice, direttrice della parabola $y=3x^2-3x-10$
\end{exercise}
\begin{exercise}[no solution]
	Determinare concavità,asse, fuoco, vertice, direttrice della parabola $y=5x^2-2x$
\end{exercise}
\begin{exercise}[no solution]
	Determinare concavità,asse, fuoco, vertice, direttrice della parabola $y=2x^2-3x$
\end{exercise}
  \begin{exercise}
	Determinare concavità,asse, fuoco, vertice, direttrice della parabola $y=3x^2-x+1$\index{Parabola!concavità}\index{Parabola!fuoco}\index{Parabola!vertice}\index{Parabola!direttrice}\index{Parabola!asse}
	\tcblower
	Determinare concavità,asse, fuoco, vertice, direttrice della parabola $y=3x^2-x+1$
		\begin{description}
			\item[Concavità] $a>0$ concavità rivolta verso l'alto.\index{Parabola!concavità}
			\item[Delta] $\Delta=b^2-4ac=1-4\cdot 4\cdot 1=1-12=-11$\index{Parabola!delta}
			\item[Asse] $x=-\dfrac{b}{2a}=-\dfrac{-1}{2\cdot 3}=\dfrac{1}{6}$\index{Parabola!asse}
			\item[Fuoco] $F\left(-\dfrac{b}{2\cdot a}\;\text{;}\dfrac{1-\Delta}{4\cdot a}\right)=F\left(\dfrac{1}{6}\;\text{;}\dfrac{1+11}{12}\right)=F\left(\dfrac{1}{6}\;\text{;} 1\right)$\index{Parabola!fuoco}
			\item[Vertice] $V\left(-\dfrac{b}{2\cdot a}\;\text{;}-\dfrac{\Delta}{4\cdot a}\right)=V\left(\dfrac{1}{6}\;\text{;}\dfrac{11}{12}\right)=V\left(\dfrac{1}{6}\;\text{;} \dfrac{11}{12}\right)$\index{Parabola!vertice}
			\item[Direttrice] $y=-\dfrac{1+\Delta}{4\cdot a}=-\dfrac{1-11}{12}=\dfrac{10}{12}=\dfrac{5}{6}$\index{Parabola!direttrice}
		\end{description}
%	\begin{center}
%		\includestandalone[width=.6\textwidth]{terzo/grafici/Piano_complesso_02}
%		\captionof{figure}{Piano complesso}\label{fig:disegnopianocomplesso02}
%	\end{center}
\end{exercise}
\section{Intersezioni}
\begin{exercise}[no solution]
	Trovare le intersezioni della parabola $y=2x^2+3x+1$
\end{exercise}
\begin{exercise}
Trovare le intersezioni della parabola $y=2x^2+3x+1$\index{Parabola!intersezione}\index{Asse!x}\index{Asse!y}\index{Parabola!sistema}
	\tcblower
	Determinare concavità,asse, fuoco, vertice, direttrice della parabola $y=2x^2+3x+1$
	\begin{description}
		\item[Intersezione asse $y$] La parabola e l'asse $y$ hanno un punto in comune, per cui l'asse e la curva passano contemporaneamente per lo stesso punto. Risolvo il sistema \[\begin{cases}
		y=2x^2+3x+1\\
		x=0
		\end{cases}\] Sostituisco $x$ e ottengo $A\left(0\;\text{;}1\right)$
			\item[Intersezione asse $x$] La parabola e l'asse $x$ hanno un punto in comune, per cui l'asse e la curva passano contemporaneamente per lo stesso punto. Risolvo il sistema \[\begin{cases}
		y=2x^2+3x+1\\
		y=0
		\end{cases}\] Ottengo l'equazione \[2x^2+3x+1=0\] che risolvo
		\[x_{1,2}=\dfrac{-b\pm\sqrt{b^2-4ac}}{2a}=\dfrac{-3\pm\sqrt{9-8}}{4}=\begin{cases}
		x_1=-\frac{1}{2}\\
		x_2=-1
		\end{cases} \] Le soluzioni ci danno le ascisse dei  punti $B\left(-1\;\text{;}0\right)$ e $C\left(-\dfrac{1}{2}\;\text{;}0\right)$
	\end{description}
\end{exercise}
\begin{exercise}[no solution]
	Trovare le intersezioni della parabola $y=3x^2-x$
\end{exercise}
\begin{exercise}[no solution]
	Trovare le intersezioni della parabola $y=3x^2-3x-10$
\end{exercise}
\begin{exercise}[no solution]
	Trovare le intersezioni della parabola $y=\frac{2}{3}x^2-1$
\end{exercise}
\begin{exercise}[no solution]
	Trovare le intersezioni della parabola $y=-10x^2-x+2$
\end{exercise}

\begin{exercise}[no solution]
	Trovare le intersezioni della parabola $y=3x^2-3x-10$
\end{exercise}
\begin{exercise}[no solution]
	Trovare le intersezioni della parabola $y=5x^2-2x$
\end{exercise}
\begin{exercise}[no solution]
	Trovare le intersezioni della parabola $y=2x^2-3x$
\end{exercise}
   \chapter{Soluzioni esercizi}
\tcbstoprecording
% \newpage
%\section{Soluzioni esercizi}
\tcbinputrecords
    \chapter{Disequazioni di primo grado}
\label{cha:DisequazioniDiPrimogrado}
 \section{Diseguaglianze}
\label{sec:Disequglianze}
Iniziamo con un po' di vocabolario. La tabella~\vref{tab:disuguaglianze} mostra le possibili disuguaglianze\index{Disuguaglianza} e il modo corretto di  leggerle.
\begin{table}
\centering
\begin{tabular}{lcll}
	\toprule
<&$a<b$&minore stretto&<<a è minore di b>>\\
>&$a>b$&maggiore stretto& <<a è maggiore di b>>\\
$\leq$&$a\leq b$&minore o uguale& <<a è minore di b>> o <<a è uguale a b>> \\
$\geq$&$a\geq b$&maggiore o uguale&<<a è maggiore di b>> o <<a è uguale a b>>\\
\bottomrule
\end{tabular}
\caption{Disuguaglianze}
\label{tab:disuguaglianze}
\end{table}
\begin{figure}
	\centering
\begin{tikzpicture}[>=latex',line join=bevel,]
%%
\node (1) at (27bp,76.177bp) [draw,ellipse] {Una disequazione};
\node (3) at (123.25bp,18bp) [draw,ellipse] {Determinata};
\node (2) at (104.11bp,95.152bp) [draw,draw=none] {è};
\node (5) at (181.47bp,113.47bp) [draw,ellipse] {Impossibile};
\node (4) at (86bp,172.48bp) [draw,ellipse] {Sempre verificata};
\draw [->] (1) ..controls (57.307bp,83.635bp) and (62.217bp,84.843bp)  .. (2);
\draw [->] (2) ..controls (135.93bp,102.69bp) and (140.95bp,103.88bp)  .. (5);
\draw [->] (2) ..controls (110.95bp,67.577bp) and (113.8bp,56.091bp)  .. (3);
\draw [->] (2) ..controls (97.637bp,122.79bp) and (94.941bp,134.3bp)  .. (4);
\end{tikzpicture}
	\caption{Disequazione e soluzioni}
	\label{fig:DidequazioniEsoluzioni}
\end{figure}
\subsection{Principi di equivalenza per le disuguaglianze}
\label{sec:PrincipiDiEquvalenzaPerLeDisuguaglianze}
Una disuguaglianza\index{Disuguaglianza} è un confronto fra due quantità. Ovviamente è vera o è falsa.\par Consideriamo l'esempio\nobs\vref{fig:DisPgradoesempio1a},partendo da una disuguaglianza vera, sommando la stessa quantità positiva a sinistra e a destra, otteniamo una disuguaglianza ancora vera.\par  Analogo discorso con l'esempio\nobs\vref{fig:DisPgradoesempio1b}. In questo caso la disuguaglianza si mantiene vera, sommando una quantità negativa.\par
Per la moltiplicazione il discorso è quasi analogo. Nell'esempio\nobs\vref{fig:esempioDisPrimoGrado3} la disuguaglianza si mantiene vera moltiplicando entrambi i lati per una quantità positiva.\par Il discorso cambia se moltiplichiamo a sinistra e a destra per una quantità negativa. Infatti, nell'esempio\nobs\vref{fig:esempioDisPrimoGrado4}, la disuguaglianza, per mantenersi vera, deve essere invertita.
\begin{figure}
	\centering
	\begin{subfigure}[b]{.4\linewidth}
		\begin{NodesList}
			\centering
			\begin{align*}
				-3<&6\AddNode\\
				-3+2<&6+2\AddNode\\[.5cm] 
				-1<&8\AddNode
			\end{align*}
			%\tikzset{LabelStyle/.style = {left=0.1cm,pos=0.5,text=red,fill=white}}
			\LinkNodes{Sommo $+2$}%    
			\LinkNodes{\begin{minipage}[h]{3cm}
					La disuguaglianza è ancora verificata
				\end{minipage}}%
			\end{NodesList}
		%\includestandalone[width=\textwidth]{DisPromoGrado/disPrimogradoEsempio1}
		\caption{Sommando quantità positive}
		\label{fig:DisPgradoesempio1a}
	\end{subfigure}%
	\centering
	\begin{subfigure}[b]{.4\linewidth}
	\begin{NodesList}
		\centering
		\begin{align*}
			6<&8\AddNode\\
			6-3<&8-3\AddNode\\[.5cm]
			3<&5\AddNode
		\end{align*}
		%\tikzset{LabelStyle/.style = {left=0.1cm,pos=0.5,text=red,fill=white}}
		\LinkNodes{sommo $-3$}%    
		\LinkNodes{\begin{minipage}[h]{3cm}
				La disuguaglianza è ancora verificata
			\end{minipage}}%
		\end{NodesList}
		\caption{Sommando quantità negative}
		\label{fig:DisPgradoesempio1b}
	\end{subfigure}%
	\captionsetup{format=esempio,list=no}
	\caption{Diseguaglianze equivalenti per la somma}
	\label{fig:DisPgradoesempio1}
\end{figure}
\begin{figure}
	\centering
	\begin{subfigure}[b]{.4\linewidth}
	\begin{NodesList}
		\centering
		\begin{align*}
			3<&6\AddNode\\
			6\cdot 2<&6\cdot 2\AddNode\\[.5cm] 
			6<&12\AddNode
		\end{align*}
		%\tikzset{LabelStyle/.style = {left=0.1cm,pos=0.5,text=red,fill=white}}
		\LinkNodes{Moltiplico per $+2$}%    
		\LinkNodes{\begin{minipage}[h]{3cm}
				La disuguaglianza è ancora verificata
			\end{minipage}}%
		\end{NodesList}
			%\includestandalone[width=\textwidth]{DisPromoGrado/disPrimogradoEsempio1}
			\caption{Moltiplicando quantità positive}
			\label{fig:esempioDisPrimoGrado3}
		\end{subfigure}%
		\centering
		\begin{subfigure}[b]{.4\linewidth}
			\begin{NodesList}
				\centering
				\begin{align*}
					-2<&5\AddNode\\
					6\cdot(-2) <&6\cdot (-2)\AddNode\\[.5cm]
					4>&-10\AddNode
				\end{align*}
				%\tikzset{LabelStyle/.style = {left=0.1cm,pos=0.5,text=red,fill=white}}
				\LinkNodes{Moltiplico per $-2$}%    
				\LinkNodes{\begin{minipage}[h]{3cm}
						La disuguaglianza è ancora verificata
					\end{minipage}}%
				\end{NodesList}
				\caption{Moltiplicando quantità negative}
				\label{fig:esempioDisPrimoGrado4}
			\end{subfigure}%
			\captionsetup{format=esempio,list=no}
			\caption{Diseguaglianze equivalenti per il prodotto}
			\label{fig:DisuguaglianzePrimogrado2}
		\end{figure}
Per le disuguaglianze valgono tre principi elencati in seguito:
\begin{enumerate}
	\item Sommando e sottraendo la stessa espressione a entrambi i lati della disuguaglianza, ottengo una disuguaglianza equivalente.
	\item Moltiplicando e dividendo per un numero positivo diverso da zero entrambi i lati della disuguaglianza, ottengo una disuguaglianza equivalente.
	\item  Moltiplicando e dividendo per un numero negativo diverso da zero entrambi i lati della disuguaglianza, ottengo una disuguaglianza equivalente se inverto il verso della disuguaglianza.
\end{enumerate}
\section{Disequazioni di primo grado}
\label{sec:Disequuazionidiprimogrado}
\begin{definizionet}{}{}
Una disequazione\index{Disequazione} è una diseguaglianza\index{Disuguaglianza} in cui compare un'incognita.
\end{definizionet}
\begin{definizionet}{Forma normale}{}
	Una disequazione di primo grado è in forma normale\index{Disequazione!forma normale} se è scritta in una di queste forme
\begin{equation}
ax\left\{ \begin{aligned}
<b\\
\leq b\\
\geq b\\
>b
\end{aligned}\right .   
\end{equation}
\end{definizionet}
La disequazione è una disuguaglianza che è vera o falsa a seconda dei valori che sostituiamo all'incognita.
Il segno o verso di diseguaglianza divide la disequazione in due parti:il membro sinistro e quello destro.\par Una disequazione può essere o intera\index{Disequazione!intera} o frazionaria\index{Disequazione!frazionaria}, è intera se l'incognita non si trova mai al denominatore, è frazionaria se compare anche al denominatore.
\[\centering
\begin{array}{cc}
\toprule
\mathbf{Intere}  & 3x+5<2x+4  \\ [.25cm] 
  &\dfrac{3}{4}x<\dfrac{5}{2}+x+1  \\ [.25cm]
\mathbf{Frazionarie}  &\dfrac{3x+1}{2x+1}>0  \\ [.25cm]
 &\dfrac{3x+1}{x}>\dfrac{1}{2}+\dfrac{1}{2x+1}  \\ [.25cm]
\bottomrule
\end{array} 
\]
\subsection{Risolvere una disequazione di primo grado}
Per risolvere una disequazione bisogna avere chiaro cosa si intende per soluzione\index{Disequazione!soluzione}
\begin{definizionet}{Soluzione}{}
Una soluzione\index{Disequazione!soluzione} per una disequazione è un valore che sostituito all'incognita rende vera la disuguaglianza
\end{definizionet}

La definizione sembra simile a quella per le equazioni. Per un'equazione abbiamo: <<una soluzione è quel valore che rende vera l'uguaglianza>>\par
La somiglianza è solo apparente, infatti per un'equazione di primo grado in un incognita, la soluzione è un valore, per una disequazione la soluzione è un intervallo. Per esempio la disequazione elementare$X>1$ ha per soluzione tutti i numeri che sono maggiori di uno cioè l'intervallo $]1 +\infty [$.\par
Il metodo per risolvere una disequazione di primo è simile a quello per risolvere una equazione di pari grado, cioè la separazione delle variabili\index{Separazione!variabili}.\par Un esempio è il seguente. Supponiamo di dover risolvere 
\begin{esempiot}{Disequazione di primo grado}{}
\begin{equation}
3x+5<2x+6\label{equ:PrimoGradoDisequazione1}
\end{equation}
\end{esempiot}
 procediamo come nella figura\nobs\vref{fig:esempioDisequazioniPgrado1}
\begin{figure}
	\begin{NodesList}
		\centering
		\begin{align*}
			3x+5<&2x+6\AddNode\\[.5cm] 
			3x+5-2x<&6\AddNode\\[.5cm] %\AddNode[2]\\ 
			3x-2x<&6-5\AddNode\\
			x<&1\AddNode
		\end{align*}
		\LinkNodes[margin=6cm]{\begin{minipage}[h]{5cm}
				Sposto $2x$ a sinistra e cambio di segno
			\end{minipage}}
			%\LinkNodes{Sposto $2x$ a sinistra e cambio di segno}%
			\LinkNodes[margin=6cm]{\begin{minipage}[h]{5cm}
					Sposto $+5$ a destra e cambio di segno
				\end{minipage}}%
				\LinkNodes[margin=6cm]{\begin{minipage}[h]{5cm}
						Sommo
					\end{minipage}}%
				\end{NodesList}
		\captionsetup{format=esempio,list=no}
	\caption{Risoluzione disequazione\nobs\vref{equ:PrimoGradoDisequazione1}}
	\label{fig:esempioDisequazioniPgrado1}
\end{figure}

Il procedimento è  quello della risoluzione di un'equazione di primo grado, si trasportano a sinistra i valori con l'incognita, a destra i numeri, vale la stessa regola che si usa per le equazioni: spostando i termini rispettto al verso, si cambia di segno. Per rappresentare la soluzione si usa un metodo grafico che rappresenta le soluzioni. Il grafico dell'esempio è la figura\nobs\vref{fig:esempioDisequazioniPgradografico1}. Per disegnare il grafico della soluzione si procede in questa maniera: 
\begin{procedurat}{}{}
\begin{enumerate}
	\item si traccia una linea orizzontale orientata, l'asse dei numeri.
	\item si mette sotto di essa la soluzione trovata.
	\item in corrispondenza della soluzione si traccia un segmento verticale.
	\item  si guarda la soluzione e dalla parte superiore del segmento si traccia una semiretta continua nella direzione della freccia e una semiretta tratteggiata dal lato opposto.
\end{enumerate}
\end{procedurat}
\begin{figure}{I}{0pt}
	\centering
	\begin{tikzpicture}
	\draw[ -triangle 90](0,0)--(5,0);
	\draw(2,0)--(2,1);
	%%%%%soluzioni
	%%%sinistra	
	\draw[dashed](2,1)--(5,1);
	%%destra
	\draw(2,1)--(0,1);
	\node at (2,-0.5) {1};
	\end{tikzpicture}
	\captionsetup{format=grafico,list=no}
	\caption[]{Disequazione\nobs\vref{equ:PrimoGradoDisequazione1}}
	\label{fig:esempioDisequazioniPgradografico1}
\end{figure}\par Un caso leggermente più complesso è l'esempio
\begin{esempiot}{Disequazione di primo grado}{}
\begin{equation}
 3x+2\geq\dfrac{1}{2}x+3\label{equ:PrimoGradoDisequazione2}
\end{equation}
\end{esempiot}
  che vene risolto nella figura\nobs\vref{fig:esempioDisequazioniPgrado2} qui vi è un termine frazionario che può essere facilmente tolto moltiplicando entrambi i lati della disuguaglianza per il denominatore della frazione. Fatto ciò, si procede separando le incognite e sommando i termini. Al termine basta solo dividere per il numero davanti l'incognita e ottenere così il risultato. L'importante è notare che avendo moltiplicato e diviso per termini positivi, il verso della disequazione non cambia. Il grafico della disequazione è quello della figura\nobs\vref{fig:esempioDisequazioniPgradografico2} In questo caso, dato che la soluzione prevede un maggiore o uguale nel grafico è inserito un pallino $\bullet$ per indicare che il valore $\dfrac{2}{3}$ è compreso fra le soluzioni.\par
Supponiamo di dover risolvere 
\begin{esempiot}{Disequazione di primo grado}{}
\begin{equation}
3x+2>4x+3\label{equ:PrimoGradoDisequazione3}
\end{equation}
\end{esempiot}
 l'esempio\nobs\vref{fig:esempioDisequazioniPgrado3} è minimo, tuttavia nell'ultimo passaggio è importante ricordarsi che cambiando di segno si cambia di verso della diseguaglianza  avendo in questo caso moltiplicato per un termine negativo.\par
Il grafico della soluzione è il grafico\nobs\vref{fig:esempioDisequazioniPgrado3}.  
\begin{figure}
\begin{NodesList}
\centering
\begin{align*}
	3x+2\geq&\dfrac{1}{2}x+3\AddNode\\
	2(3x+2)\geq&2(\dfrac{1}{2}x+3)\AddNode\\ %[.5cm] %\AddNode[2]\\ 
	6x+4\geq&x+6\AddNode\\
	6x-x\geq&6-4\AddNode\\
	5x\geq&2\AddNode\\
	x\geq&\dfrac{2}{5}\AddNode
\end{align*}
\LinkNodes[margin=6cm]{\begin{minipage}[h]{5cm}
Moltiplico per $2x$ ed elimino la frazione
\end{minipage}}
\LinkNodes[margin=6cm]{\begin{minipage}[h]{5cm}
Semplifico
\end{minipage}}%
\LinkNodes[margin=6cm]{\begin{minipage}[h]{5cm}
Sposto $x$ e $4$ cambiando di segno
\end{minipage}}%
\LinkNodes[margin=6cm]{\begin{minipage}[h]{5cm}
Sommo
\end{minipage}}%
\LinkNodes[margin=6cm]{\begin{minipage}[h]{5cm}
Divido
\end{minipage}}%
\end{NodesList}
\captionsetup{format=esempio,list=no}\caption{Risoluzione disequazione\nobs\vref{equ:PrimoGradoDisequazione2}}
\label{fig:esempioDisequazioniPgrado2}
\end{figure}
\begin{figure}
	\centering
	\begin{tikzpicture}
	\draw[ -triangle 90](0,0)--(5,0);
	\draw(2,0)--(2,1);
	%%%%%soluzioni
	%%%sinistra	
	\draw(2,1)--(5,1);
	%%destra
	\draw[dashed](2,1)--(0,1);
	%%pallino
	\node at (2,1) {$\bullet$};
	\node at (2,-0.5) {$\dfrac{2}{5}$};
	\end{tikzpicture}
	\captionsetup{format=grafico,list=no}
	\caption{Disequazione\nobs\vref{equ:PrimoGradoDisequazione2}}
	\label{fig:esempioDisequazioniPgradografico2}
\end{figure}
\begin{figure}
	\begin{NodesList}
		\centering
		\begin{align*}
			3x+2>&4x+3\AddNode\\
			3x-4x>&3-2\AddNode\\
			-x>&1\AddNode\\
			x<&-1\AddNode
		\end{align*}
		%\LinkNodes{Sposto $2x$ a sinistra e cambio di segno}%
		\LinkNodes[margin=6cm]{\begin{minipage}[h]{5cm}
				Sposto $4x$ e $+3$ cambiando di segno
			\end{minipage}}%
			\LinkNodes[margin=6cm]{\begin{minipage}[h]{5cm}
					Sommo
				\end{minipage}}%
				\LinkNodes[margin=6cm]{\begin{minipage}[h]{5cm}
						Cambio di segno e di verso
					\end{minipage}}%
				\end{NodesList}
	\captionsetup{format=esempio,list=no}
	\caption{Risoluzione disequazione\nobs\vref{equ:PrimoGradoDisequazione3}}
	\label{fig:esempioDisequazioniPgrado3}
\end{figure}
\begin{figure}
	\centering
	\begin{tikzpicture}
	\draw[ -triangle 90](0,0)--(5,0);
	\draw(2,0)--(2,1);
	%%%%%soluzioni
	%%%sinistra	
	\draw[dashed](2,1)--(5,1);
	%%destra
	\draw(2,1)--(0,1);
	\node at (2,-0.5) {-1};
	\end{tikzpicture}
	\captionsetup{format=grafico,list=no}
	\caption{Disequazione\nobs\vref{equ:PrimoGradoDisequazione3}}
	\label{fig:esempioDisequazioniPgradografico3}
\end{figure}
\subsection{Classificare le soluzioni}
La figura\nobs\vref{fig:DidequazioniEsoluzioni} riassume la classificazione delle soluzioni per una disequazione. Possiamo avere tre casi se dopo le semplificazioni otteniamo:
\begin{enumerate}
	\item se otteniamo un risultato  del tipo $x<3$ diremo che la soluzione ottenuta è determinata\index{Soluzione!determianta}.
	\item se otteniamo un risultato del tipo $2<3$ diremo che la soluzione ottenuta è sempre verificata\index{Soluzione!indeterminata}. \'E sempre vera, e non dipende dall'incognita. 
	\item se otteniamo un risultato del tipo $5<2$ diremo che la soluzione ottenuta è impossibile\index{Soluzione!impossibile}. \'E sempre falsa, e non dipende dall'incognita.
\end{enumerate}
\section{Disequazioni frazionarie o prodotti di primo grado}
\label{DisequazioniFrazionarieProdottiPrimoGrado}
\subsection{Prodotti}
Cominciamo a introdurre il problema con un esempio
\begin{esempiot}{Disequazioni di primo grado prodotti}{}
	\begin{equation}
(x-5)(2-3x)<0\label{equ:ProdDis1}
\end{equation}
\end{esempiot}
La disequazione~\vref{equ:ProdDis1} chiede quando il prodotto\index{Disequazione!prodotto} di due binomi è negativo.  Per ottenere il segno di un prodotto bisogna conoscere il segno dei fattori\index{Fattori!segno} e li applicare la regola dei segni.\par Qui bisogna aprire una premessa. Come si è detto quando si risolve una disequazione di primo grado è possibile associare alla disequazione un grafico che esprime quando la disequazione è vera. Per esempio, banalmente, la disequazione
\begin{equation}
x-2\leq 0\label{equ:esempioDisequazioniPgradografico4}
\end{equation} ha soluzione $x\leq 2$ a cui corrisponde il grafico\nobs\vref{fig:esempioDisequazioniPgradografico4}
\begin{figure}
	\centering
	\begin{tikzpicture}
	\draw[ -triangle 90](0,0)--(5,0);
	\draw(2,0)--(2,1);
	\draw[dashed](2,1)--(5,1);
	\draw(2,1)--(0,1);
	\node at (2,1) {$\bullet$};
	\node at (2,-0.5) {2};
	\end{tikzpicture}
		\captionsetup{format=grafico,list=no}
	\caption{Disequazione\nobs\vref{equ:esempioDisequazioniPgradografico4}}
	\label{fig:esempioDisequazioniPgradografico4}
\end{figure}\par Leggendo il grafico vediamo che  per valori minori di $x$  minori di $2$ la disequazione è vera. Possiamo che $x-2$ è negativo per valori minori di due dell'incognita, positivo per valori maggiori di due e che vale zero per $x=2$. Se la disequazione è 
\begin{equation}
x-2\geq 0\label{equ:esempioDisequazioniPgradografico5}
\end{equation}
otteniamo il grafico\nobs\vref{fig:esempioDisequazioniPgradografico5}
\begin{figure}
	\centering
		\begin{tikzpicture}
		\draw[ -triangle 90](0,0)--(5,0);
		\draw(2,0)--(2,1);
		\draw(2,1)--(5,1);
		\draw[dashed](2,1)--(0,1);
		\node at (2,1) {$\bullet$};
		\node at (2,-0.5) {2};
		\end{tikzpicture}
	\captionsetup{format=grafico,list=no}
	\caption{Disequazione\nobs\vref{equ:esempioDisequazioniPgradografico5}}
	\label{fig:esempioDisequazioniPgradografico5}
\end{figure}

Il grafico ottenuto è l'opposto del precedente. La linea continua ci dice quando è vero che $x-2$ è positivo. Riflettendoci un po, questo grafico ci dice che $x-2$ è negativo per $x$ minore di due, positivo per valori maggiori di due e che vale zero per $x=2$.\par Il secondo grafico quindi, letto in maniera opportuna, ci da la soluzione anche per la precedente disequazione. Ora, per convenzione, si considera quindi che alla linea continua corrispondano valori positivi, mentre alla linea tratteggiata  valori negativi. Per evitare ambiguità  si usa per costruire il grafico che tutte le disequazioni siano del secondo tipo cioè maggiori o maggior uguale a zero.\par Costruito questo lo si legge secondo le disuguaglianze di partenza. Praticamente se devo risolvere $x-2\leq 0$ procedo in questo modo
\begin{enumerate}
	\item Risolvo  $x-2\geq 0$
	\item Costruisco il grafico\nobs\vref{fig:esempioDisequazioniPgradografico5}
	\item Dato che la disequazione generale chiede quando deve essere minore o uguale a zero, scrivo la soluzione $x\leq 0$
\end{enumerate}
  
Ritorniamo alla disequazione~\vref{equ:ProdDis1}. La disequazione è un prodotto e voglio sapere quando  è negativo. Per conoscere il segno di un prodotto bisogna conoscere il segno dei fattori che lo compongono.  Per comodità e quanto detto prima, sostituisco la disequazione con la disequazione~\vref{equ:ProdDis2} che spezzo nelle due disequazioni~\vref{equ:ProdDis2a} e~\vref{equ:ProdDis2b} 
% \begin{subequations}
% 	\begin{align}
% 	(x-5)(2-3x)>0\label{equ:ProdDis2}
% 	\intertext{formata dalla disequazione}	
% 	(x-5)>0\label{equ:ProdDis2a}
% 	\intertext{e dalla disequazione}
% 	(2-3x)>0\label{equ:ProdDis2b}
% 	\end{align}
% \end{subequations}
\begin{subequations}
	\begin{equation}
	(x-5)(2-3x)>0\label{equ:ProdDis2}
	\end{equation}
	formata dalla disequazione
	\begin{equation}
	(x-5)>0\label{equ:ProdDis2a}
	\end{equation}
	e dalla disequazione
	\begin{equation}
	(2-3x)>0\label{equ:ProdDis2b}
	\end{equation}
\end{subequations}
Risolvo la disequazione\nobs\vref{equ:ProdDis2a}. La disequazione ha per soluzione $x>5$ e per grafico\nobs\vref{fig:ProdottoDis2a}
\begin{figure}
	\centering
	\begin{subfigure}[b]{.4\linewidth}
		\begin{tikzpicture}
		\draw[ -triangle 90](0,0)--(5,0);
		\draw(2,0)--(2,1);
		\draw(2,1)--(5,1);
		\draw[dashed](2,1)--(0,1);
		%\node at (2,1) {$\bullet$};
		\node at (2,-0.5) {$5\vphantom{\dfrac{2}{3}}$};
		%\node at (2,-0.5) {5};
		\end{tikzpicture}
%		\renewcommand\thesubfigure{Grafico \thefigure\alph{subfigure}}
		\caption{Disequazione\nobs\vref{equ:ProdDis2a}}
		\label{fig:ProdottoDis2a}
	\end{subfigure}%
	\centering
	\begin{subfigure}[b]{.4\linewidth}
		\centering
		\begin{tikzpicture}
		\draw[ -triangle 90](0,0)--(5,0);
		\draw(2,0)--(2,1);
		\draw[dashed](2,1)--(5,1);
		%\node at (2,1) {$\bullet$};
		\draw(2,1)--(0,1);
		\node at (2,-0.5) {$\dfrac{2}{3}$};
		\end{tikzpicture}
	%	\renewcommand\thesubfigure{Grafico \thefigure\alph{subfigure}}
		\caption{Disequazione\nobs\vref{equ:ProdDis2b}}
		\label{fig:ProdottoDis2b}
	\end{subfigure}%
		\qquad\qquad\centering
		\begin{subfigure}[b]{.4\linewidth}
			\centering
				\begin{tikzpicture}
				\draw[ -triangle 90](0,0)--(5,0);
				\draw(2,0)--(2,1);
				\draw[dashed](2,1)--(5,1);
				%\node at (2,1) {$\bullet$};
				\draw(2,1)--(0,1);
				\node at (2,-0.5) {$\dfrac{2}{3}$};
				\draw(3,0)--(3,2);
				\draw(3,2)--(5,2);
				\draw[dashed](3,2)--(0,2);
				\node at (3,-0.5) {$5$};
				%\node at (3,2) {$\bullet$};
				\end{tikzpicture}
		%	\renewcommand\thesubfigure{Grafico \thefigure\alph{subfigure}}
			\caption{Disequazione\nobs\vref{equ:ProdDis2}}
			\label{fig:ProdottoDis2c}
		\end{subfigure}%
		\captionsetup{format=grafico,list=no}
	\caption{Disequazione\nobs\vref{equ:ProdDis2}}
\end{figure}

Mentre la  disequazione\nobs\vref{equ:ProdDis2a} ha per soluzione $x<\dfrac{2}{3}$ con grafico\nobs\vref{fig:ProdottoDis2b}
 
Interessante è il grafico\nobs\vref{fig:ProdottoDis2c} che riunisce i due precedenti. Nel grafico, l'asse delle $x$ è diviso in tre parti, prima di $\dfrac{2}{3}$, fra $\dfrac{2}{3}$ e $5$ e dopo il $5$. Prima di $\dfrac{2}{3}$, guardando il grafico,  è positiva la disequazione\nobs\vref{equ:ProdDis2b} (linea continua)  ed è negativa la disequazione\nobs\vref{equ:ProdDis2a} (linea tratteggiata). Quindi il loro prodotto è negativo. Per valori compresi fra $\dfrac{2}{3}$ e $5$ entrambe le disequazioni  sono negative, quindi il loro prodotto è positivo. Dopo il $5$ è positiva\nobs\vref{equ:ProdDis2a} ed è negativa\nobs\vref{equ:ProdDis2b}. 

Per rispondere finalmente, alla disequazione\nobs\vref{equ:ProdDis1} basta leggere il grafico precedente e vedere che il segno del prodotto è negativo per valori di $x<\dfrac{2}{3}$ e per valori di $x>5$

Un altro esempio risolviamo passo passo la disequazione
\begin{esempiot}{Disequazione prodotto}{}
\begin{equation}
(2-x)(3x-1)(\dfrac{2}{3}-x)\leq 0\label{equ:ProdDis3}
\end{equation}
\end{esempiot}
suddivido la disequazione in tre parti che verranno risolte a parte.
%
\begin{subequations}
	\begin{equation}
	(2-x)\geq 0\label{equ:ProdDis3a}
	\end{equation}
	\begin{equation}
	(3x-1)\geq 0\label{equ:ProdDis3b}
	\end{equation}
	\begin{equation}
	(\dfrac{2}{3}-x)\geq 0\label{equ:ProdDis3c}
	\end{equation}
\end{subequations}
Iniziamo con il risolvere  la disequazione\nobs\vref{equ:ProdDis3a}. Utilizzando  il procedimento\nobs\vref{svo:ProDis3a} e otteniamo il grafico\nobs\vref{graf:ProDis3a}. Continuiamo con la disequazione\nobs\vref{equ:ProdDis3b} dal  procedimento\nobs\vref{svo:ProDis3b} si ha il grafico\nobs\vref{graf:ProDis3b}. Terminiamo  con la disequazione\nobs\vref{equ:ProdDis3c} dal  procedimento\nobs\vref{svo:ProDis3c} si ottiene il grafico\nobs\vref{graf:ProDis3c}. Non resta che riunire i tre grafici nel grafico\nobs\vref{equ:ProdDis3c}. L'asse delle $x$ è suddiviso in quattro parti. Prima di $\dfrac{1}{3}$, tra $\dfrac{1}{3}$ e $\dfrac{2}{3}$, tra $\dfrac{2}{3}$ e $2$ ed infine dopo $2$. Prima di $\dfrac{1}{3}$ abbiamo due linee continue ed una tratteggiata quindi $(+)\cdot(+)\cdot(-)=-$. Tra $\dfrac{1}{3}$ e $\dfrac{2}{3}$ abbiamo tre linee continue $(+)\cdot(+)\cdot(+)=+$. Tra $\dfrac{2}{3}$ e $2$ abbiamo una linea continua, una tratteggiata e una linea continua. Dopo $2$ abbiamo due linee tratteggiate ed una continua $(-)\cdot(-)\cdot(+)=+$.  La disequazione\nobs\vref{equ:ProdDis3} chiede quando il prodotto è negativo, riguardando quello che si è detto, la risposta è $x\leq \dfrac{1}{3}$ e $\dfrac{2}{3}\leq x \leq 2$.
\begin{figure}
	\centering
	\begin{subfigure}[]{\linewidth}
		\begin{NodesList}
			\begin{align*}
				2-x\geq 0&\AddNode\\%
				-x\geq-2&\AddNode\\%
				x\leq 2&\AddNode%
			\end{align*}
			\LinkNodes[margin=6cm]{}%
			\LinkNodes[margin=6cm]{}%
		\end{NodesList}
	%	\renewcommand\thesubfigure{Svolgimento \thefigure\alph{subfigure}}
		\caption{Risoluzione disequazione}
		\label{svo:ProDis3a}
	\end{subfigure}%
	\qquad
	\begin{subfigure}[]{\linewidth}
		\centering
		\begin{tikzpicture}
		\draw[ -triangle 90](0,0)--(5,0);
		\draw(2,0)--(2,1);
		\draw[dashed](2,1)--(5,1);
		\node at (2,1) {$\bullet$};
		\draw(2,1)--(0,1);
		\node at (2,-0.5) {$2$};
		\end{tikzpicture}
	%	\renewcommand\thesubfigure{Grafico \thefigure\alph{subfigure}}
		\caption{Grafico disequazione}
		\label{graf:ProDis3a}
	\end{subfigure}%
	\captionsetup{format=esempio,list=no}
	\caption{Disequazione\nobs\vref{equ:ProdDis3a} }
	\label{esempio:ProDisa3a}
\end{figure} 
\begin{figure}
	\centering
	\begin{subfigure}[]{\linewidth}
		\begin{NodesList}
			\centering
			\begin{align*}
				3x-1\geq& 0\AddNode\\
				3x\geq&1\AddNode\\
				x\geq&\dfrac{1}{3}\AddNode
			\end{align*}
			%\LinkNodes{Sposto $2x$ a sinistra e cambio di segno}%
			\LinkNodes[margin=6cm]{}%
			\LinkNodes[margin=6cm]{}%
		\end{NodesList}
		\caption{Risoluzione disequazione}
		\label{svo:ProDis3b}
	\end{subfigure}%
	\qquad
	\begin{subfigure}[]{\linewidth}
		\centering
		\begin{tikzpicture}
		\draw[ -triangle 90](0,0)--(5,0);
		\draw(2,0)--(2,1);
		\draw(2,1)--(5,1);
		\node at (2,1) {$\bullet$};
		\draw[dashed](2,1)--(0,1);
		\node at (2,-0.5) {$\dfrac{1}{3}$};
		\end{tikzpicture}
		\caption{Grafico disequazione}
		\label{graf:ProDis3b}
	\end{subfigure}%
	\captionsetup{format=esempio,list=no}
	\caption{Disequazione\nobs\vref{equ:ProdDis3b}}
	\label{esempio:ProDisa3b}
	\end{figure}
\begin{figure}
	\centering
	\begin{subfigure}[]{\linewidth}
		\begin{NodesList}
			\centering
			\begin{align*}
				\dfrac{2}{3}-x\geq& 0\AddNode\\
				-x\geq&-\dfrac{2}{3}\AddNode\\
				x\leq& \dfrac{2}{3}\AddNode
			\end{align*}
			%\LinkNodes{Sposto $2x$ a sinistra e cambio di segno}%
			\LinkNodes[margin=6cm]{}%
			\LinkNodes[margin=6cm]{}%
		\end{NodesList}
		\caption{Risoluzione disequazione}
		\label{svo:ProDis3c}
	\end{subfigure}%
	\qquad
	\begin{subfigure}[]{\linewidth}
		\centering
		\begin{tikzpicture}
		\draw[ -triangle 90](0,0)--(5,0);
		\draw(2,0)--(2,1);
		\draw[dashed](2,1)--(5,1);
		\node at (2,1) {$\bullet$};
		\draw(2,1)--(0,1);
		\node at (2,-0.5) {$\dfrac{2}{3}$};
		\end{tikzpicture}
		\caption{Grafico disequazione}
		\label{graf:ProDis3c}
	\end{subfigure}%
	\captionsetup{format=esempio,list=no}
	\caption{Disequazione\nobs\vref{equ:ProdDis3c}}
	\label{esempio:ProDisa3c}
\end{figure}
\begin{figure}
	\centering
		\begin{tikzpicture}
		\draw[ -triangle 90](0,0)--(6,0);
		\draw(2,0)--(2,1);
		\draw(2,1)--(6,1);
		\node at (2,1) {$\bullet$};
		\draw[dashed](2,1)--(0,1);
		\node at (2,-0.5) {$\dfrac{1}{3}$};
		\draw(3,0)--(3,2);
		\draw[dashed](3,2)--(6,2);
		\draw(3,2)--(0,2);
		\node at (3,-0.5) {$\dfrac{2}{3}$};
		\node at (3,2) {$\bullet$};
		\draw(4,0)--(4,3);
		\draw[dashed](4,3)--(6,3);
		\draw(4,3)--(0,3);
		\node at (4,-0.5) {$2$};
		\node at (4,3) {$\bullet$};
		\end{tikzpicture}
	\captionsetup{format=grafico,list=no}
	\caption{Disequazione\nobs\vref{equ:ProdDis3}}
	\label{graf:ProDis3d}
\end{figure}
\subsection{Frazioni}
Iniziamo con il definire una disequazione frazionaria di primo grado in forma normale.
\begin{definizionet}{Disequazione frazionaria di primo grado}{}
Una disequazione frazionaria\index{Disequazione!frazionaria} è una disequazione del tipo 
\begin{equation}
\dfrac{ax+b}{cx+d}\left\{ \begin{aligned}
<0\\
\leq 0\\
\geq 0\\
>0
\end{aligned}\right .   
\end{equation}
\end{definizionet}
Supponiamo di dover risolvere
\begin{esempiot}{Disequazione fratta}{}
 \begin{equation}
\dfrac{3x+1}{1-x}\leq 0\label{equ:DisFrazPrimoG1}
\end{equation}
\end{esempiot}
Una disequazione frazionaria si risolve come le precedenti disequazioni. Viene anche qui usata la regola dei segni. Si parte dal segno del denominatore e si confronta con il segno del numeratore.\par Solo un appunto, prima di procedere con la risoluzione della disequazione bisogna ricordarsi che una disequazione frazionaria è una frazione  e una frazione esiste se il suo denominatore è diverso da zero. In questo caso la frazione esiste se $1-x$ non vale zero. Qui è evidente che $1-x$ è zero se $x=1$.\par Il procedimento è quello solito,suddivido la frazione nelle sue parti e risolvo due disequazioni, anche qui cercando valori positivi. Avremo
\begin{subequations}
	\begin{equation}
	3x+1\geq 0\label{equ:DisFrazPrimoG1a} 
	\end{equation}
\begin{equation}
1-x> 0\label{equ:DisFrazPrimoG1b} 
\end{equation}
\end{subequations}   
Iniziamo con il risolvere la disequazione\nobs\vref{equ:DisFrazPrimoG1a}. Dallo svolgimento\nobs\vref{svo:DisFrazPrimoG1a} otteniamo il grafico\nobs\vref{graf:DisFrazPrimoG1a}. 
\begin{figure}
	\centering
	\begin{subfigure}[]{\linewidth}
		\begin{NodesList}
			\centering
			\begin{align*}
				3x+1\geq& 0\AddNode\\
				3x\geq&-1\AddNode\\
				x\geq&-\dfrac{1}{3}\AddNode
			\end{align*}
			%\LinkNodes{Sposto $2x$ a sinistra e cambio di segno}%
			\LinkNodes[margin=6cm]{}%
			\LinkNodes[margin=6cm]{}%
		\end{NodesList}
		\caption{Risoluzione disequazione}
		\label{svo:DisFrazPrimoG1a}
	\end{subfigure}%
	\qquad
	\begin{subfigure}[]{\linewidth}
		\centering
		\begin{tikzpicture}
		\draw[ -triangle 90](0,0)--(5,0);
		\draw(2,0)--(2,1);
		\draw(2,1)--(5,1);
		\node at (2,1) {$\bullet$};
		\draw[dashed](2,1)--(0,1);
		\node at (2,-0.5) {$-\dfrac{1}{3}$};
		\end{tikzpicture}
		\caption{Grafico disequazione}
		\label{graf:DisFrazPrimoG1a}
	\end{subfigure}%
	\captionsetup{format=esempio,list=no}
	\caption{Disequazione\nobs\vref{equ:DisFrazPrimoG1a}}
	\label{esempio:DisFrazPrimoG1a}
\end{figure}
     \begin{figure}
	\centering
	\begin{tikzpicture}
	\draw[ -triangle 90](0,0)--(5,0);
	\draw(2,0)--(2,1);
	\draw(2,1)--(5,1);
	\node at (2,1) {$\bullet$};
	\draw[dashed](2,1)--(0,1);
	\node at (2,-0.5) {$-\dfrac{1}{3}$};
	\draw(3,0)--(3,2);
	\draw[dashed] ( 3,2)--(5,2);
	\draw(3,2)--(0,2);
	\node at (3,-0.5) {$1$};
	%\node at (3,2) {$\bullet$};
	\end{tikzpicture}
	\captionsetup{format=esempio,list=no}
	\caption{Disequazione\nobs\vref{equ:DisFrazPrimoG1}}
	\label{graf:DisFrazPrimoG1}
\end{figure}
Continuiamo con la disequazione\nobs\vref{equ:DisFrazPrimoG1b}.
Questa disequazione è differente dalla precedente perché si passa da un maggiore o uguale a zero ad un maggiore di zero. Infatti tale termine corrisponde al denominatore della frazione e un denominatore non può essere mai uguale a zero. Anche 
qui dallo svolgimento\nobs\vref{svo:DisFrazPrimoGìb} otteniamo il grafico\nobs\vref{graf:DisFrazPrimoG1b}. Unendo i due grafici otteniamo il grafico\nobs\vref{graf:DisFrazPrimoG1}. Abbiamo tre zone prima di $-\dfrac{1}{3}$, fra $-\dfrac{1}{3}$ e $1$  e dopo $1$. Prima di $-\dfrac{1}{3}$ abbiamo una linea continua ed una linea tratteggiata quindi $(+)\cdot(-)=-$. Fra $-\dfrac{1}{3}$ e $1$ abbiamo due linee continue quindi $(+)\cdot(+)=+$. Dopo $1$ abbiamo una linea tratteggiata ed una continua quindi $(-)\cdot(+)=-$. La disequazione\nobs\vref{equ:DisFrazPrimoG1} chiede quando il rapporto è negativo e guardando i precedenti risultati la risposta è $x\leq -\dfrac{1}{3}$ e $x>1$, maggiore e non uguale perché riferita al denominatore.  

\begin{figure}
	\centering
	\begin{subfigure}[]{\linewidth}
		\begin{NodesList}
			\centering
			\begin{align*}
				1-x>& 0\AddNode\\
				-x>&-1\AddNode\\
				x<&1\AddNode
			\end{align*}
			%\LinkNodes{Sposto $2x$ a sinistra e cambio di segno}%
			\LinkNodes[margin=6cm]{}%
			\LinkNodes[margin=6cm]{}%
		\end{NodesList}
		\caption{Risoluzione disequazione}
		\label{svo:DisFrazPrimoGìb}
	\end{subfigure}%
	\qquad
	\begin{subfigure}[]{\linewidth}
		\centering
		\begin{tikzpicture}
		\draw[ -triangle 90](0,0)--(5,0);
		\draw(2,0)--(2,1);
		\draw[dashed](2,1)--(5,1);
		\node at (2,1) {$\bullet$};
		\draw(2,1)--(0,1);
		\node at (2,-0.5) {$1$};
		\end{tikzpicture}
		\caption{Grafico disequazione}
		\label{graf:DisFrazPrimoG1b}
	\end{subfigure}%
	\captionsetup{format=esempio,list=no}
	\caption{Disequazione\nobs\vref{equ:DisFrazPrimoG1b}}
	\label{esempio:DisFrazPrimoG1b}
\end{figure}

Un esempio più complesso è il seguente
\begin{esempiot}{Disequazione fratta}{}
\begin{equation}
\dfrac{1}{x+2}\geq-\dfrac{3}{2x+1}\label{equ:DisFrazPrimoG2}
\end{equation}
\end{esempiot}

La disequazione è diversa dalle precedenti, abbiamo due termini frazionari quindi prima di risolverla bisognerà prima discuterla poi semplificarla e quindi  risolverla. L'esempio\nobs\vref{esempio:DisFrazPrimoG2a} mostra quanto detto. 

Resta da risolvere la disequazione
\begin{equation}
\dfrac{5x+7}{(x+2)(2x+1)}\geq 0 \label{equ:DisFrazPrimoG2a}
\end{equation}
Questa disequazione è formata da tre parti 
\begin{subequations}
	\begin{equation}
	5x+7\geq 0\label{equ:DisFrazPrimoG2aP1}
	\end{equation}
	\begin{equation}
	x+2>0\label{equ:DisFrazPrimoG2aP2}
	\end{equation}
	\begin{equation}
	2x+1> 0\label{equ:DisFrazPrimoG2aP3}
	\end{equation}
\end{subequations}
Avremo i seguenti risultati che ci permettono di costruire il grafico\nobs\vref{equ:DisFrazPrimoG2} della disequazione. 
\begin{align*}
	5x+7\geq& 0 & x\geq&-\dfrac{7}{5}\\
	x+1>&0&x>&-1\\
	2x+1>&0&x>&-\dfrac{1}{2}
\end{align*}
\begin{figure}
	\centering
	\begin{tikzpicture}
	\draw[ -triangle 90](0,0)--(6,0);
	\draw(2,0)--(2,1);
	\draw(2,1)--(6,1);
	\node at (2,1) {$\bullet$};
	\draw[dashed](2,1)--(0,1);
	\node at (2,-0.5) {$-\dfrac{7}{5}$};
	\draw(3,0)--(3,2);
	\draw(3,2)--(6,2);
	\draw[dashed](3,2)--(0,2);
	\node at (3,-0.5) {$-1$};
	%\node at (3,2) {$\bullet$};
	\draw(4,0)--(4,3);
	\draw(4,3)--(6,3);
	\draw[dashed](4,3)--(0,3);
	\node at (4,-0.5) {$-\dfrac{1}{2} $};
	%\node at (4,3) {$\bullet$};
	\end{tikzpicture}
	\captionsetup{format=grafico,list=no}
	\caption[]{Disequazione\nobs\vref{equ:DisFrazPrimoG2}}
	\label{graf:DisFrazPrimoG2}
\end{figure}
\begin{figure}
		\centering
\begin{minipage}{\linewidth}
\begin{NodesList}
	\begin{align*}
		\dfrac{1}{x+2}\geq-\dfrac{3}{2x+1}&\AddNode\AddNode[2]\AddNode[3]\\
		&\\
		\left .\begin{aligned}
			 x+2= 0&\\
			 x=-2&\\
			 x\neq 2&\\
			 2x+1= 0&\\
			 2x=-1&\\
			 x=-\dfrac{1}{2}&\\
			 x\neq-\dfrac{1}{2}&& 
		\end{aligned}\right\}&\qquad\text{Discussione}\AddNode\\
		&\\
		\left .\begin{aligned}
			\dfrac{2x+1\geq -3(x+2)}{(x+2)(2x+1)}&\\
			\dfrac{2x+1\geq -3x-6}{(x+2)(2x+1)}&\\
			\dfrac{3x+2x+1+6}{(x+2)(2x+1)}\geq 0&\\
			\dfrac{5x+7}{(x+2)(2x+1)}\geq 0 &&
		\end{aligned}\right\}&\qquad \text{Semplificazione }\AddNode[2]\\
		&\\
	\left .\begin{aligned}
		\dfrac{5x+7}{(x+2)(2x+1)}\geq 0 &&
	\end{aligned}\right\}& \qquad \text{Risoluzione}\AddNode[3]
\end{align*}
	{\tikzset{ArrowStyle/.style={>=triangle 90,->}}
	\tikzset{LabelStyle/.style = {left=0.1cm,pos=.4,text=red}}
	\LinkNodes{}%
	\LinkNodes{}
\LinkNodes{}} %
\end{NodesList}
\end{minipage}
	\captionsetup{format=esempio,list=no}
	\caption{Disequazione\nobs\vref{equ:DisFrazPrimoG2}}
	\label{esempio:DisFrazPrimoG2a}
\end{figure}
Le    
disequazioni\nobs\vrefrange{equ:DisFrazPrimoG2aP2}{equ:DisFrazPrimoG2aP3} sono maggiori di zero e non maggiori e uguali a zero perché sono nel denominatore della frazione. Il grafico della disequazione è diviso in quattro parti: prima di $x<-\dfrac{7}{5} $, qui abbiamo tre linee tratteggiate quindi $(-)\cdot(-)\cdot(-)=-$, $-\dfrac{7}{5}<x<-1$ qui abbiamo due linee tratteggiate e una continua $(-)\cdot(-)\cdot(+)=+$, $x>-\dfrac{1}{2} $ qui abbiamo una linea tratteggiata  e due linee continue $(-)\cdot(+)\cdot(+)=-$, $x>-\dfrac{1}{2}$ qui abbiamo tre linee continue $(+)\cdot(+)\cdot(+)=+$. La disequazione\nobs\vref{equ:DisFrazPrimoG2a} chiede quando la frazione sia maggiore o uguale a zero quindi, per quanto detto prima le soluzioni sono $-\dfrac{7}{5}\leq x<-1$ e $x>-\dfrac{1}{2}$.


\subsection{Riepilogo}
\begin{procedurat}{}{}
\begin{enumerate}
	\item Verifico se la disequazione è in forma normale. Altrimenti semplifico l'espressione.
	\item Separo il numeratore e il denominatore della frazione e li pongo maggiori di zero.
	\item Risolvo separatamente le due disequazioni.
	\item Sovrappongo i grafici delle due disequazioni.
	\item Applico la regola dei segni.
	\item Risolvo la disequazione confrontando i risultati del grafico da quanto richiesto dalla disequazione.
\end{enumerate}
\end{procedurat}
 


    \chapter{Disequazioni di secondo grado}
\label{cha:Disequazionisecondogrado}
%\section{Disequazioni intere}
%Una disequazione di secondo grado\index{Disequazione!secondo grado!intera} è un'espressione del tipo
%\begin{equation}
%2x^2-x-1\geq 0\label{equ:DisSecondoGrado1}
%\end{equation} 
%in cui abbiamo un trinomio posto maggiore o uguale di zero. Per risolvere la disequazione  possiamo procedere in questo modo. Trasformiamo  il problema in un altro. Per far ciò  utilizziamo la relazione
%\begin{align}
%&ax^2+bx+c=a(x-x_{1})(x-x_{2})\label{DisTrinsecGrado0}\\
%&x_{1}=\dfrac{-b+\sqrt{b^2-4ac}}{2a}\label{DisTrinsecGrado1}\\
%&x_{2}=\dfrac{-b-\sqrt{b^2-4ac}}{2a}\label{DisTrinsecGrado2}
%\end{align}
%Quindi utilizzando le relazioni\nobs\vrefrange{DisTrinsecGrado0}{DisTrinsecGrado2} possiamo scrivere 
%\begin{align}
%&2x^2-x-1\\
%&x_{1}=\dfrac{1+\sqrt{9}}{4}\notag
%&x_{1}=\dfrac{1-\sqrt{9}}{4}\notag\\
%&x_{1}=\dfrac{1+3}{4}\notag
%&x_{1}=\dfrac{1-3}{4}\notag\\
%&x_{1}=\dfrac{4}{4}=1\notag
%&x_{1}=-\dfrac{2}{4}=-\dfrac{1}{2}\notag\\
%&2x^2-x-1=2(x-1)(x+\dfrac{1}{2})\label{DisTrinsecGradoEs1}
%\end{align}
%La relazione\nobs\vref{DisTrinsecGradoEs1} trasforma un trinomio di secondo grado nel prodotto di due binomi e un numero. Quindi la disequazione\nobs\vref{equ:DisSecondoGrado1} diventa 
%\begin{equation}
%2(x-1)(x+\dfrac{1}{2})\geq 0\label{equ:DisSecondogrado1a}
%\end{equation} 
%Ottengo il grafico\nobs\vref{fig:esempioDisSecGrad1}. Il disegno mostra tre righe una per ogni fattore del prodotto\nobs\vref{equ:DisSecondogrado1a}. Nel particolare il trinomio è positivo per $x\leq-\dfrac{1}{2}$ o $x\geq 1$ negativo per $-\dfrac{1}{2}\leq x\leq 1$
%
%Consideriamo un altro esempio
%\begin{equation}
%-3x^2+4x-1\geq 0\label{equ:DisSecondoGrado2}
%\end{equation} 
%anche qui abbiamo un trinomio posto maggiore o uguale di zero. Per risolvere la disequazione  procediamo come prima  Trasformando  il problema in un altro. 
%Quindi utilizzando le relazioni\nobs\vrefrange{DisTrinsecGrado0}{DisTrinsecGrado2} possiamo scrivere: 
%\begin{align}
%&-3x^2+4x-1\\
%&x_{1}=\dfrac{-4+\sqrt{4}}{-6}\notag
%&x_{1}=\dfrac{-4-\sqrt{4}}{-6}\notag\\
%&x_{1}=\dfrac{-4+2}{-6}\notag
%&x_{1}=\dfrac{-4-2}{-6}\notag\\
%&x_{1}=\dfrac{-2}{-6}=\dfrac{1}{3}\notag
%&x_{1}=-\dfrac{-6}{-6}=1\notag\\
%&-3x^2+4x-1=-3(x-1)(x-\dfrac{1}{3})\label{DisTrinsecGradoEs2}
%\end{align}
%Anche questa volta relazione\nobs\vref{DisTrinsecGradoEs2} trasforma il trinomio di secondo grado nel prodotto di due binomi e un numero. Quindi la disequazione\nobs\vref{equ:DisSecondoGrado2} diventa 
%\begin{equation}
%-3(x-1)(x-\dfrac{1}{3})\geq 0\label{equ:DisSecondogrado2a}
%\end{equation} 
%Ottengo il grafico\nobs\vref{fig:esempioDisSecGrad2}. Il disegno mostra tre righe una per ogni fattore del prodotto\nobs\vref{equ:DisSecondogrado2a}. Nel particolare il trinomio è negativo per $x\leq\dfrac{1}{3}$ o $x\geq 1$, positivo per $-\dfrac{1}{3}\leq x\leq 1$
%\begin{figure}
%	\centering
%		\begin{subfigure}[b]{.4\linewidth}
%			\centering
%			\includestandalone[width=\textwidth]{quarto/DisSecGrado/DisSecGradoesempio1}
%			\caption{Esempio 1}
%			\label{fig:esempioDisSecGrad1}
%		\end{subfigure}%
%	\centering
%	\quad
%	\begin{subfigure}[b]{.4\linewidth}
%			\centering
%			\includestandalone[width=\textwidth]{quarto/DisSecGrado/DisSecGradoesempio2}
%			\caption{Esempio 2}
%			\label{fig:esempioDisSecGrad2}
%	\end{subfigure}%
%\caption{$\Delta>0$}
%\label{fig:DeltaMagZeroEsempio1}
%\end{figure}
%
%La figura\nobs\vref{fig:DeltaMagZeroEsempio1} permette di confrontare i due grafici. Questi sono praticamente identici. \'{E} la terza riga, continua nel primo esempio, tratteggiata nel secondo, che fa la differenza. Guardando i due grafici, vediamo che all'esterno dell'intervallo formato dalle due soluzioni, il grafico ha lo stesso segno del coefficiente $a$. Mentre nello spazio compreso fra le due soluzioni, il segno è opposto a quello di $a$. Otteniamo due grafici come\nobs\vrefrange{graf:dis2GDeltaMagZa2x}{graf:dis2GDeltaMagZb2x}. Questo è riassunto graficamente dalla prima riga della tabella\nobs\vref{tab:segnodisequazioni2grado}
%\begin{figure}
%	\begin{subfigure}[b]{.5\linewidth}
%		\centering
%\includestandalone[width=\textwidth]{quarto/DisSecGrado/DeltaMaggioreDiZeroAmaggioreDizero}
%		\caption{$\Delta>0$ $a>0$}\label{graf:dis2GDeltaMagZa2x}
%	\end{subfigure}%
%\quad
%	\begin{subfigure}[b]{.5\linewidth}
%		\centering
%	\includestandalone[width=\textwidth]{quarto/DisSecGrado/DeltaMaggioreDiZeroAminoreDizero}
%		\caption{$\Delta>0$ $a<0$}\label{graf:dis2GDeltaMagZb2x}
%	\end{subfigure}
%\vskip .8cm
%	\begin{subfigure}[b]{.5\linewidth}
%		\centering
%		\includestandalone[width=\textwidth]{quarto/DisSecGrado/DeltaUgualeaZeroAmaggioreDizero}
%		\caption{$\Delta=0$ $a>0$}\label{graf:dis2GDeltaUguaZa2x}
%			\end{subfigure}%
%\quad
%	\begin{subfigure}[b]{.5\linewidth}
%		\centering
%		\includestandalone[width=\textwidth]{quarto/DisSecGrado/DeltaUgualeaZeroAminoreDizero}
%		\caption{$\Delta=0$ $a<0$}\label{graf:dis2GDeltaUguaZb2x}
%	\end{subfigure}
%\vskip .8cm
%\begin{subfigure}[b]{.5\linewidth}
%	\centering
%		\includestandalone[width=\textwidth]{quarto/DisSecGrado/DeltaMinoreZeroAmaggioreDizero}
%	\caption{$\Delta<0$ $a>0$}\label{graf:dis2GDeltaMinorZa2x}
%\end{subfigure}%
%\quad
%\begin{subfigure}[b]{.5\linewidth}
%	\centering
%\includestandalone[width=\textwidth]{quarto/DisSecGrado/DeltaMinoreZeroAminoreDizero}
%	\caption{$\Delta<0$ $a<0$}\label{graf:dis2GDeltaMinorZb2x}
%\end{subfigure}
%	\caption{Grafici disequazione di secondo grado}
%\end{figure}
%\begin{table}
%	\begin{tabular}{@{}m{1cm}m{7.8cm}m{7.8cm}}
%	%	\toprule
%		& \centering $a>0$ & \centering$a<0$\tabularnewline
%\centering$\Delta>0$ &\tabincludestandalone[width=7.5cm]{quarto/DisSecGrado/DeltaMaggioreDiZeroAmaggioreDizero}  & \vskip .8cm 	\tabincludestandalone[width=7.5cm]{quarto/DisSecGrado/DeltaMaggioreDiZeroAminoreDizero} \\[1cm] 
%		\centering$\Delta=0$ & 	\tabincludestandalone[width=7.5cm]{quarto/DisSecGrado/DeltaUgualeaZeroAmaggioreDizero} & \vskip .8cm \tabincludestandalone[width=7.5cm]{quarto/DisSecGrado/DeltaUgualeaZeroAminoreDizero}\\[1cm] 
%		\centering$\Delta<0$ & \tabincludestandalone[width=7.5cm]{quarto/DisSecGrado/DeltaMinoreZeroAmaggioreDizero} &\vskip .8cm \tabincludestandalone[width=7.5cm]{quarto/DisSecGrado/DeltaMinoreZeroAminoreDizero}\\[1cm] 
%	%	\bottomrule
%	\end{tabular}
%	\caption{Segno disequazioni secondo grado}
%\end{table}
%\begin{table}
%	\centering
%	\begin{tikzpicture}
%	\tkzTabInit[color,lgt=5,espcl=3]%
%	{$x$ / .8,$\Delta>0$\\ Il segno di\\ $ax^2+bx+c$ /2}%
%	{$-\infty$,$x_1$,$x_2$,$+\infty$}%
%	\tkzTabLine{ , \genfrac{}{}{0pt}{0}{\text{segno di}}{a}, z
%		, \genfrac{}{}{0pt}{0}{\text{segno}}{\text{opposto di}\ a}, z
%		, \genfrac{}{}{0pt}{0}{\text{segno di}}{a}, }
%	\end{tikzpicture}\\
%	\begin{tikzpicture}
%	\tkzTabInit[color,lgt=5,espcl=3]%
%	{$x$ / .8, $\Delta=0$\\ Il segno di\\ $ax^2+bx+c$ / 2}%
%	{$-\infty$,$x_1$,$+\infty$}%
%	\tkzTabLine{ , \genfrac{}{}{0pt}{0}{\text{segno di}}{ a} , z
%		, \genfrac{}{}{0pt}{0}{\text{segno di}}{a}, }
%	\end{tikzpicture}\\
%	\begin{tikzpicture}
%	\tkzTabInit[color,lgt=5,espcl=5]%
%	{$x$/.8,$\Delta<0$\\ Il segno di\\ $ax^2+bx+c$/2}%
%	{$-\infty$,$+\infty$}%
%	\tkzTabLine{ , \genfrac{}{}{0pt}{0}{\text{segno di}}{ a}, }
%	\end{tikzpicture}
%	\caption{Segno disequazione di secondo grado}
%	\label{tab:segnodisequazioni2grado}
%\end{table}
%
%Consideriamo una disequazione del tipo
%\begin{equation}
%2x^2-4x+2\geq 0\label{equ:DisSecondoGrado3}
%\end{equation} 
%come negli esempi precedenti abbiamo un trinomio di secondo grado  posto maggiore o uguale di zero. Anche qui trasformiamo  il problema in un altro. Per far ciò  utilizziamo le relazioni\nobs\vrefrange{DisTrinsecGrado0}{DisTrinsecGrado2}.
%Possiamo scrivere:
%\begin{align}
%&2x^2-4x+2\\
%&x_{1}=\dfrac{4+\sqrt{0}}{4}\notag
%&x_{1}=\dfrac{4-\sqrt{0}}{4}\notag\\
%&x_{1}=\dfrac{4}{4}=1\notag
%&x_{1}=\dfrac{4}{4}=1\notag\\
%&2x^2-4x+2=2(x-1)(x-1)=2(x-1)^2\label{DisTrinsecGradoEs3}
%\end{align}
%La relazione\nobs\vref{DisTrinsecGradoEs1} trasforma un trinomio di secondo grado nel prodotto di un binomio al quadrato e un numero. Quindi la disequazione\nobs\vref{equ:DisSecondoGrado3} diventa 
%\begin{equation}
%2(x-1)^2\geq 0\label{equ:DisSecondogrado3a}
%\end{equation} 
%Ottengo il grafico\nobs\vref{fig:esempioDisSecGrad3}. Il disegno mostra tre righe una per ogni fattore del prodotto\nobs\vref{equ:DisSecondogrado3a}. Nel particolare il trinomio è positivo per $x1$ e $x>1$, vale zero  per $x=1$
%\begin{figure}
%	\centering
%	\begin{subfigure}[b]{.4\linewidth}
%		\centering
%		\includestandalone[width=\textwidth]{quarto/DisSecGrado/DisSecGradoesempio3}
%		\caption{Esempio 3}
%		\label{fig:esempioDisSecGrad3}
%	\end{subfigure}%
%	\quad\centering
%	\begin{subfigure}[b]{.4\linewidth}
%		\centering
%		\includestandalone[width=\textwidth]{quarto/DisSecGrado/DisSecGradoesempio4}
%		\caption{Esempio 4}
%		\label{fig:esempioDisSecGrad4}
%	\end{subfigure}%
%	\caption{$\Delta=0$}
%	\label{fig:DeltaUguZeroEsempio2}
%\end{figure}
%
%Continuiamo con gli esempi
%\begin{equation}
%-3x^2+12x-12\geq 0\label{equ:DisSecondoGrado4}
%\end{equation} 
%come in precedenza abbiamo un trinomio di secondo grado  posto maggiore o uguale di zero.  Utilizziamo anche qui le relazioni\nobs\vrefrange{DisTrinsecGrado0}{DisTrinsecGrado2}.
%Possiamo scrivere:
%\begin{align}
%&-3x^2+12x-12\\
%&x_{1}=\dfrac{-12+\sqrt{0}}{-6}\notag
%&x_{1}=\dfrac{-12-\sqrt{0}}{-6}\notag\\
%&x_{1}=\dfrac{-12}{-6}=2\notag
%&x_{1}=\dfrac{-12}{-6}=2\notag\\
%&-3x^2+12x-12=-3(x-2)(x-2)=-3(x-2)^2\label{DisTrinsecGradoEs4}
%\end{align}
%La relazione\nobs\vref{DisTrinsecGradoEs1} trasforma un trinomio di secondo grado nel prodotto di un binomio al quadrato e un numero. Quindi la disequazione\nobs\vref{equ:DisSecondoGrado4} diventa 
%\begin{equation}
%-3(x-2)^2\geq 0\label{equ:DisSecondogrado4a}
%\end{equation} 
%Ottengo il grafico\nobs\vref{fig:esempioDisSecGrad4}. Il disegno mostra tre righe una per ogni fattore del prodotto\nobs\vref{equ:DisSecondogrado4a}. Nel particolare il trinomio è negativo per $x<2$ e $x>2$, vale zero  per $x=2$
%
%Confrontiamo i due grafici tramite la figura\nobs\vref{fig:DeltaUguZeroEsempio2}. Anche  la terza  riga, continua nel primo esempio, tratteggiata nel secondo, fa la differenza. Guardando i due grafici, vediamo che il grafico ha lo stesso segno del coefficiente $a$ tranne per $x=x_1$ in cui vale zero.  come\nobs\vrefrange{graf:dis2GDeltaUguaZa2x}{graf:dis2GDeltaUguaZb2x}. Questo è riassunto graficamente dalla seconda riga della tabella\nobs\vref{tab:segnodisequazioni2grado}
%
%Nella prima coppia di esempi avevamo due soluzioni distinte, nella seconda le soluzioni erano coincidenti. Resta da considerare quando le soluzioni non esistono.
%
%In questo caso il discriminate dell'equazione è un numero minore di zero. Si può dimostrare con qualche calcolo in più che i grafici sono come quelli delle figure\nobs\vrefrange{graf:dis2GDeltaMinorZa2x}{graf:dis2GDeltaMinorZb2x}. In questo caso il grafico ha sempre lo stesso segno di $a$  nella terza riga della tabella\nobs\vref{tab:segnodisequazioni2grado}
%\begin{table}
%	\centering
%	 \begin{tabular}{@{}cc>{\centering}m{6.5cm}>{\centering}m{6.5cm}}
%	 	&  & $a>0$ & \vskip .2cm $a<0$ \tabularnewline[0.5cm] 
%	 	&  & 	\tabincludestandalone[width=6.5cm]{quarto/DisSecGrado/DeltaMaggioreDiZeroAmaggioreDizero}  & \vskip .2cm	\tabincludestandalone[width=6.5cm]{quarto/DisSecGrado/DeltaMaggioreDiZeroAminoreDizero} \tabularnewline[0.5cm] 
%	 	\multirow{4}{1cm}{$\Delta>0$}	& $ax^2+bx+c\geq 0$ & $x\leq x_1$ e $x\geq x_2$  & $x_1\leq x \leq x_2$ \tabularnewline  
%	 	& $ax^2+bx+c > 0$ &$x< x_1$ e $x>x_2$  & $x_1< x < x_2$ \tabularnewline
%	 	& $ax^2+bx+c\leq 0$ & $x_1\leq x \leq x_2$ & $x\leq x_1$ e $x\geq x_2$ \tabularnewline  
%	 	& $ax^2+bx+c< 0$ & $x_1< x < x_2$ & $x< x_1$ e $x>x_2$ \tabularnewline
%	 	&&&\tabularnewline
%	 	&  & 	\tabincludestandalone[width=6.5cm]{quarto/DisSecGrado/DeltaUgualeaZeroAmaggioreDizero} &\vskip .2cm  \tabincludestandalone[width=6.5cm]{quarto/DisSecGrado/DeltaUgualeaZeroAminoreDizero}\tabularnewline[0.5cm] 
%	 	\multirow{4}{1cm}{$\Delta=0$}	& $ax^2+bx+c\geq 0$ & Sempre & $x=x_1$ \tabularnewline  
%	 	& $ax^2+bx+c > 0$ & Sempre $x\neq x_1$ & Mai \tabularnewline
%	 	& $ax^2+bx+c\leq 0$ & $x=x_1 $  & Sempre \tabularnewline  
%	 	& $ax^2+bx+c< 0$ & Mai & Sempre $x\neq x_1$ \tabularnewline  
%	 		&&&\tabularnewline
%	 	&  & 	\tabincludestandalone[width=6.5cm]{quarto/DisSecGrado/DeltaMinoreZeroAmaggioreDizero} &\vskip .2cm \tabincludestandalone[width=6.5cm]{quarto/DisSecGrado/DeltaMinoreZeroAminoreDizero}\tabularnewline[0.5cm] 
%	 	\multirow{4}{1cm}{$\Delta<0$}	& $ax^2+bx+c\geq 0$ & Sempre. Uguale a zero mai & Mai \tabularnewline  
%	 	& $ax^2+bx+c > 0$ & Sempre & Mai \tabularnewline
%	 	& $ax^2+bx+c\leq 0$ & Mai & Sempre. Uguale a zero mai \tabularnewline  
%	 	& $ax^2+bx+c< 0$ & Mai & Sempre \tabularnewline  
%	 \end{tabular} 
%	\caption{Soluzioni disequazioni secondo grado}
%	\label{tab:SoluzioniDisequazioniSecondoGrado}
%\end{table}
\section{Disequazioni di secondo grado intere}
\begin{definizionet}{Disequazione intera in forma normale}{}
Una disequazione di secondo grado\index{Disequazione!secondo grado} intera è in forma normale se \[ax^2+bx+c\;\begin{cases}
>\\
<\\
\leq\\
\geq
\end{cases} 0\; \text{con}\; a\neq 0\]
\end{definizionet}
\begin{osservazionet}{}{}
Le seguenti disequazioni sono tutte in forma normale\index{Disequazione!forma normale}
\begin{align*}
&2x^2+3x+2>0\\
&3x^2+2\leq0\\
&-x^2+4x\geq0\\
&5x^2\geq0
\end{align*}
\end{osservazionet} 
\begin{esempiot}{Delta maggiore di zero $a$ maggiore di zero }{DeltaMaggiorediZeroamaggiore}
	Consideriamo la disequazione in forma normale
	\begin{align*}
	&x^2+x-12\geq0
	\intertext{ad essa è associata l'equazione}
	&x^2+x-12=0\\
	&x_{1,2}=\dfrac{-1\pm\sqrt{1+48}}{2}=\dfrac{-1\pm7}{2}=
	\begin{cases}
	x_1=+3\\x_2=-4
	\end{cases}
	\end{align*} 
Possiamo associare alla disequazione una parabola del tipo \[y=x^2+x-12\]
la parabola, visti i punti di intersezione calcolati prima e il coefficiente $a$ positivo, ha come grafico la figura~\vref{fig:DeltaMaggioreZeroEsempio1}. La disequazione può essere vista come \[y=x^2+x-12\geq 0 \] Guardando il grafico è evidente che $y$ è positiva per valori  di $x<-4$ e per valori di $x>3$. Inoltre è negativa per $-4<x<3$ mentre per $x=-4$ e per $x=3$ $y$ vale zero. Possiamo riportare quanto detto in precedenza nel grafico della figura~\vref{fig:DeltaMaggioreZeroGraficoEsempio1}. Possiamo risolvere la disequazione, e visto che richiede quando è maggiore o uguale a zero, la soluzione è per $x\leq -4$ o per $x\geq3$ 
\end{esempiot}
\begin{figure}
	\centering
	\includestandalone[width=8.5cm]{quarto/DisSecGrado/parabolaDeltapiuApiu}
	\caption{$\Delta>0$ $a>0$ }
	\label{fig:DeltaMaggioreZeroEsempio1}
\end{figure}
\begin{figure}
	\centering
	\includestandalone[width=8.5cm]{quarto/DisSecGrado/parabolaDeltapiuApiuGrafico}
	\caption{Segno $\Delta>0$ $a>0$}
	\label{fig:DeltaMaggioreZeroGraficoEsempio1}
\end{figure}
\begin{esempiot}{Delta maggiore di zero $a$ minore di zero}{}
	Consideriamo la disequazione in forma normale
	\begin{align*}
	&-2x^2+5x+7<0
	\intertext{ad essa è associata l'equazione}
	&-2x^2+5x+7=0\\
	&x_{1,2}=\dfrac{-5\pm\sqrt{25+56}}{-4}=\dfrac{-5\pm9}{-4}=
	\begin{cases}
	x_1=+\dfrac{7}{2}\\
	\\x_2=-1
	\end{cases}
	\end{align*} 
Come per l'esempio precedente disegniamo il grafico della parabola \[y=-2x^2+5x+7\] Questa parabola, per quanto calcolato prima e per $a<0$, ha il grafico è come quello della figura~\vref{fig:DeltaMaggioreZeroEsempio2} 

Possiamo dire che la parabola è negativa per valori di $x$ minori di meno uno e maggiori di $\dfrac{7}{2}$ mentre è positiva per valori di $x$ compresi tra meno uno e  $\dfrac{7}{2}$. Riportando graficamente quanto detto otteniamo il grafico~\vref{fig:DeltaMaggioreZeroGraficoEsempio2}. La soluzione è  $x\leq -4$ o per $x\geq3$ 
\end{esempiot}
\begin{figure}
	\centering 
	\includestandalone[width=7.5cm]{quarto/DisSecGrado/parabolaDeltapiuAmeno}
	\caption{$\Delta>0$ $a<0$}
	\label{fig:DeltaMaggioreZeroEsempio2}
\end{figure}
\begin{figure}
	\centering
	\includestandalone[width=8.5cm]{quarto/DisSecGrado/parabolaDeltapiuAmenoGrafico}
	\caption{Segno $\Delta>0$ $a<0$}
	\label{fig:DeltaMaggioreZeroGraficoEsempio2}
\end{figure}
\begin{esempiot}{Delta uguale a zero $a$ maggiore di zero}{DeltaUgualeaZeroaMaggiore}
	Consideriamo la disequazione in forma normale
\begin{align*}
&x^2-2x+1<0
\intertext{ad essa è associata l'equazione}
&x^2-2x+1=0\\
&x_{1,2}=\dfrac{2\pm\sqrt{4-4}}{2}=\dfrac{2}{2}=1
\end{align*} 
Possiamo associare alla disequazione una parabola del tipo \[y=x^2-2x+1\]
la parabola, visti il punto di intersezione calcolato prima e il coefficiente $a$ positivo, ha come grafico la figura~\vref{fig:DeltaUgualeaZeroEsempio3}

Dal grafico è evidente che la parabola è positiva per qualunque valore di $x$, tranne che per $x=1$ in cui $y=0$. Si ottiene un grafico come quello della figura~\vref{fig:DeltaUgualeaZeroGraficoEsempio3}. La disequazione richiedendo quali sono i valori di $x$ per cui il trinomio è negativo, non ha soluzione. 
\end{esempiot}
\begin{figure}
	\centering
	\includestandalone[width=8.5cm]{quarto/DisSecGrado/parabolaDeltazeroApiuGrafico}
	\caption{Segno $\Delta=0$ $a>0$}
	\label{fig:DeltaUgualeaZeroGraficoEsempio3}
\end{figure}
\begin{figure}
	\centering 
	\includestandalone[width=7.5cm]{quarto/DisSecGrado/parabolaDeltazeroApiu}
	\caption{$\Delta=0$ $a>0$}
	\label{fig:DeltaUgualeaZeroEsempio3}
\end{figure}
\begin{esempiot}{Delta uguale a zero $a$ minore di zero}{}
	Considero la disequazione
	\begin{align*}
	&-x^2+4x-4<0
	\intertext{ad essa è associata l'equazione}
	&-x^2+4x-4=0\\
	&x_{1,2}=\dfrac{-4\pm\sqrt{16-16}}{-2}=\dfrac{-4}{-2}=2
	\end{align*} 
Possiamo associare alla disequazione una parabola del tipo \[y=-x^2+4x-4\]
la parabola, visti il punto di intersezione calcolato prima e il coefficiente $a$ negativo, ha come grafico la figura~\ref{fig:DeltaUgualeaZeroEsempio4}. Il trinomio ha per segno come indicato nel grafico~\vref{fig:DeltaUgualeaZeroGraficoEsempio4}. La soluzione sarà sempre verificata con $x\neq 2$
\end{esempiot}
\begin{figure}
	\centering
	\includestandalone[width=8.5cm]{quarto/DisSecGrado/parabolaDeltazeroAmenoGrafico}
	\caption{Segno $\Delta=0$ $a<0$}
	\label{fig:DeltaUgualeaZeroGraficoEsempio4}
\end{figure}
\begin{figure}
	\centering 
	\includestandalone[width=7.5cm]{quarto/DisSecGrado/parabolaDeltazeroAmeno}
	\caption{$\Delta=0$ $a<0$}
	\label{fig:DeltaUgualeaZeroEsempio4}
\end{figure}
\begin{esempiot}{Delta minore di zero a maggiore di zero}{}
	\begin{align*}
	&x^2+x+1<0
	\intertext{ad essa è associata l'equazione}
	&x^2+x+1=0\\
	&x_{1,2}=\dfrac{-1\pm\sqrt{1-4}}{2}
	\intertext{non ha soluzione}
	\end{align*} 
	Per questo e dato che $a>0$ il grafico della parabola è come quello della figura~\vref{fig:DeltaminoreZeroEsempio5}. Il grafico della disequazione è il grafico~\vref{fig:DeltaMinoreZeroGraficoEsempio5}. Quindi visto che è sempre positivo la disequazione non è mai verificata.
\end{esempiot}
\begin{figure}
	\centering
	\includestandalone[width=8.5cm]{quarto/DisSecGrado/parabolaDeltamenoApiuGrafico}
	\caption{Segno $\Delta<0$ $a>0$}
	\label{fig:DeltaMinoreZeroGraficoEsempio5}
\end{figure}
\begin{figure}
	\centering 
	\includestandalone[width=7.5cm]{quarto/DisSecGrado/parabolaDeltamenoApiu}
	\caption{$\Delta<0$ $a>0$}
	\label{fig:DeltaminoreZeroEsempio5}
\end{figure}
\begin{esempiot}{Delta minore di zero a minore di zero}{}
	\begin{align*}
&-2x^2-1<0
\intertext{ad essa è associata l'equazione}
&-2x^2-1=0\\
&x_{1,2}=\dfrac{0\pm\sqrt{0-8}}{2}
\intertext{non ha soluzione}
\end{align*} Per questo e dato che $a<0$ il grafico della parabola è come quello della figura~\vref{fig:DeltaminoreZeroEsempio6}. Il grafico corrispondente è quello della figura~\vref{fig:DeltaMinoreZeroGraficoEsempio6}
\end{esempiot}
\begin{figure}
	\centering
	\includestandalone[width=8.5cm]{quarto/DisSecGrado/parabolaDeltamenoAmenoGrafico}
	\caption{Segno $\Delta<0$ $a<0$}
	\label{fig:DeltaMinoreZeroGraficoEsempio6}
\end{figure}
\begin{figure}
	\centering 
	\includestandalone[width=7.5cm]{quarto/DisSecGrado/parabolaDeltamenoAmeno}
	\caption{$\Delta<0$ $a<0$}
	\label{fig:DeltaminoreZeroEsempio6}
\end{figure}

Ricapitolando gli esercizi precedenti, ad ogni equazione sono associati un numero detto delta\index{Equazione!delta} \[\Delta=b^2-4ac\] ed un'equazione\index{Equazione!secondo grado} di secondo grado\[ax^2+bx+c=0\]  Per risolvere una disequazione di secondo grado intera procedo come segue
\begin{enumerate}
	\item Metto la disequazione in forma normale
	\item Memorizzo il segno di $a$
	\item Risolvo l'equazione corrispondente $ax^"+bx+c=0$ tramite $x_{1,2}=\dfrac{-b\pm\sqrt{\Delta}}{2a}$. Avremo tre casi
	\begin{description}
		\item[3a] l'equazione ha due soluzioni distinte $x_1\neq x_2$, $\Delta>0$
		\item[3b] l'equazione ha due soluzioni coincidenti $x_1= x_2$, $\Delta=0$
		\item[3c] l'equazione non ha soluzioni $\Delta<0$
	\end{description}
\item Disegno il grafico. Dal numero delle soluzioni ho tre casi
\begin{description}
	\item[Caso 3a] Soluzioni distinte\begin{enumerate}
		\item riporto le soluzioni sull'asse $x$, ordinandole dalla minore  alla maggiore.
		\item traccio due segmenti verticali di stessa lunghezza per ogni soluzione
		\item fuori dello spazio tra le due soluzioni, il grafico ha lo stesso segno di $a$. Quindi se $a$ è positiva fuori traccio due linee continue. Se $a$ è negativa fuori  disegno due linee tratteggiate. Tra le due soluzioni traccio l'opposto di quello che c'è fuori. Si dice che i grafici sono DICE cioè \textbf{C}oncordi \textbf{I}nterni \textbf{C}oncordi \textbf{E}sterni
	\end{enumerate}
\item[Caso 3b] Soluzioni coincidenti
\begin{enumerate}
	\item riporto la soluzione sull'asse $x$.
	\item traccio un segmento verticale per la soluzione
	\item  se $a$ è positiva traccio una linea continua. Se $a$ è negativa  traccio una linea tratteggiata. Il grafico è concorde tranne in un punto.
\end{enumerate} 
\item[Caso 3c] Nessuna soluzione
\begin{enumerate}
	\item se $a$ è positiva traccio una linea continua, se $a$ è negativa traccio una linea tratteggiata. Il grafico è concorde
\end{enumerate}
\end{description}
\end{enumerate}
\section{Disequazioni intere e soluzioni}
\begin{esempiot}{$\Delta>0$ $a>0$ e soluzioni}{}
	Troviamo il segno di $2x^2+5x+3$
	\begin{align*}
&2x^2+5x+3=0
\intertext{che risolta}
&x_{1,2}=\dfrac{-5\pm\sqrt{25-24}}{4}=\dfrac{-5\pm 1}{4}=\begin{cases}
x_1=-\dfrac{7}{2}\\
\\x_2=-1
\end{cases}
	\end{align*}
Dato che il delta è positivo e $a$ è positivo il grafico è di tipo DICE quindi è la  figura~\vref{fig:DeltaMaggioreZeroGraficoEsempio7}.	 Al  trinomio dell'esempio possiamo associare quattro disequazioni con quattro soluzioni diverse. 
\begin{align*}
&2x^2+5x+3\geq0&&x\leq-\dfrac{3}{2}\quad x\geq-1\\
&2x^2+5x+3>0&&x<-\dfrac{3}{2}\quad x>-1\\
&2x^2+5x+3\leq 0&&-\dfrac{3}{2}\leq x\leq-1\\
&2x^2+5x+3<0&&-\dfrac{3}{2}< x<-1
\end{align*}
\end{esempiot}
\begin{figure}
	\centering
	\includestandalone[width=8.5cm]{quarto/DisSecGrado/parabolaDeltapiuApiuGrafico2}
	\caption{Segno $\Delta>0$ $a>0$}
		\label{fig:DeltaMaggioreZeroGraficoEsempio7}
\end{figure}
\begin{esempiot}{$\Delta>0$ $a<0$ e soluzioni}{}
	Troviamo il segno di $-3x^2+4x+4$
\begin{align*}
&-3x^2+4x+4=0
\intertext{che risolta}
&x_{1,2}=\dfrac{-4\pm\sqrt{16+48}}{-6}=\dfrac{-4\pm 8}{-6}=\begin{cases}
x_1=-\dfrac{2}{3}\\
\\x_2=2
\end{cases}
\end{align*}
Dato che il delta è positivo e $a$ è negativo il grafico è di tipo DICE quindi è la  figura~\vref{fig:DeltaMaggioreZeroGraficoEsempio8}.	 Al  trinomio dell'esempio possiamo associare quattro disequazioni con quattro soluzioni diverse. 
\begin{align*}
&-3x^2+4x+4\geq0&&-\dfrac{2}{3}\leq x\leq 2\\
&-3x^2+4x+4>0&&-\dfrac{2}{3}< x<2\\
&-3x^2+4x+4\leq 0&&x\leq-\dfrac{2}{3}\quad x\geq 2\\
&-3x^2+4x+4<0&&x<-\dfrac{3}{2}\quad x>2
\end{align*}
\end{esempiot}
\begin{figure}
	\centering
	\includestandalone[width=8.5cm]{quarto/DisSecGrado/parabolaDeltapiuAmenoGrafico2}
	\caption{Segno $\Delta>0$ $a<0$}
	\label{fig:DeltaMaggioreZeroGraficoEsempio8}
\end{figure}
\begin{esempiot}{$\Delta=0$ $a>0$}{}
		Troviamo il segno di $x^2+6x+9$
	\begin{align*}
	&x^2+6x+9=0
	\intertext{che risolta}
	&x_{1,2}=\dfrac{-6\pm\sqrt{36-36}}{2}=-3
	\end{align*}
Delta è uguale a zero il coefficiente $a$ è positivo quindi il grafico è concorde con $a$ ed è come quello della figura~\vref{fig:DeltaUgualeaZeroGraficoEsempio9}
 Al  trinomio dell'esempio possiamo associare quattro disequazioni con quattro soluzioni diverse. 
\begin{align*}
&x^2+6x+9\geq0&&\text{Sempre verificata}\\
&x^2+6x+9>0&&x\neq -3\\
&x^2+6x+9\leq 0&&x=-3\\
&x^2+6x+9<0&&\text{Mai verificata}
\end{align*}
\end{esempiot}
\begin{figure}
	\centering
	\includestandalone[width=8.5cm]{quarto/DisSecGrado/parabolaDeltazeroApiuGrafico2}
	\caption{Segno $\Delta=0$ $a>0$}
	\label{fig:DeltaUgualeaZeroGraficoEsempio9}
\end{figure}
\begin{esempiot}{$\Delta=0$ $a<0$}
	Troviamo il segno di $-4x^2+12x-9$
	\begin{align*}
	&-4x^2+12x-9=0
	\intertext{che risolta}
	&x_{1,2}=\dfrac{-12\pm\sqrt{144-144}}{-8}=\dfrac{3}{4}
	\end{align*}
	Delta è uguale a zero il coefficiente $a$ è negativo quindi il grafico, concorde, è come quello della figura~\vref{fig:DeltaUgualeaZeroGraficoEsempio10}
	Al  trinomio dell'esempio possiamo associare quattro disequazioni con quattro soluzioni diverse. 
	\begin{align*}
	&-4x^2+12x-9\geq0&&x=\dfrac{3}{4}\\
	&-4x^2+12x-9>0&&\text{Mai verificata}\\
	&-4x^2+12x-9\leq 0&&\text{Sempre verificata}\\
	&-4x^2+12x-9<0&&x\neq \dfrac{3}{4}
	\end{align*}
\end{esempiot}
\begin{figure}
	\centering
	\includestandalone[width=8.5cm]{quarto/DisSecGrado/parabolaDeltazeroAmenoGrafico2}
	\caption{Segno $\Delta=0$ $a<0$}
	\label{fig:DeltaUgualeaZeroGraficoEsempio10}
\end{figure}
\begin{esempiot}{$\Delta<0$ $a>0$}
	Troviamo il segno di $x^2+2x+5$
	\begin{align*}
	&x^2+2x+5=0
	\intertext{che risolta}
	&x_{1,2}=\dfrac{-2\pm\sqrt{4-20}}{2}
	\intertext{non ha soluzione}
	\end{align*}
	Delta è minore di zero, il coefficiente $a$ è maggiore di zero quindi il grafico, concorde, è come quello della figura~\vref{fig:DeltaMinoreZeroGraficoEsempio11}
	Al  trinomio dell'esempio possiamo associare quattro disequazioni con quattro soluzioni diverse. 
	\begin{align*}
	&x^2+2x+5\geq0&&\text{Sempre maggiore di zero, mai uguale a zero}\\
	&x^2+2x+5>0&&\text{Sempre maggiore di zero}\\
	&x^2+2x+5\leq 0&&\text{Mai verificata}\\
	&x^2+2x+5<0&&\text{Mai verificata}
	\end{align*}
\end{esempiot}
\begin{figure}
	\centering
	\includestandalone[width=8.5cm]{quarto/DisSecGrado/parabolaDeltamenoApiuGrafico2}
	\caption{Segno $\Delta<0$ $a>0$}
	\label{fig:DeltaMinoreZeroGraficoEsempio11}
\end{figure}
\begin{esempiot}{$\Delta<0$ $a<0$}
	Troviamo il segno di $-2x^2+3x-3$
	\begin{align*}
	&-2x^2+3x-3=0
	\intertext{che risolta}
	&x_{1,2}=\dfrac{-3\pm\sqrt{9-24}}{-4}
	\intertext{non ha soluzione}
	\end{align*}
	Delta è minore di zero, il coefficiente $a$ è minore di zero quindi il grafico, concorde, è come quello della figura~\vref{fig:DeltaMinoreZeroGraficoEsempio12}
	Al  trinomio dell'esempio possiamo associare quattro disequazioni con quattro soluzioni diverse. 
	\begin{align*}
	&-2x^2+3x-3\geq0&&\text{Mai verificata}\\
	&-2x^2+3x-3>0&&\text{Mai verificata}\\
	&-2x^2+3x-3\leq 0&&\text{Sempre minore di zero, uguale a zero mai}\\
	&-2x^2+3x-3<0&&\text{Sempre verificata}
	\end{align*}
\end{esempiot}
\begin{figure}
	\centering
	\includestandalone[width=8.5cm]{quarto/DisSecGrado/parabolaDeltamenoAmenoGrafico2}
	\caption{Segno $\Delta<0$ $a<0$}
	\label{fig:DeltaMinoreZeroGraficoEsempio12}
\end{figure}
%\altapriorita{aggiungere disequazioni frazionarie di secondo grado}
% % % % % % % % % % % % % % % % % % % % % % % % % % % % %
%\section{Metodo grafico}
%\label{sec:MetodoGrafico}
%\begin{figure}
%	
%		\begin{subfigure}[b]{.5\linewidth}
%		\centering
%		\begin{tikzpicture}[line cap=round,line join=round,>=triangle 45,x=1.0cm,y=1.0cm]
%		\draw[->,color=black] (-3,0) -- (3,0);
%		%\foreach \x in {-2.5,-2,-1.5,-1,-0.5,0.5,1,1.5,2,2.5,3}
%		%\draw[shift={(\x,0)},color=black] (0pt,-2pt);
%		\clip(-3,-2.28) rectangle (3,1);
%		\draw [samples=50,rotate around={0:(0,-2.13)},xshift=0cm,yshift=-2.13cm] %plot (\x,\x^2/2/0.2599999999999998);
%		plot(\x,{(\x)^2-0.1}); 
%		%\draw [samples=50,rotate around={0:(0,-2.13)},xshift=0cm,yshift=-2.13cm] plot (\x,(\x)^2/2/0.2599999999999998);
%		\draw (-3,0.37) node[anchor=north west] {$+++++$};
%		\draw (-1.17,0.05) node[anchor=north west] {$x_1$};
%		\draw (1.11,0.05) node[anchor=north west] {$x_2$};
%		\draw (1.17,0.37) node[anchor=north west] {$++++++$};
%		\draw (-0.7,-0.06) node[anchor=north west] {$-----$};
%		\end{tikzpicture}
%		\caption{$\Delta>0$ $a>0$}\label{graf:dis2GDeltaMagZGa1}
%	\end{subfigure}%
%	\begin{subfigure}[b]{.5\linewidth}
%		\centering
%			\begin{tikzpicture}[line cap=round,line join=round,>=triangle 45,x=1.0cm,y=1.0cm]
%			\draw[->,color=black] (-3,0) -- (3,0);
%			%\foreach \x in {-3,-2.5,-2,-1.5,-1,-0.5,0.5,1,1.5,2,2.5}
%			%\draw[shift={(\x,0)},color=black] (0pt,-2pt);
%			\clip(-3,-1) rectangle (3,2.26);
%			\draw [samples=50,rotate around={-180:(0,2.13)},xshift=0cm,yshift=2.13cm] 
%			%plot (\x,\x^2/2/0.2599999999999998);
%			plot(\x,{(-\x)^2-0.1}); 
%			\draw (-0.8,0.37) node[anchor=north west] {$+++++$};
%			\draw (-1.17,0.04) node[anchor=north west] {$x_1$};
%			\draw (1.11,0.06) node[anchor=north west] {$x_2$};
%			\draw (-3,-0.37) node[anchor=north west] {$-----$};
%			\draw (1.39,-0.37) node[anchor=north west] {$-----$};
%			\end{tikzpicture}
%		\caption{$\Delta>0$ $a<0$}\label{graf:dis2GDeltaMagZGb1}
%	\end{subfigure}
%		\begin{subfigure}[b]{.5\linewidth}
%			\centering
%			\begin{tikzpicture}[line cap=round,line join=round,>=triangle 45,x=1.0cm,y=1.0cm]
%			\draw[->,color=black] (-3,0) -- (3,0);
%			%\foreach \x in {-3,-2.5,-2,-1.5,-1,-0.5,0.5,1,1.5,2,2.5}
%			%\draw[shift={(\x,0)},color=black] (0pt,-2pt);
%			\clip(-3,-0.3) rectangle (3,0.5);
%			\draw (-1,0.5)-- (-1,0);
%			\draw (1,0.5)-- (1,0);
%			\draw [line width=1.2pt,dash pattern=on 5pt off 5pt] (-1,0.5) -- (1,0.5);
%			\draw (1,0.5)-- (3,0.5);
%			\draw (-1,0.5)-- (-3,0.5);
%			\draw (-1.02,0.02) node[anchor=north west] {$x_1$};
%			\draw (0.98,0.02) node[anchor=north west] {$x_2$};
%			\end{tikzpicture}
%			\caption{$\Delta>0$ $a>0$}\label{graf:dis2GDeltaMagZGa2}
%		\end{subfigure}%
%		\begin{subfigure}[b]{.5\linewidth}
%			\centering
%				\begin{tikzpicture}[line cap=round,line join=round,>=triangle 45,x=1.0cm,y=1.0cm]
%				\draw[->,color=black] (-3,0) -- (3,0);
%				%\foreach \x in {-3,-2.5,-2,-1.5,-1,-0.5,0.5,1,1.5,2,2.5}
%				%\draw[shift={(\x,0)},color=black] (0pt,-2pt);
%				\clip(-3,-0.3) rectangle (3,0.5);
%				\draw (-1,0.5)-- (-1,0);
%				\draw (1,0.5)-- (1,0);
%				\draw (-1,0.5)-- (1,0.5);
%				\draw [dash pattern=on 5pt off 5pt] (1,0.5)-- (3.0,0.5);
%				\draw [dash pattern=on 5pt off 5pt] (-1,0.5)-- (-3.0,0.5);
%				\draw (-1.02,0.02) node[anchor=north west] {$x_1$};
%				\draw (0.98,0.02) node[anchor=north west] {$x_2$};
%				\end{tikzpicture}
%			\caption{$\Delta>0$ $a<0$}\label{graf:dis2GDeltaMagZGb2}
%		\end{subfigure}
%	\caption{$\Delta>0$}%
%	\label{fig:deltamz}
%\end{figure}
%
%\begin{figure}
%	\begin{subfigure}[b]{.5\linewidth}
%		\centering
%			\begin{tikzpicture}[line cap=round,line join=round,>=triangle 45,x=1.0cm,y=1.0cm]
%			\draw[->,color=black] (-3,0) -- (2.98,0);
%			%\foreach \x in {-3,-2.5,-2,-1.5,-1,-0.5,0.5,1,1.5,2,2.5}
%			%\draw[shift={(\x,0)},color=black] (0pt,-2pt);
%			\clip(-3,-0.29) rectangle (2.98,2);
%			\draw [samples=50,rotate around={0:(0,0)},xshift=0cm,yshift=0cm]
%			%plot (\x,\x^2/2/1.0);
%			plot(\x,{(\x)^2}); 
%			\draw (0,0.02) node[anchor=north west] {$x_1$};
%			\draw (-3,0.37) node[anchor=north west] {$+++++++++++++++++++++$};
%			\end{tikzpicture}
%		\caption{$\Delta=0$ $a>0$}\label{graf:dis2GDeltaUguaZGa1}
%	\end{subfigure}%
%	\qquad
%	\begin{subfigure}[b]{.5\linewidth}
%		\centering
%		\begin{tikzpicture}[line cap=round,line join=round,>=triangle 45,x=1.0cm,y=1.0cm]
%		\draw[->,color=black] (-3,0) -- (2.98,0);
%		%\foreach \x in {-3,-2.5,-2,-1.5,-1,-0.5,0.5,1,1.5,2,2.5}
%		%\draw[shift={(\x,0)},color=black] (0pt,-2pt);
%		\clip(-3,-2) rectangle (2.98,0.29);
%		\draw [samples=50,rotate around={-180:(0,0)},xshift=0cm,yshift=0cm] 
%		%plot (\x,\x^2/2/1.0);
%		plot(\x,{(-\x)^2}); 
%		\draw (0,0.02) node[anchor=north west] {$x_1$};
%		\draw (-3,0.37) node[anchor=north west] {$---------------------$};
%		\end{tikzpicture}
%		\caption{$\Delta=0$ $a<0$}\label{graf:dis2GDeltaDeltaUguaZGb1}
%	\end{subfigure}
%		\begin{subfigure}[b]{.5\linewidth}
%			\centering
%		\begin{tikzpicture}[line cap=round,line join=round,>=triangle 45,x=1.0cm,y=1.0cm]
%		\draw[->,color=black] (-3,0) -- (3,0);
%		%\foreach \x in {-2.5,-2,-1.5,-1,-0.5,0.5,1,1.5,2,2.5,3}
%		%\draw[shift={(\x,0)},color=black] (0pt,-2pt);
%		\clip(-3,-0.3) rectangle (3,0.5);
%		\draw (0,0.5)-- (0,0);
%		\draw [domain=-3:3] plot(\x,{(--0.56-0*\x)/1.12});
%		\draw (-0.03,0.02) node[anchor=north west] {$x_1$};
%		\end{tikzpicture}
%			\caption{$\Delta=0$ $a>0$}\label{graf:dis2GDeltaUguaZGa2}
%		\end{subfigure}%
%		\qquad
%		\begin{subfigure}[b]{.5\linewidth}
%			\centering
%			\begin{tikzpicture}[line cap=round,line join=round,>=triangle 45,x=1.0cm,y=1.0cm]
%			\draw[->,color=black] (-3,0) -- (3,0);
%			\foreach \x in {-2.5,-2,-1.5,-1,-0.5,0.5,1,1.5,2,2.5,3}
%			\draw[shift={(\x,0)},color=black] (0pt,-2pt);
%			\clip(-3,-0.3) rectangle (3,0.5);
%			\draw (0,0.5)-- (0,0);
%			%\draw [domain=-3:3] plot(\x,{(--0.56-0*\x)/1.12});
%			%\draw [dash pattern=on 5pt off 5pt,domain=-3:3] plot(\x,{(--1.18-0*\x)/1.18});
%			\draw [dash pattern=on 5pt off 5pt,domain=-3:3] plot(\x,{(--0.56-0*\x)/1.12});
%			\draw (-0.03,0.02) node[anchor=north west] {$x_1$};
%			\end{tikzpicture}
%			\caption{$\Delta=0$ $a<0$}\label{graf:dis2GDeltaUguaZGb2}
%		\end{subfigure}
%		\caption{$\Delta=0$}%
%		\label{fig:deltaugz}
%\end{figure}
%
%
%\begin{figure}
%	\begin{subfigure}[b]{.5\linewidth}
%		\centering
%		\begin{tikzpicture}[line cap=round,line join=round,>=triangle 45,x=1.0cm,y=1.0cm]
%		\draw[->,color=black] (-3,0) -- (3,0);
%		%\foreach \x in {-3,-2.5,-2,-1.5,-1,-0.5,0.5,1,1.5,2,2.5}
%		%\draw[shift={(\x,0)},color=black] (0pt,-2pt);
%		\clip(-3,-0.3) rectangle (3,2);
%		\draw [samples=50,rotate around={0:(0,0.15)},xshift=0cm,yshift=0.15cm] 
%		plot(\x,{(\x)^2+.2}); 
%		%plot (\x,\x^2/2/0.3);
%		\draw (-3,-0.01) node[anchor=north west] {$+++++++++++++++++++++$};
%		\end{tikzpicture}
%		\caption{$\Delta<0$ $a>0$}\label{graf:dis2GDeltaMinorZGa1}
%	\end{subfigure}%
%	\qquad
%	\begin{subfigure}[b]{.5\linewidth}
%		\centering
%		\begin{tikzpicture}[line cap=round,line join=round,>=triangle 45,x=1.0cm,y=1.0cm]
%		\draw[->,color=black] (-3,0) -- (3,0);
%		%\foreach \x in {-3,-2.5,-2,-1.5,-1,-0.5,0.5,1,1.5,2,2.5}
%		%\draw[shift={(\x,0)},color=black] (0pt,-2pt);
%		\clip(-3,-2) rectangle (3,0.3);
%		\draw [samples=50,rotate around={-180:(0,-0.15)},xshift=0cm,yshift=-0.15cm] 
%		%plot (\x,\x^2/2/0.3);
%		plot(\x,{(-\x)^2+.2}); 
%		\draw (-3,0.37) node[anchor=north west] {$---------------------$};
%		\end{tikzpicture}
%		\caption{$\Delta<0$ $a<0$}\label{graf:dis2GDeltaMinorZGb1}
%	\end{subfigure}
%	\begin{subfigure}[b]{.5\linewidth}
%			\centering
%			\begin{tikzpicture}[line cap=round,line join=round,>=triangle 45,x=1.0cm,y=1.0cm]
%			\draw[->,color=black] (-3,0) -- (3,0);
%			%\foreach \x in {-2.5,-2,-1.5,-1,-0.5,0.5,1,1.5,2,2.5,3}
%			%\draw[shift={(\x,0)},color=black] (0pt,-2pt);
%			\clip(-3,-0.3) rectangle (3,0.5);
%			\draw [domain=-3:3] plot(\x,{(--0.56-0*\x)/1.12});
%			\end{tikzpicture}
%			\caption{$\Delta<0$ $a>0$}\label{graf:dis2GDeltaMinorZGa2}
%	\end{subfigure}%
%	\begin{subfigure}[b]{.5\linewidth}
%			\centering
%			\begin{tikzpicture}[line cap=round,line join=round,>=triangle 45,x=1.0cm,y=1.0cm]
%			\draw[->,color=black] (-3,0) -- (3,0);
%			%\foreach \x in {-2.5,-2,-1.5,-1,-0.5,0.5,1,1.5,2,2.5,3}
%			%\draw[shift={(\x,0)},color=black] (0pt,-2pt);
%			\clip(-3,-0.3) rectangle (3,0.5);
%			\draw [dash pattern=on 5pt off 5pt,domain=-3:3] plot(\x,{(--0.56-0*\x)/1.12});
%			\end{tikzpicture}
%			\caption{$\Delta<0$ $a<0$}\label{graf:dis2GDeltaMinorZGb2}
%	\end{subfigure}
%	\caption{$\Delta<0$}%
%	\label{fig:deltaminz}
%\end{figure}

%\begin{table}[H]
%	%\ContinuedFloat
%	\centering%
%	\subfloat[][$\Delta<0$ $a>0$\label{graf:dis2GDeltaMinorZGa1}]{
%		\begin{tikzpicture}[line cap=round,line join=round,>=triangle 45,x=1.0cm,y=1.0cm]
%		\draw[->,color=black] (-3,0) -- (3,0);
%		%\foreach \x in {-3,-2.5,-2,-1.5,-1,-0.5,0.5,1,1.5,2,2.5}
%		%\draw[shift={(\x,0)},color=black] (0pt,-2pt);
%		\clip(-3,-0.3) rectangle (3,2);
%		\draw [samples=50,rotate around={0:(0,0.15)},xshift=0cm,yshift=0.15cm] 
%		plot(\x,{(\x)^2+.2}); 
%		%plot (\x,\x^2/2/0.3);
%		\draw (-3,-0.01) node[anchor=north west] {$+++++++++++++++++++++$};
%		\end{tikzpicture}
%	}
%	\subfloat[][$\Delta<0$ $a<0$\label{graf:dis2GDeltaMinorZGb1}]{
%		\begin{tikzpicture}[line cap=round,line join=round,>=triangle 45,x=1.0cm,y=1.0cm]
%		\draw[->,color=black] (-3,0) -- (3,0);
%		%\foreach \x in {-3,-2.5,-2,-1.5,-1,-0.5,0.5,1,1.5,2,2.5}
%		%\draw[shift={(\x,0)},color=black] (0pt,-2pt);
%		\clip(-3,-2) rectangle (3,0.3);
%		\draw [samples=50,rotate around={-180:(0,-0.15)},xshift=0cm,yshift=-0.15cm] 
%		%plot (\x,\x^2/2/0.3);
%		plot(\x,{(-\x)^2+.2}); 
%		\draw (-3,0.37) node[anchor=north west] {$---------------------$};
%		\end{tikzpicture}
%	}\quad%
%	\subfloat[][$\Delta<0$ $a>0$\label{graf:dis2GDeltaMinorZGa2}]{
%		\begin{tikzpicture}[line cap=round,line join=round,>=triangle 45,x=1.0cm,y=1.0cm]
%		\draw[->,color=black] (-3,0) -- (3,0);
%		%\foreach \x in {-2.5,-2,-1.5,-1,-0.5,0.5,1,1.5,2,2.5,3}
%		%\draw[shift={(\x,0)},color=black] (0pt,-2pt);
%		\clip(-3,-0.3) rectangle (3,0.5);
%		\draw [domain=-3:3] plot(\x,{(--0.56-0*\x)/1.12});
%		\end{tikzpicture}
%	}
%	\subfloat[][$\Delta<0$ $a<0$\label{graf:dis2GDeltaMinorZGb2}]{
%		\begin{tikzpicture}[line cap=round,line join=round,>=triangle 45,x=1.0cm,y=1.0cm]
%		\draw[->,color=black] (-3,0) -- (3,0);
%		%\foreach \x in {-2.5,-2,-1.5,-1,-0.5,0.5,1,1.5,2,2.5,3}
%		%\draw[shift={(\x,0)},color=black] (0pt,-2pt);
%		\clip(-3,-0.3) rectangle (3,0.5);
%		\draw [dash pattern=on 5pt off 5pt,domain=-3:3] plot(\x,{(--0.56-0*\x)/1.12});
%		\end{tikzpicture}}
%	\caption{$\Delta<0$}%
%	\label{fig:deltaminz}
%	%\label{graf:dis2Ggrafici}
%\end{table}


    \tcbstartrecording
\chapter{Esercizi disequazioni secondo grado}
\section{Disequazioni secondo grado fratte}
{\centering \begin{tabular}{ccc}
\toprule
	& $a>0$ &  $a<0$\\ 
	\midrule
$\Delta >0$	& A  & D \\ 
$\Delta=0$	& B & E \\ 
$\Delta<0$	& C &  F\\ 
\bottomrule
\end{tabular}\par }
\begin{exercise}
\tipo{AA}	Risolvere la seguente disequazione $\dfrac{2x^2+3x-2}{3x^2-x-2}\geq 0$
\tcblower
	Risolvere la seguente disequazione $\dfrac{2x^2+3x-2}{3x^2-x-2}\geq 0$
\begin{align*}
3x^2-x-2>&0\\
3x^2-x-2=&0\\
x_{1,2}=&\xunodue{3}{-1}{-2}=\dfrac{1\pm\sqrt{1+24}}{6}\\
=&\dfrac{1\pm\sqrt{25}}{6}\\
=&\dfrac{1\pm 5}{6}=\begin{cases}
x_1=\dfrac{6}{6}=1\\
x_2=-\dfrac{4}{6}=-\dfrac{2}{3}
\end{cases}\\
2x^2+3x-2\geq&0\\
2x^2+3x-2=&0\\
x_{1,2}=&\xunodue{2}{+3}{-2}=\dfrac{-3\pm\sqrt{9+16}}{4}\\
=&\dfrac{-3\pm\sqrt{25}}{4}\\
=&\dfrac{-3\pm 5}{4}=\begin{cases}
x_1=-\dfrac{8}{4}=-2\\
x_2=\dfrac{2}{4}=\dfrac{1}{2}
\end{cases}\\
\end{align*}
\begin{center}
	\includestandalone{quarto/grafici/disfrazAA1}
\end{center}
L'esercizio chiede quando la frazione è maggiore o uguale a zero quindi la riposta è 
\begin{align*}
x\leq& 2\\ -\dfrac{2}{3}<x&\leq \dfrac{1}{2}\\ 1<&x\\
\end{align*}
\end{exercise}
\begin{exercise}
\tipo{AD}	Risolvere la seguente disequazione $\dfrac{6x^2+x-2}{2-9x-5x^2}\leq 0$
\tcblower
Risolvere la seguente disequazione $\dfrac{6x^2+x-2}{2-9x-5x^2}\leq 0$
\begin{align*}
x<& -2\\ -\dfrac{2}{3}\leq x&< \dfrac{1}{5}\\ \dfrac{1}{2}\leq&x\\
\end{align*}
\end{exercise}
\begin{exercise}
	\tipo{AF}	Risolvere la seguente disequazione $\dfrac{6x^2-5x-4}{-1-x-x^2}< 0$
	\tcblower
	Risolvere la seguente disequazione $\dfrac{6x^2-5x-4}{-1-x-x^2}< 0$
\begin{align*}
-1-x-x^2>&0\\
-1-x-x^2=&0\\
x_{1,2}=&\xunodue{-1}{-1}{-4}=\dfrac{1\pm\sqrt{1-4}}{-2}\\
=&\dfrac{1\pm\sqrt{-3}}{-2}
\qquad\text{nessuna soluzione}\\
6x^2-5x-4>&0\\
6x^2-5x-4=&0\\
x_{1,2}=&\xunodue{6}{-5}{-4}=\dfrac{5\pm\sqrt{25+96}}{12}\\
=&\dfrac{5\pm\sqrt{121}}{12}\\
=&\dfrac{5\pm 11}{12}=\begin{cases}
x_1=\dfrac{16}{12}=\dfrac{4}{3}\\
x_2=-\dfrac{6}{12}=-\dfrac{1}{2}
\end{cases}\\
\end{align*}
\begin{center}
	\includestandalone{quarto/grafici/disfrazAF1}
\end{center}
L'esercizio chiede quando la frazione è minore di zero quindi la riposta è 
\begin{align*}
x\leq& -\dfrac{1}{2}\\  \dfrac{4}{3}<&x\\
\end{align*}
\end{exercise}
\begin{exercise}
	\tipo{BD}	Risolvere la seguente disequazione $\dfrac{x^2+2x+1}{1-x^2}\geq 0$
		\tcblower
		Risolvere la seguente disequazione $\dfrac{x^2+2x+1}{1-x^2}\geq 0$
	\begin{align*}
-1\leq& x<1\\
	\end{align*}
\end{exercise}
\begin{exercise}
	\tipo{CC}	Risolvere la seguente disequazione $\dfrac{3x^2+2x+1}{x^2-x+6}> 0$
\tcblower
Risolvere la seguente disequazione $\dfrac{3x^2+2x+1}{x^2-x+6}> 0$
\begin{align*}
x^2-x+6>&0\\
x^2-x+6=&0\\
x_{1,2}=&\xunodue{1}{-1}{+6}=\dfrac{1\pm\sqrt{1-24}}{2}\\
=&\dfrac{1\pm\sqrt{-23}}{2}
\qquad\text{nessuna soluzione}\\
3x^2+2x+1>&0\\
3x^2+2x+1=&0\\
x_{1,2}=&\xunodue{3}{+2}{1}=\dfrac{-2\pm\sqrt{4-12}}{6}\\
=&\dfrac{-2\pm\sqrt{-8}}{6}
\qquad\text{nessuna soluzione}\\
\end{align*}
\begin{center}
	\includestandalone{quarto/grafici/disfrazCC1}
\end{center}
Sempre verificata.
\end{exercise}
\begin{exercise}
	\tipo{BB}	Risolvere la seguente disequazione $\dfrac{x^2+4x+4}{x^2+2x+1}< 0$
		\tcblower
Risolvere la seguente disequazione $\dfrac{x^2+4x+4}{x^2+2x+1}< 0$

Nessuna soluzione
\end{exercise}
\begin{exercise}
	\tipo{CF}	Risolvere la seguente disequazione $\dfrac{2x^2-x+1}{x-2-3x^2}> 0$
	\tcblower
		Risolvere la seguente disequazione $\dfrac{2x^2-x+1}{x-2-3x^2}> 0$
	\begin{align*}
	x-2-3x^2>&0\\
	x-2-3x^2=&0\\
	x_{1,2}=&\xunodue{-3}{+1}{-2}=\dfrac{-1\pm\sqrt{1-24}}{-6}\\
	=&\dfrac{1\pm\sqrt{-23}}{-6}
	\qquad\text{nessuna soluzione}\\
	2x^2-x+1>&0\\
	2x^2-x+1=&0\\
	x_{1,2}=&\xunodue{2}{-1}{1}=\dfrac{1\pm\sqrt{1-8}}{4}\\
	=&\dfrac{-1\pm\sqrt{-7}}{4}
	\qquad\text{nessuna soluzione}\\
	\end{align*}
	\begin{center}
		\includestandalone{quarto/grafici/disfrazCF1}
	\end{center}
Mai verificata
\end{exercise}
\begin{exercise}
	\tipo{BF}	Risolvere la seguente disequazione $\dfrac{4x^2+4x+1}{2x-8-x^2}\leq 0$
	\tcblower
	Risolvere la seguente disequazione $\dfrac{4x^2+4x+1}{2x-8-x^2}\leq 0$
	
	Sempre verificata.
\end{exercise}
\begin{exercise}
	\tipo{BD}	Risolvere la seguente disequazione $\dfrac{25x^2-20x+4}{10x-16x^2+1}> 0$
		\tcblower
	Risolvere la seguente disequazione $\dfrac{25x^2-20x+4}{10x-16x^2+1}> 0$
\begin{align*}
10x-16x^2+1>&0\\
10x-16x^2+1=&0\\
x_{1,2}=&\xunodue{-16}{+10}{1}=\dfrac{-10\pm\sqrt{100-64}}{-32}\\
=&\dfrac{-10\pm\sqrt{36}}{-32}\\
=&\dfrac{-10\pm 6}{-32}=\begin{cases}
x_1=\dfrac{-16}{-32}=\dfrac{1}{2}\\
x_2=\dfrac{-4}{-32}=\dfrac{1}{8}
\end{cases}\\
25x^2-20x+4>&0\\
25x^2-20x+4=&0\\
x_{1,2}=&\xunodue{25}{-20}{+4}=\dfrac{20\pm\sqrt{400-400}}{100}\\
=&\dfrac{20\pm\sqrt{0}}{100}=\dfrac{20}{100}\\
=&\dfrac{1}{5}\\
\end{align*}
	\begin{center}
	\includestandalone{quarto/grafici/disfrazBD1}
\end{center}
La frazione è positiva per
\begin{align*}
\dfrac{1}{8}<&x<\dfrac{1}{5}\\ \dfrac{1}{5}<&x<\dfrac{1}{2}\\
\end{align*}
\end{exercise}
\begin{exercise}
	\tipo{DC}	Risolvere la seguente disequazione $\dfrac{3x+1-4x^2}{2x^2-3x+4}\geq 0$
	\tcblower
	Risolvere la seguente disequazione $\dfrac{3x+1-4x^2}{2x^2-3x+4}\geq 0$
	\begin{align*}
	-\dfrac{1}{4}\leq&x\leq 1
	\end{align*}
\end{exercise}
\begin{exercise}
	\tipo{CA}	Risolvere la seguente disequazione $\dfrac{x^2+1}{x^2-x}\geq0$
	\tcblower
	Risolvere la seguente disequazione $\dfrac{x^2+1}{x^2-x}\geq0$	
\begin{align*}
x^2-x>&0\\
x^2-x=&0\\
x_{1,2}=&\xunodue{1}{+1}{0}=\dfrac{1\pm\sqrt{1-0}}{2}\\
=&\dfrac{1\pm\sqrt{1}}{2}\\
=&\dfrac{1\pm 1}{2}=\begin{cases}
	x_1=\dfrac{2}{2}=1\\
	x_2=\dfrac{0}{2}=0
\end{cases}\\
x^2+1>&0\\
x^2+1=&0\\
x_{1,2}=&\xunodue{1}{0}{1}=\dfrac{0\pm\sqrt{0-4}}{2}\\
=&\dfrac{0\pm\sqrt{-4}}{2}
\qquad\text{nessuna soluzione}\\
\end{align*}
\begin{center}
\includestandalone{quarto/grafici/disfrazCA1}
\end{center}
La frazione è positiva ma non uguale a zero per
\begin{align*}
x<&0\\ 1<&x\\
\end{align*}
\end{exercise}
\begin{exercise}
	\tipo{BC}	Risolvere la seguente disequazione $\dfrac{x^2}{2x^2+2x+7}> 0$
	\tcblower
		Risolvere la seguente disequazione $\dfrac{x^2}{2x^2+2x+7}> 0$
	\begin{align*}
	2x^2+2x+7>&0\\
	2x^2+2x+7=&0\\
	x_{1,2}=&\xunodue{2}{+2}{7}=\dfrac{-2\pm\sqrt{4-32}}{4}\\
	=&\dfrac{-2\pm\sqrt{-28}}{4}
	\qquad\text{nessuna soluzione}\\
	x^2>&0\\
	x^2=&0\\
	x_{1,2}=&\xunodue{1}{0}{0}=\dfrac{0\pm\sqrt{0-0}}{2}\\
	=&\dfrac{0\pm\sqrt{0}}{2}=\dfrac{0}{2}=0\\
	\end{align*}
	\begin{center}
		\includestandalone{quarto/grafici/disfrazBC1}
	\end{center}	
		\begin{align*}
		\intertext{Riposte equivalenti}
		x<&0\\ 0<&x\\
		\intertext{oppure}
		x\neq&0\qquad\text{Sempre positiva}
		\end{align*}
\end{exercise}
\begin{exercise}
	\tipo{DE}	Risolvere la seguente disequazione $\dfrac{1-6x^2-3x}{-x^2-6x-9}\geq 0$
	\tcblower
	Risolvere la seguente disequazione $\dfrac{1-6x^2-3x}{-x^2-6x-9}\geq 0$
	\begin{align*}
	-x^2-6x-9>&0\\
	-x^2-6x-9=&0\\
	x_{1,2}=&\xunodue{-1}{-6}{7}=\dfrac{6\pm\sqrt{36-36}}{-2}\\
	=&\dfrac{6\pm\sqrt{0}}{-2}=\dfrac{6}{-2}=-3\\
	1-6x^2-3x>&0\\
	1-6x^2-3x=&0\\
	x_{1,2}=&\xunodue{-6}{-3}{1}=\dfrac{1\pm\sqrt{1+24}}{-12}\\
	=&\dfrac{1\pm\sqrt{25}}{-12}=\dfrac{1\pm 5}{-12}=\begin{cases}
	x_1=\dfrac{6}{-12}=-\dfrac{1}{2}\\
	x_2=\dfrac{-4}{-12}=\dfrac{1}{3}
	\end{cases}\\
	\end{align*}
	\begin{center}
		\includestandalone{quarto/grafici/disfrazDE1}
	\end{center}
La frazione è positiva per 
	\begin{align*}
	x<&-3\\ -3<&x<-\dfrac{1}{2}\\ \dfrac{1}{3}\leq x\\
	\end{align*}
\end{exercise}
\begin{exercise}
	\tipo{DA}	Risolvere la seguente disequazione $\dfrac{21-10x^2+x}{x^2+10x+21}< 0$
	\tcblower
	Risolvere la seguente disequazione $\dfrac{21-10x^2+x}{x^2+10x+21}< 0$
	\begin{align*}
	x<&-7\\ -3<&x<-\dfrac{7}{5}\\ \dfrac{2}{3}< x\\
	\end{align*}
\end{exercise}
\begin{exercise}
	\tipo{FA}	Risolvere la seguente disequazione $\dfrac{x^2-2x+3}{3+5x-3x^2}\geq 0$
\tcblower
Risolvere la seguente disequazione $\dfrac{x^2-2x+3}{3+5x-3x^2}\geq 0$
\begin{align*}
3+5x-3x^2>&0\\
3+5x-3x^2=&0\\
x_{1,2}=&\xunodue{-3}{5}{+3}=\dfrac{-5\pm\sqrt{25+24}}{-6}\\
=&\dfrac{-5\pm\sqrt{49}}{-6}\\
=&\dfrac{-5\pm 7}{12}=\begin{cases}
x_1=\dfrac{-12}{-6}=2\\
x_2=\dfrac{1}{-6}=-\dfrac{1}{6}
\end{cases}\\
x^2-2x+3\geq&0\\
x^2-2x+3=&0\\
x_{1,2}=&\xunodue{1}{-2}{+3}=\dfrac{2\pm\sqrt{4-12}}{2}\\
=&\dfrac{2\pm\sqrt{-8}}{2}
\qquad\text{nessuna soluzione}\\
\end{align*}
\begin{center}
	\includestandalone{quarto/grafici/disfrazFA1}
\end{center}
La frazione è positiva ma non nulla per 
\begin{align*}
-\dfrac{1}{6}<&x<2\\
\end{align*}
\end{exercise}
\begin{exercise}
	\tipo{CB}	Risolvere la seguente disequazione $\dfrac{4x+x^2+8}{1-6x+9x^2}< 0$
	\tcblower
	Risolvere la seguente disequazione $\dfrac{4x+x^2+8}{1-6x+9x^2}< 0$
	
	Mai verificata.
\end{exercise}
\begin{exercise}
	\tipo{DB}	Risolvere la seguente disequazione $\dfrac{17x-3x^2-10}{25-30x+9x^2}\geq 0$
	\tcblower
	Risolvere la seguente disequazione $\dfrac{17x-3x^2-10}{25-30x+9x^2}\geq 0$
	\begin{align*}
25-30x+9x^2>&0\\
25-30x+9x^2=&0\\
x_{1,2}=&\xunodue{9}{-30}{25}=\dfrac{30\pm\sqrt{900-900}}{18}\\
=&\dfrac{30\pm\sqrt{0}}{18}=\dfrac{30}{18}=\dfrac{5}{3}\\
17x-3x^2-10>&0\\
17x-3x^2-10=&0\\
x_{1,2}=&\xunodue{-3}{+17}{-10}=\dfrac{-17\pm\sqrt{289-120}}{-6}\\
=&\dfrac{-17\pm\sqrt{169}}{-6}\\
=&\dfrac{-17\pm 13}{-6}=\begin{cases}
x_1=\dfrac{-30}{-6}=5\\
x_2=\dfrac{-4}{-6}=\dfrac{2}{3}
\end{cases}\\
	\end{align*}
Positiva ed uguale a zero per
	\begin{align*}
	\dfrac{2}{3}\leq&x<\dfrac{5}{3}\\
	\dfrac{5}{3}<&x\leq 6\\
	\end{align*}
\begin{center}
	\includestandalone{quarto/grafici/disfrazDB1}
\end{center}
\end{exercise}
\begin{exercise}
	\tipo{FB}	Risolvere la seguente disequazione $\dfrac{2x-x^2-3}{9x^2+16-24x}> 0$
	\tcblower
	
	Mai verificata.
\end{exercise}
\begin{exercise}
	\tipo{AC}	Risolvere la seguente disequazione $\dfrac{3x^2+x-2}{5+4x^2-3x}\leq 0$
	\tcblower
	Risolvere la seguente disequazione $\dfrac{3x^2+x-2}{5+4x^2-3x}\leq 0$
	\begin{align*}
	5+4x^2-3x>&0\\
5+4x^2-3x=&0\\
x_{1,2}=&\xunodue{4}{-3}{5}=\dfrac{3\pm\sqrt{9-80}}{8}\\
=&\dfrac{3\pm\sqrt{-71}}{8}
\qquad\text{nessuna soluzione}\\
3x^2+x-2>&0\\
3x^2+x-2=&0\\
x_{1,2}=&\xunodue{3}{+1}{-2}=\dfrac{-1\pm\sqrt{1+24}}{6}\\
=&\dfrac{-1\pm\sqrt{25}}{6}\\
=&\dfrac{-1\pm 5}{6}=\begin{cases}
x_1=\dfrac{6}{6}=1\\
x_2=\dfrac{-4}{6}=-\dfrac{2}{3}
\end{cases}\\
	\end{align*}
	\begin{center}
		\includestandalone{quarto/grafici/disfrazCA2}
	\end{center}
assume valori negativi ed uguale a zero per:
\begin{align*}
-\dfrac{2}{3}\leq&x\leq 1\\
\end{align*}
\end{exercise}
\begin{exercise}
	\tipo{FC}	Risolvere la seguente disequazione $\dfrac{x^2+5-3x}{3x-x^2-4}<0$
	\tcblower
	Risolvere la seguente disequazione $\dfrac{x^2+5-3x}{3x-x^2-4}< 0$
	
	Sempre verificata.
\end{exercise}
\begin{exercise}[no solution]
	\tipo{CD}	Risolvere la seguente disequazione $\dfrac{3x^2+2+3x}{7x-2-5x^2}\geq0$
\end{exercise}
\begin{exercise}[no solution]
	\tipo{CE}	Risolvere la seguente disequazione $\dfrac{6-7x-5x^2}{-25x^2-10x-1}< 0$
\end{exercise}
    \section{Soluzioni esercizi}
\tcbstoprecording
% \newpage
%\section{Soluzioni esercizi}
\tcbinputrecords
    \tcbstartrecording
\section{Disequazioni di primo e secondo grado fratte}
\begin{exercise}
	\tipo{AP}	Risolvere la seguente disequazione $\dfrac{6x^2+13x+6}{2x+1}\geq0$
	\tcblower
	Risolvere la seguente disequazione $\dfrac{6x^2+13x+6}{2x+1}\geq0$
	\begin{align*}
	2x+1>&0\\
	2x>&-1\\
	x>&-\dfrac{1}{2}\\
	x_{1,2}=&\xunodue{6}{+13}{+6}=\dfrac{-13\pm\sqrt{169-154}}{12}\\
	=&\dfrac{-13\pm\sqrt{25}}{12}=\dfrac{-13\pm 5}{12}\\
	=&\begin{cases}
	x_1=-\dfrac{18}{12}=-\dfrac{3}{2}\\
	x_2=-\dfrac{8}{12}=-\dfrac{2}{3}
	\end{cases}\\
	\end{align*}
	\begin{center}
		\includestandalone{quarto/grafici/disfrazAP1}
	\end{center}
L'esercizio chiede quando la frazione è maggiore o uguale a zero quindi la risposta è 
\begin{align*}
-\dfrac{3}{2}<x&< -\dfrac{2}{3}\\ x\geq&-\dfrac{1}{2}\\
\end{align*}
\end{exercise}
\begin{exercise}
	\tipo{DP}	Risolvere la seguente disequazione $\dfrac{10-2x^2-x}{2-4x}< 0$
	\tcblower
	Risolvere la seguente disequazione $\dfrac{10-2x^2-x}{2-4x}< 0$
	\begin{align*}
		2-4x>&0\\
		-4x>&-2\\
		x<&\dfrac{1}{2}\\
		x_{1,2}=&\xunodue{-2}{-1}{+10}=\dfrac{1\pm\sqrt{1+80}}{-4}\\
		=&\dfrac{1\pm\sqrt{81}}{-4}=\dfrac{1\pm 9}{-4}\\
		=&\begin{cases}
			x_1=\dfrac{-8}{-4}=2\\
			x_2=\dfrac{10}{-4}=-\dfrac{5}{2}
		\end{cases}\\
	\end{align*}
	\begin{center}
		\includestandalone{quarto/grafici/disfrazDP1}
	\end{center}
	L'esercizio chiede quando la frazione è minore di zero quindi la risposta è 
	\begin{align*}
		\dfrac{1}{2}<x&<2\\ x<&-\dfrac{5}{2}\\
	\end{align*}
\end{exercise}
\begin{exercise}
	\tipo{PF}	Risolvere la seguente disequazione $\dfrac{1-3x}{2x-3x^2-2}\leq 0$
	\tcblower
	Risolvere la seguente disequazione $\dfrac{1-3x}{2x-3x^2-2}\leq 0$
	\begin{align*}
	x_{1,2}=&\xunodue{-3}{+2}{-2}=\dfrac{-2\pm\sqrt{4-24}}{-6}\\
	=&\dfrac{-2\pm\sqrt{-20}}{-6}=\dfrac{1\pm 9}{-4}\\
\qquad\text{nessuna soluzione}\\
		1-3x&\geq0\\
	-3x\geq&-1\\
	x\leq&\dfrac{1}{3}\\
	\end{align*}
	\begin{center}
		\includestandalone{quarto/grafici/disfrazPF1}
	\end{center}
	L'esercizio chiede quando la frazione è minore o uguale zero quindi la risposta è 
	\begin{align*}
	 x\leq&\dfrac{1}{3}\\
	\end{align*}
\end{exercise}
\begin{exercise}
	\tipo{AP}	Risolvere la seguente disequazione $\dfrac{x+2-15x^2}{3x+1}\leq0$
	\tcblower
	Risolvere la seguente disequazione $\dfrac{x+2-15x^2}{3x+1}\leq0$
	\begin{align*}
	3x+1>&0\\
	2x>&-1\\
	x>&-\dfrac{1}{3}\\
	x_{1,2}=&\xunodue{-15}{+1}{+2}=\dfrac{-1\pm\sqrt{1+120}}{-30}\\
	=&\dfrac{-1\pm\sqrt{121}}{-30}=\dfrac{-1\pm 11}{-30}\\
	=&\begin{cases}
	x_1=-\dfrac{10}{30}=-\dfrac{1}{3}\\
	x_2=\dfrac{-12}{-30}=\dfrac{2}{5}
	\end{cases}\\
	\end{align*}
	\begin{center}
		\includestandalone{quarto/grafici/disfrazAP2}
	\end{center}
	L'esercizio chiede quando la frazione è minore a zero quindi la risposta è 
	\begin{align*}
	x<&-\dfrac{1}{3}\\
	-\dfrac{1}{3}<x&\leq \dfrac{2}{5}\\ 
	\end{align*}
\end{exercise}
    \section{Soluzioni esercizi}
\tcbstoprecording
% \newpage
%\section{Soluzioni esercizi}
\tcbinputrecords
	%\chapter{Funzioni reali}
\label{cha:FunzioniEquazioniEsponenziali}
\section{Funzioni di variabili reali}
\label{sec:FunzioniVariabileReale}
Una prima definizione di funzione è la seguente:
\begin{definizionet}{Funzione}{}
Una funzione è una relazione che associa ad un elemento di un insieme (Dominio\index{Funzione!Dominio}) uno e un solo elemento di un altro insieme (Codominio\index{Funzione!Codominio}). 
\end{definizionet}
Se $f$ è la funzione\index{Funzione} avremo \[\function{f}{D}{C}{x}{f(x)}\]
dove $f(x)$ è l'immagine\index{Funzione!Immagine} tramite $f$ di $x$.
\begin{osservazionet}{}{}
 Il biglietto del cinema è una funzione fra l'insieme degli spettatori (Dominio) e l'insieme delle poltrone di un cinema (Codominio). La figura\nobs\vref{fig:funzioniEsempio1} mostra questo. 
 \end{osservazionet}
\begin{figure}
	\centering
	\begin{subfigure}[b]{.4\linewidth}
		\centering
		\includestandalone[width=\textwidth]{quarto/funzioniBase/esempio1}
		\caption{Biglietto del cinema}
		\label{fig:funzioniEsempio1}
	\end{subfigure}\qquad
	\centering
		\begin{subfigure}[b]{.4\linewidth}
			\centering
			\includestandalone[width=\textwidth]{quarto/funzioniBase/esempio2}
			\caption{Essere padre di}
			\label{fig:funzioniEsempio2}
		\end{subfigure}%
%	\caption{$\Delta>0$}
%	\label{fig:DeltaMagZeroEsempio1}
\end{figure}
\begin{osservazionet}{}{}
La relazione <<essere padre di>> definita fra l'inseme dei dei padri e l'insieme dei figli, non è una funzione. Un padre infatti può avere più di un figlio come mostrato nella  figura\nobs\vref{fig:funzioniEsempio1}.
\end{osservazionet}
\begin{osservazionet}{}{}
La relazione <<essere figlio di>> fra l'insieme dei figli e l'insieme e quello delle madri è una funzione. Un figlio ha una sola madre. Come nella figura\nobs\vref{fig:funzioniEsempio3}. Due figli possono avere la stessa madre ma un figlio ha sempre una sola madre.
\end{osservazionet}
\begin{figure}
	\centering
	\begin{subfigure}[b]{.4\linewidth}
		\centering
		\includestandalone[width=\textwidth]{quarto/funzioniBase/esempio3}
		\caption{Essere figlio di}
		\label{fig:funzioniEsempio3}
	\end{subfigure}\qquad
	%	\centering
	%	\begin{subfigure}[b]{.4\linewidth}
	%		\centering
	%		\includestandalone[width=\textwidth]{quarto/funzioniBase/esempio4}
	%		\caption{Essere padre di}
	%		\label{fig:funzioniEsempio4}
	%	\end{subfigure}%
	%	%	\caption{$\Delta>0$}
	%	%	\label{fig:DeltaMagZeroEsempio1}
\end{figure}
\begin{osservazionet}{}{}
La relazione che associa ad un numero il suo quadrato è una funzione. Formalmente abbiamo \[\function{f}{\R}{\R^{+}_{0}}{x}{x^2} \] oppure possiamo scrivere $y=x^2$. In questo caso abbiamo due incognite, la $x$ viene chiamata variabile indipendente, la $y$ è detta variabile dipendente. In questo caso il dominio della funzione è $\R$ e il codominio $\R^{+}_{0}$ cioè l'insieme dei numeri reali positivo incluso lo zero. La figura\nobs\vref{fig:funzioniEsempio4} mostra alcuni valori utilizzati per l'esempio. Alla funzione è possibile associare un grafico cioè l'insieme delle coppie $x,x^2$. Il grafico\nobs\vref{fig:funzioniEsempio9} è una parabola. 
\end{osservazionet}
\begin{figure}
	\centering
	\begin{subfigure}[b]{.4\linewidth}
		\centering
		\includestandalone[width=\textwidth]{quarto/funzioniBase/esempio4}
		\caption{Quadrato}
		\label{fig:funzioniEsempio4}
	\end{subfigure}\qquad
	\centering
	\begin{subfigure}[b]{.4\linewidth}
		\centering
		\includestandalone[width=\textwidth]{quarto/funzioniBase/esempio5}
		\caption{Radice quadrata}
		\label{fig:funzioniEsempio5}
	\end{subfigure}%
	%	\caption{$\Delta>0$}
	%	\label{fig:DeltaMagZeroEsempio1}
\end{figure}
\begin{figure}
	\centering
	\begin{subfigure}[b]{.4\linewidth}
		\centering\includestandalone[width=\textwidth]{quarto/funzioniBase/esempio9}
		
		\caption{Radice quadrata}
		\label{fig:funzioniEsempio9}
	\end{subfigure}\qquad
	\centering
	\begin{subfigure}[b]{.4\linewidth}
		\centering
		\includestandalone[width=\textwidth]{quarto/funzioniBase/esempio8}
		\caption{Grafico quadrato}
		\label{fig:funzioniEsempio8}
	\end{subfigure}%
	%	\caption{$\Delta>0$}
	%	\label{fig:DeltaMagZeroEsempio1}
\end{figure}
\begin{osservazionet}{}{}
	La relazione che associa ad un numero la sua radice non è una funzione. Formalmente abbiamo \[\function{f}{\R}{\R}{x}{\sqrt{x}} \] La figura\nobs\vref{fig:funzioniEsempio5} mostra alcuni valori utilizzati per l'esempio. Dalla figura è chiaro che in questo caso non è una funzione. La relazione diventa una funzione se modifichiamo per esempio il codominio da $\R$ a $\R^{+}$. In questo caso ad un valore in ingresso corrisponde un solo valore in uscita. Il grafico\nobs\vref{fig:funzioniEsempio9} mostra il grafico della radice. 
\end{osservazionet}
%\section{Classificazione delle funzioni}
%Una prima rozza classificazione delle funzioni matematiche le divide in due gruppi: intere o fratte.  
\section{La funzione esponenziale}
\label{sec:LaFunzioneEsponenziale}
\subsection{Proprietà delle potenze}
\label{subsec:ProprietaDellePotenze}
Iniziamo con ripassare le proprietà delle potenze. La tabella\nobs\vref{tab:potenzeriepilogo} ne elenca le principali proprietà. Per le potenze a esponente reale la base deve un numero reale positivo. Se questo non viene rispettato, possiamo avere delle situazioni paradossali come la seguente:
\begin{cesempiot}{Base negativa}{}
\[-2=\sqrt[3]{-8}=(-8)^{\frac{1}{3}}=(-8)^{\frac{2}{6}}=\sqrt[6]{(-8)^{2}}=\sqrt[6]{64}=2\]
\end{cesempiot}
 \begin{table}
	\centering
	\begin{tabular}{RCLL}
		\toprule
		a^n&=&\overbrace{a\times a\times\cdots\times a}^{n{}\mbox{volte}}&\forall a\in\R\qquad n\in\Ni\qquad n>1 \\[.6cm]
		a^0&=&1&\forall a\in\R\qquad a\neq 0\\[.6cm]
		0^0&=&?\\[.6cm]
		a^1&=&a&\forall a\in\R\\[.6cm]
		
		a^n\cdot a^m&=&a^{n+m}&\forall a\in\R\qquad n,m\in\Ni\\[.6cm]
		a^m\div a^n&=&a^{m-n}&\forall a\in\R\qquad a\neq 0\qquad n,m\in\Ni\\[.6cm]
		\left(a^n\right)^m&=&a^{n\cdot m}&\forall a\in\R\\[.6cm]
		a^{n}b^{n}&=&{\left(ab\right)}^n& a,b\in\R\qquad n\in\Ni\\[.6cm]
		a^{n}\div b^{n}&=&{\left(a\div  b\right)}^n& a,b\in\R\qquad b\neq 0\qquad n\in\Ni\\[.6cm]
		\left(\dfrac{a}{b}\right)^n&=&\dfrac{a^n}{b^n}& a,b\in\R\qquad b\neq 0\qquad n\in\Ni\\[.6cm]
		a^{-n}&=&\left(\dfrac{1}{a}\right)^n&\forall a\in\R\qquad a\neq 0\qquad n\in\Ni \\[.6cm]
		a^{\frac{m}{n}}&=&\sqrt[n]{a^{m}}&\forall a\in\R\qquad a\geq 0\qquad n,m\in\Ni \\[.6cm]
		\bottomrule
	\end{tabular}
	\caption{Proprietà delle potenze}
	\label{tab:potenzeriepilogo}
\end{table}
\begin{definizionet}{Funzione esponenziale}{}
	Chiamo funzione esponenziale\index{Funzione!Esponenziale} una funzione del tipo \[y=a^x\quad
	\text{per $a>0$.} \]
\end{definizionet}
Il comportamento della funzione esponenziale varia al variare della base. La figura\nobs\vref{fig:funzioniEsempio6} mostra il comportamento per $a>1$.
\begin{enumerate}
\item Il grafico della funzione esponenziale occupa il semipiano positivo delle y.
\item Tutti i grafici passano per il punto $(0,1)$.
\item La funzione è crescente\index{Funzione!Crescente} all'aumentare dell'incognita. $x_1<x_2\quad f(x_1)<f(x_2)$ 
\item L'asse delle $x$ è un asintoto orizzontale\index{Asintoto!orizzontale}.
\end{enumerate}
 La figura\nobs\vref{fig:funzioniEsempio7} mostra il comportamento della funzione esponenziale per $0<a<1$. 
 \begin{enumerate}
 \item Il grafico della funzione esponenziale occupa il semipiano positivo delle y.
 \item Tutti i grafici passano per il punto $(0,1)$.
 \item La funzione è  decrescente\index{Funzione!Decrescente} all'aumentare dell'incognita. $x_1<x_2\quad f(x_1)>f(x_2)$ 
 \item L'asse delle $x$ è un asintoto orizzontale\index{Asintoto!orizzontale}.
 \end{enumerate}
\begin{figure}
\centering
%\begin{center}
\begin{tikzpicture}[>=triangle 90]
\begin{axis}
[xmin=-6,xmax=6,ymin=-1,ymax=6, %grid,
axis x line=middle,xtick={-6,-5,...,6},ytick={-6,-5,...,6},
axis y line=middle,xlabel=$x$,ylabel=$y$]
\addplot[samples=300] {(0.5^(x))};
\addplot [samples=300] {(2^(x))};

\end{axis}
\node (a1) at (1,1.5) {$a>1$};
\node (a2) at (6,1.5) {$0<a<1$};
\end{tikzpicture}
\caption{Funzione esponenziale riepilogo}
\label{fig:FunzioneExp}
%\end{center}
\end{figure}
\begin{figure}
	\centering
	\begin{subfigure}[b]{.4\linewidth}
		\centering
		\includestandalone[width=\textwidth]{quarto/funzioniBase/esempio6}
		\caption{$a>1$}
		\label{fig:funzioniEsempio6}
	\end{subfigure}\qquad
	\centering
	\begin{subfigure}[b]{.4\linewidth}
		\centering
		\includestandalone[width=\textwidth]{quarto/funzioniBase/esempio7}
		\caption{$0<a<1$}
		\label{fig:funzioniEsempio7}
	\end{subfigure}%
		\caption{Funzione esponenziale}
		\label{fig:funzExp2}
\end{figure}
\section{Logaritmo}
\label{sec:Lograritmo}
%Un'equazione esponenziale non ha sempre una soluzione per esempio \[10^x=20\] In questo caso la soluzione esiste ma non è nota, chiamo logaritmo\index{Logaritmo} di venti in base dieci l'esponente che bisogna dare a dieci per ottenere venti e lo indico con $\log_{10}20$
\begin{definizionet}{Logaritmo}{}
	Chiamo logaritmo in base $a$ di $b$ l'esponente che bisogna dare ad $a$ per ottenere $b$ \[x=\log_ba\Leftrightarrow b^{x}=a\quad a>0\quad a\neq  1\quad b>0 \]
\end{definizionet} 
\section{Funzione Logaritmica}
\label{FunzioneLogaritmica}
La funzione $y=log_ax$ si chiama funzione logaritmica\index{Funzione!Logaritmica} è fortemente collegata alla funzione esponenziale $y=a^x$. La figura\nobs\vref{fig:funzioniLogEsempio1} confronta la funzione esponenziale e la funzione logaritmica  quando la base è maggiore di uno $a>1$. La curva è simmetrica rispetto alla bisettrice del primo terzo quadrante. 
\begin{enumerate}
	\item Il grafico della funzione logaritmo occupa il semipiano positivo delle x.
	\item Tutti i grafici passano per il punto $(1,0)$.
	\item La funzione è crescente\index{Funzione!Crescente} all'aumentare dell'incognita. $x_1<x_2\quad f(x_1)<f(x_2)$ 
	\item L'asse delle $y$ è un asintoto verticale\index{Asintoto!verticale}.
\end{enumerate} 
\begin{figure}
	\centering
	\begin{subfigure}[b]{.4\linewidth}
		\centering
		\includestandalone[width=\textwidth]{quarto/FunzioniLog/esempio1}
		\caption{$a>1$}
		\label{fig:funzioniLogEsempio1}
	\end{subfigure}\qquad
	\centering
	\begin{subfigure}[b]{.4\linewidth}
		\centering
		\includestandalone[width=\textwidth]{quarto/FunzioniLog/esempio2}
		\caption{$0<a<1$}
		\label{fig:funzioniLogEsempio2}
	\end{subfigure}%
	\caption{Funzioni esponenziali e logaritmiche}
	\label{fig:funzExp1}
\end{figure}
La figura\nobs\vref{fig:funzioniLogEsempio2} confronta la funzione esponenziale e la funzione logaritmica  quando la base è compresa fra zero e uno $0<a<1$ . La curva è simmetrica rispetto alla bisettrice del primo terzo quadrante. 
\begin{enumerate}
	\item Il grafico della funzione logaritmo occupa il semipiano positivo delle x.
	\item Tutti i grafici passano per il punto $(1,0)$.
	\item La funzione è decrescente\index{Funzione!Decrescente} all'aumentare dell'incognita. $x_1<x_2\quad f(x_1)>f(x_2)$ 
	\item L'asse delle $y$ è un asintoto verticale\index{Asintoto!verticale}.
\end{enumerate}
La figura\nobs\vref{fig:FunzioniLogEsempio3} confranta fre di loro i due grafici.
\begin{figure}
	\centering
	\includestandalone[width=0.6\linewidth]{quarto/FunzioniLog/esempio3}
	%\includegraphics[width=0.7\linewidth]{./}
	\caption{Funzioni logaritmiche}
	\label{fig:FunzioniLogEsempio3}
\end{figure}
\section{Razionale fratta}
\begin{definizionet}{Funzione razionale fratta}{}
	Chiamo funzione razionale\index{Funzione!Razionale fratta} una funzione del tipo \[y=\dfrac{A(x)}{B(x)}\] con $A(x)$ e $B(x)$ due polinomi.
\end{definizionet}
	%\chapter{Operazioni con i Logaritmi}
\label{sec:Logaritmi}

\section{Proprietà del logaritmi}
Le proprietà del logaritmi sono legate alle proprietà delle potenze. $\log_ab=x$ è l'esponente $x$ che bisogna dare ad $a$ per ottenere $b$ quindi $x=\log_ab\Leftrightarrow a^{x}=b$. 
\label{sec:ProprietadelLogaritmi}
%\begin{table}
%	\centering
%	%\begin{center}
%	\begin{tikzpicture}[line cap=round,line join=round,>=triangle 45,x=1.0cm,y=1.0cm]
%	\draw[->,color=black] (-1,0) -- (6.0,0);
%	\foreach \x in {-1,1,2,3,4,5}
%	\draw[shift={(\x,0)},color=black] (0pt,2pt) -- (0pt,-2pt) node[below] {\footnotesize $\x$};
%	\draw[color=black] (5.86,0.02) node [anchor=south west] { x};
%	\draw[->,color=black] (0,-2.13) -- (0,2.18);
%	\foreach \y in {-2,-1,1,2}
%	\draw[shift={(0,\y)},color=black] (2pt,0pt) -- (-2pt,0pt) node[left] {\footnotesize $\y$};
%	\draw[color=black] (0pt,-10pt) node[right] {\footnotesize $0$};
%	\clip(-1,-2.13) rectangle (5.0,2.18);
%	\draw plot[raw gnuplot, id=func0] function{set samples 100; set xrange [0.1:6.0]; plot log(x)/log(2)};
%	\draw plot[raw gnuplot, id=func1] function{set samples 100; set xrange [0.1:6.0]; plot log(x)/log(0.5)};
%	\draw (4.0,1.5) node[anchor=north west] {$\mathbf{log_a x}$};
%	\draw (2.1,-0.78) node[anchor=north west] {$\mathbf{0<a<1}$};
%	\draw (2.1,1.06) node[anchor=north west] {$\mathbf{a>1}$};
%	\end{tikzpicture}
%	%\subfloat[][]{\input{funzLog}}
%	\caption{Funzioni logaritmiche}
%	\label{tab:FunzioneLog}
%	%\end{center}
%\end{table}
\begin{table}
\centering
\begin{tabular}{ll}
\toprule
Proprietà&Formula\\
\midrule
definizione&$x=\log_ba\Leftrightarrow b^{x}=a$\\
&\\
logaritmo della base&$\log_bb=1$\\
&\\
logaritmo dell'unità&$\log_b1=0$\\
&\\
logaritmo del prodotto&$\log_bac=\log_ba+\log_bc$\\
&\\
logaritmo della divisione&$\log_b\dfrac{a}{c}=\log_ba-\log_bc$\\
&\\
logaritmo della potenza&$\log_ba^{n}=n\log_ba$\\
&\\
logaritmo della radice&$\log_b\sqrt[n]{a}=\dfrac{1}{n}log_ba$\\
&\\
cambio di base&$\log_cb=\dfrac{\log_ab}{\log_ac}$\\
&\\
potenza della base&$\log_ab=\log_{a^n}b^{n}$\\
&\\
&$\log_ab\cdot\log_ba=1$\\
\bottomrule
\end{tabular}
\caption{Logaritmi}
\label{tab:logaritmi}
\end{table}
\section{Disequazioni Logaritmiche}
\label{sec:DisequazioniLogaritmiche}
La disequazione logaritmica\index{Disequazione!logaritmica} 
\[\log_{a}f(x) \geq log_{a}g(x)\]
per $a>1$ equivale al sistema
\[\begin{cases}
f(x)>0\\ 
g(x)>0\\ 
f(x)\geq g(x)
\end{cases}\]
per $0<a<1$ equivale al sistema
\[\begin{cases}
f(x)>0\\ 
g(x)>0\\ 
f(x)\leq g(x)
\end{cases}\]

%\chapter{Soluzioni esercizi}
\tcbstoprecording
% \newpage
%\section{Soluzioni esercizi}
\tcbinputrecords
	\backmatter

	
	\addcontentsline{toc}{chapter}{\indexname}
	\printindex
	\printindex[dissec]
	\appendix
	\chapter{Mezzi usati}
	\CDMezziUsati
\end{document}
