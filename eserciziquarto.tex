% !TEX encoding = UTF-8 Unicode
% !TEX TS-program = pdflatex

\documentclass[a4paper,oneside]{book}%
\usepackage{base}
%\geometry{top=2cm,bottom=2cm,left=2cm,right=2cm}
\usepackage[big]{layaureo}
\usepackage{grafica}
\usepackage{matematica}
\usepackage{tabelle}
\DeclareCaptionFormat{grafico}{\textbf{Grafico \thefigure}#2#3}
\DeclareCaptionFormat{esempio}{\textbf{Esempio \thefigure}#2#3}
\usepackage{imakeidx}
\makeindex
\makeindex[name=dissec,title=Disequazioni secondo grado]
\makeindex[options=-s ../Mod_base/oldclaudio.sti]
\usepackage{date}
\usepackage{pagina}
\usepackage{unita_misura}
\usepackage{indice}
\usepackage{utili}
\usepackage{copyright}
\usepackage{CDloghi}
\usepackage{stand_class}

\newcommand{\HRule}{\rule{\linewidth}{0.5mm}}
\usepackage{placeins} 
\makeatletter
\renewcommand\frontmatter{%
	\cleardoublepage
	\@mainmatterfalse
	\pagenumbering{arabic}}
\renewcommand\mainmatter{%
	\cleardoublepage
	\@mainmattertrue}
\makeatother
\newcommand{\tipo}[1]{\textbf{#1}\index[dissec]{#1}}

\usepackage{qrcode}
%%%%%%%%%%%%%%%%%%%%%%%%%%%%%%%%
%%%lunghezza arrotondamenti%%%%%
\newcommand{\lungarrotandamento}{4}
%%%%%%%%%%%%%%%%%%%%%%%%%%%%%%%%%%
\includeonly{%
quarto/Parabola,
quarto/disequazioni_primogrado,
quarto/disequazioni_secondogrado,
quarto/tabelle_disequazioni,
quarto/EseDisSecGrad
}
\usepackage[grumpy,mark,markifdirty,raisemark=0.95\paperheight]{gitinfo2}
\usepackage[toc,page]{appendix}

\renewcommand{\appendixtocname}{Appendice}

\renewcommand{\appendixpagename}{Appendice}

\usepackage[italian]{varioref}
\usepackage{hyperxmp}
\usepackage[pdfpagelabels]{hyperref}
\usepackage[italian]{cleveref}
\usepackage{tcolorboxgest}
\title{Esercizi svolti quarto}
\author{Claudio Duchi}
\date{\datetime}
\hypersetup{%
	pdfencoding=auto,
	urlcolor={blue},
	pdftitle={Esercizi svolti},
	pdfsubject={terzo},
	pdfstartview={FitH},
	pdfpagemode={UseOutlines},
	pdflicenseurl={http://creativecommons.org/licenses/by-nc-nd/3.0/},
	pdflang={it},
	pdfmetalang={it},
	pdfkeywords={Parabola,Disequazioni},
	pdfcopyright={Copyright (C) 2019, Claudio Duchi},
	pdfcontacturl={http://breviariomatematico.altervista.org},
	pdfcontactpostcode={},
	pdfcontactphone={},
	pdfcontactemail={claduc},
	pdfcontactcountry={Italy},
	pdfcontactcity={Perugia},
	pdfcontactaddress={},
	pdfcaptionwriter={Claudio Duchi},
	pdfauthortitle={},%
	pdfauthor={Claudio Duchi},
	linkcolor={blue},
	colorlinks=true,
	citecolor={red},
	breaklinks,
	bookmarksopen,
	verbose,
	baseurl={http://breviariomatematico.altervista.org}
}
\listfiles
\begin{document}
	\frontmatter
	\begin{titlepage}
		\begin{center}
				\Lgrandedue\\[1cm]
			\textsc{\LARGE Claudio Duchi}\\[1.2cm]
			\HRule \\[0.4cm]
			{ \huge \bfseries ESERCIZI SVOLTI DI MATEMATICA}\\[0.4cm]
			{\LARGE \textsc{QUARTO MANUTENZIONE}}
			\HRule \\[1.2cm]
			\polylogo[5.5]{18}		
		{\large $-$\DTMnow$-$}	
	\end{center}
{\centering
Release:\gitReln\ (\gitAbbrevHash)\ Autore:\gitAuthorName\ 
\gitCommitterDate \\
}
	\end{titlepage}
	\setcounter{page}{2} 
	\CDcopyright
	\tableofcontents 
	%\addcontentsline{toc}{chapter}{\listtablename}
	%\listoftables
	\addcontentsline{toc}{chapter}{\listfigurename}
	\listoffigures
	\renewcommand\lstlistlistingname{Esempi e contro esempi}
	\addcontentsline{toc}{chapter}{\lstlistlistingname}
	\addcontentsline{toc}{section}{Esempi}
	\lstlistoflistings%{}
	{
		%	https://tex.stackexchange.com/questions/318486/number-freestyle-causes-an-overlay-in-the-list-of-tcolorboxes/318512#318512
		\makeatletter
		\renewcommand{\l@tcolorbox}{\@dottedtocline{1}{0pt}{3em}}
		\makeatother
		\tcblistof[\section*]{thm}{Esempi}
		\addcontentsline{toc}{section}{Contro esempi}
		\tcblistof[\section*]{cthm}{Contro esempi}
	}
	
	\mainmatter%
	     \include{quarto/disequazioni_primogrado}
	\include{quarto/disequazioni_secondogrado}
	%\include{funzExpLog}
	%\include{logaritmi}

	\backmatter

	
	\addcontentsline{toc}{chapter}{\indexname}
	\printindex
	\printindex[dissec]
	\appendix
	\chapter{Mezzi usati}
	\CDMezziUsati
\end{document}
