% !TEX encoding = UTF-8 Unicode
% !TEX TS-program = pdflatex

\documentclass[openany]{book}%
\input{../Mod_base/base}
\geometry{top=2cm,bottom=1cm,left=2cm,right=2cm}
\input{../Mod_base/grafica}
\input{../Mod_base/matematica}
\input{../Mod_base/tabelle}
\DeclareCaptionFormat{grafico}{\textbf{Grafico \thefigure}#2#3}
\DeclareCaptionFormat{esempio}{\textbf{Esempio \thefigure}#2#3}
\usepackage{imakeidx}
\makeindex
\makeindex[name=dissec,title=Disequazioni secondo grado]
\makeindex[options=-s ../Mod_base/oldclaudio.sti]
\input{../Mod_base/pagina}
\input{../Mod_base/indice}
\input{../Mod_base/date}
\input{../Mod_base/loghi}
\input{../Mod_base/unita_misura}
\input{../Mod_base/utili}
\input{../Mod_base/stand_class}

\newcommand{\HRule}{\rule{\linewidth}{0.5mm}}
\usepackage{placeins} 
\makeatletter
\renewcommand\frontmatter{%
	\cleardoublepage
	\@mainmatterfalse
	\pagenumbering{arabic}}
\renewcommand\mainmatter{%
	\cleardoublepage
	\@mainmattertrue}
\makeatother
\newcommand{\tipo}[1]{\textbf{#1}\index[dissec]{#1}}

\usepackage{qrcode}
%%%%%%%%%%%%%%%%%%%%%%%%%%%%%%%%
%%%lunghezza arrotondamenti%%%%%
\newcommand{\lungarrotandamento}{4}
%%%%%%%%%%%%%%%%%%%%%%%%%%%%%%%%%%
\includeonly{%
quarto/Parabola,
quarto/disequazioni_primogrado,
quarto/disequazioni_secondogrado,
quarto/tabelle_disequazioni,
quarto/EseDisSecGrad
}
\usepackage[grumpy,mark,markifdirty,raisemark=0.95\paperheight]{gitinfo2}
\usepackage[toc,page]{appendix}

\renewcommand{\appendixtocname}{Appendice}

\renewcommand{\appendixpagename}{Appendice}
\usepackage{tkz-berge}
\usepackage[italian]{varioref}
\usepackage{hyperxmp}
\usepackage[pdfpagelabels]{hyperref}
\usepackage[italian]{cleveref}
\input{../Mod_base/tcolorboxgest}
\title{Esercizi svolti quarto}
\author{Claudio Duchi}
\date{\datetime}
\hypersetup{%
	pdfencoding=auto,
	urlcolor={blue},
	pdftitle={Esercizi svolti},
	pdfsubject={terzo},
	pdfstartview={FitH},
	pdfpagemode={UseOutlines},
	pdflicenseurl={http://creativecommons.org/licenses/by-nc-nd/3.0/},
	pdflang={it},
	pdfmetalang={it},
	pdfkeywords={goniometria, trigoniometria, numeri complessi},
	pdfcopyright={Copyright (C) 2018, Claudio Duchi},
	pdfcontacturl={http://breviariomatematico.altervista.org},
	pdfcontactpostcode={},
	pdfcontactphone={},
	pdfcontactemail={claduc},
	pdfcontactcountry={Italy},
	pdfcontactcity={Perugia},
	pdfcontactaddress={},
	pdfcaptionwriter={Claudio Duchi},
	pdfauthortitle={},%
	pdfauthor={Claudio Duchi},
	linkcolor={blue},
	colorlinks=true,
	citecolor={red},
	breaklinks,
	bookmarksopen,
	verbose,
	baseurl={http://breviariomatematico.altervista.org}
}
\listfiles
\begin{document}
	\frontmatter
	\begin{titlepage}
		\begin{center}
			\input{../Mod_base/Lgrande}\\[1cm]
			\textsc{\LARGE Claudio Duchi}\\[1.5cm]
			\HRule \\[0.4cm]
			{ \huge \bfseries ESERCIZI SVOLTI DI MATEMATICA}\\[0.4cm]
			{\LARGE \textsc{QUARTO MANUTENZIONE}}
			\HRule \\[1.5cm]
			\vfill
			\begin{tikzpicture}
			\renewcommand*{\VertexBallColor}{green!50!black}
			\GraphInit[vstyle=Art]
			\grComplete[RA=5]{18}
			\end{tikzpicture}
		\end{center}
		{\centering
			Release:\gitReln\ (\gitAbbrevHash)\ Autore:\gitAuthorName\ 
			\gitCommitterDate \\
		}
	\end{titlepage}
	\setcounter{page}{2} 
	\input{../Mod_base/copyright}
	\tableofcontents 
	%\addcontentsline{toc}{chapter}{\listtablename}
	%\listoftables
	\addcontentsline{toc}{chapter}{\listfigurename}
	\listoffigures
	\renewcommand\lstlistlistingname{Esempi e contro esempi}
	\addcontentsline{toc}{chapter}{\lstlistlistingname}
	\addcontentsline{toc}{section}{Esempi}
	\lstlistoflistings%{}
	{
		%	https://tex.stackexchange.com/questions/318486/number-freestyle-causes-an-overlay-in-the-list-of-tcolorboxes/318512#318512
		\makeatletter
		\renewcommand{\l@tcolorbox}{\@dottedtocline{1}{0pt}{3em}}
		\makeatother
		\tcblistof[\section*]{thm}{Esempi}
		\addcontentsline{toc}{section}{Contro esempi}
		\tcblistof[\section*]{cthm}{Contro esempi}
	}
	
	\mainmatter%
	     \include{quarto/disequazioni_primogrado}
	\include{quarto/disequazioni_secondogrado}
	%\include{funzExpLog}
	%\include{logaritmi}

	\backmatter
	\begin{appendices}
		\input{../Mod_base/MezziUsati}\backmatter
		\include{quarto/tabelle_disequazioni}
	\end{appendices}
	
	\addcontentsline{toc}{chapter}{\indexname}
	\printindex
	\printindex[dissec]
\end{document}
