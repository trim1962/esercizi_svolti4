\chapter{Le equazioni esponenziali}
\label{cha:LeEquazioniEsponenziali}
\begin{definizionet}{Equazione esponenziale}{}\label{def:EquazioneEsponenziale}
Chiamo equazione esponenziale\index{Equazione!Esponenziale} un'equazione del tipo \[a^{f(x)}=a^{g(x)}\quad a>0\quad  a\neq1 \]
\end{definizionet}
L'equazione esponenziale ha una sola soluzione. 
Dall'equazione esponenziale\nobs\vref{def:EquazioneEsponenziale} è possibile passare all'equazione \[f(x)=g(x) \]Non tutte le equazioni esponenziali sono riconducibili alla forma canonica.
\begin{esempiot}{Equazione esponenziale}{}
	Risolvere l'equazione $2^{3x+1}=2^{5x+2} $
\end{esempiot}
\begin{NodesList}[margin=4cm]
\begin{align*}
2^{3x+1}=2^{5x+2}\AddNode\\
3x+1=5x+2\AddNode\\
3x-5x=2-1\AddNode\\
x=-\dfrac{1}{2}\AddNode
\end{align*}
\LinkNodes{Uguaglio gli esponenti}%
\LinkNodes{Semplifico}%
\LinkNodes{Risolvo}%
\end{NodesList}
\[x=-\dfrac{1}{2}\]
\'{e} la soluzione.

Un altro tipo di equazione esponenziale si ha quando una base è una potenza dell'altra. In questo caso bisogna prima trasformare una base nell'altra utilizzando le proprietà delle potenze.
\begin{esempiot}{Equazione esponenziale}{}
	Risolvere l'equazione $9^{x+1}=3^{3x+4}$
\end{esempiot}
	In questo caso le basi sembrano diverse $9\neq 3$ ma $9$ \'{e} una potenza del $3$ si procede come segue:
	\begin{NodesList}[margin=4cm]
		\begin{align*}
			9^{x+1}=3^{3x+4}\AddNode\\
			(3^2)^{x+1}=3^{3x+4}\AddNode\\
			\intertext{\hfil Potenza di potenze}
			3^{2x+2}=3^{3x+4}\AddNode\\
			2x+2=3x+4\AddNode\\
			2x-3x=-2+4\AddNode\\
			x=-4\AddNode
		\end{align*}
		\LinkNodes{Trasformo le basi}%
		\LinkNodes{}%
		\LinkNodes{Uguaglio gli esponenenti}%
		\LinkNodes{Semplifico}%
		\LinkNodes{Risolvo}%
	\end{NodesList}
	\[x=-4\]
	\'{e} la soluzione.
Un altra situazione è il caso in cui le basi non sono uguali. Per risolverla dobbiamo avere esponenti uguali. Con procedimenti algebrici, si tende a trasformare l'equazione semplificando l'espressione. 
\begin{esempiot}{Equazione esponenziale}{}
	Risolvere l'equazione $7^{x+1}+2\cdot7^{x}=4\cdot3^{x+2}+13\cdot 3^x$
\end{esempiot}
	\begin{NodesList} [margin=4cm]
		\begin{align*}
			7^{x+1}+2\cdot7^{x}=4\cdot3^{x+2}+13\cdot 3^x\AddNode\\
			\intertext{\hfil indici uguali}
			7\cdot7^x+2\cdot7^x=4\cdot9\cdot3^x+13\cdot3^x\AddNode\\
			(7+2)7^x=36\cdot3^x+13\cdot3^x\AddNode\\
			9\cdot7^x=49\cdot3^x\AddNode\\
			\dfrac{7^x}{3^x}=\dfrac{49}{9}\AddNode\\
			x=2\AddNode
		\end{align*}
		\LinkNodes{Semplifico l'espressione}%
		\LinkNodes{Trasformo}%
		\LinkNodes{Ottengo}%
		\LinkNodes{Risolvo}%
		\LinkNodes{Ottengo la soluzione}%
	\end{NodesList}
	\[x=2\]
	\'{e} la soluzione accettabile.
Un altro tipo di equazione che è riconducibile alle precedenti si ha quando una base si ripete. Il procedimento è semplice, con un raccoglimento totale raccolgo l'elemento in comune. 
\begin{esempiot}{Equazione esponenziale}{}
	Risolvere l'equazione $3^{2x+1}+3^{2x}-3^{2x+2}=-45$
\end{esempiot}
	In questa situazione si ripete sempre la potenza $3^{2x}$  si procede come segue:
	\begin{NodesList}[margin=4cm]
		\begin{align*}
			3^{2x+1}+3^{2x}-3^{2x+2}=-45\AddNode\\
			3^{2x}3+3^{2x}-3^{2x}3^2=-45\AddNode\\
			\intertext{\hfil Raccolgo $3^{2x}$}
			3^{2x}(3+1-9)=-45\AddNode\\
			-5\cdot 3^{2x}=-45\AddNode\\
			\intertext{\hfil Divido per $-5$}
			3^{2x}=9\AddNode\\
			2x=2\AddNode\\
			x=1\AddNode
		\end{align*}
		\LinkNodes{Separo $3^{2x}$}%
		\LinkNodes{Raccolgo}%
		\LinkNodes{Sommo}%
		\LinkNodes{Semplifico}%
		\LinkNodes{Uguaglio gli esponenti}%
		\LinkNodes{Risolvo}%
	\end{NodesList}
	\[x=1\]
	\'{e} la soluzione.
Equazioni esponenziali risolvibili tramite sostituzioni. Questa volta si trasforma l'equazione esponenziale in un altra non esponenziale. Si risolve la nuova equazione le cui soluzioni permettono di risolvere l'esponenziale di partenza.
\begin{esempiot}{Equazione esponenziale}{}
	Risolvere l'equazione $3^{2x}-8\cdot 3^x-9=0$
\end{esempiot}
	 è un'equazione del tipo $a^{2x}+a^{x}+c=0$ si risolve con la sostituzione $a^{x}=t$ Ovviamente $t>0$ 
	\begin{NodesList} [margin=4cm]
		\begin{align*}
			3^{2x}-8\cdot 3^x-9=0\AddNode\\
			%\intertext{\hfil $2^x=t$}
			t^2-8t-9=0\AddNode\\
			t_{1,2}=\dfrac{8\pm\sqrt{64+36}}{2}\AddNode\\
			\intertext{\hfil Inverto la trasformazione}
			t_{1}=9\AddNode\\
			3^{x}=9\AddNode\\
			x=2\AddNode\\
			\intertext{\hfil Uso la seconda soluzione}
			t_{2}=-1\AddNode\\
			3^x=-1\AddNode
		\end{align*}
		\LinkNodes{Trasformo l'equazione}%
		\LinkNodes{Risolvo}%
		\LinkNodes{Ottengo due soluzioni}%
		\LinkNodes{Risolvo l'equazione esponenziale}%
		\LinkNodes{Uguaglio gli esponenti e risolvo}%
		\LinkNodes{}%
		\LinkNodes{Equazione impossibile}%
	\end{NodesList}
	\[x=2\]
	\'{e} la soluzione.
Un altro esempio è il seguente
\begin{esempiot}{Equazione esponenziale}{}
	Risolvere l'equazione $ 2^{x+1}-2^{2-x}=7$ 
\end{esempiot}
	è un'equazione del tipo $a^{x}+a^{-x}+c=0$ si risolve con la sostituzione $a^{x}=t$
	Ovviamente $t>0$ 
	\begin{NodesList} [margin=4cm]
		\begin{align*}
			 2\cdot 2^{x}-4\cdot 2^{-x}-7=0\AddNode\\
			 \intertext{\hfil $2^x=t$}
			2t-4t^{-1}-7=0\AddNode\\
			2t-4\dfrac{1}{t}-7=0\AddNode\\
			2t^2-7x-4=0\AddNode\\
			t_{1,2}=\dfrac{7\pm\sqrt{49+32}}{4}\AddNode\\
			\intertext{\hfil Inverto la trasformazione}
			t_{1}=4\AddNode\\
			2^{x}=4\AddNode\\
			x=2\AddNode\\
			\intertext{\hfil Uso la seconda soluzione}
			t_{2}=-\dfrac{1}{4} \AddNode\\
			3^x=-\dfrac{1}{4}\AddNode
		\end{align*}
		\LinkNodes{Trasformo l'equazione}%
		\LinkNodes{Trasformo}%
		\LinkNodes{Ottengo}%
		\LinkNodes{Risolvo}%
		\LinkNodes{Ottengo due soluzioni}%
		\LinkNodes{Risolvo l'equazione esponenziale}%
		\LinkNodes{Uguaglio gli esponenti e risolvo}%
		\LinkNodes{}%
		\LinkNodes{Equazione impossibile}%
	\end{NodesList}
	\[x=2\]
	\'{e} la soluzione accettabile.
\altapriorita{risistemare il tutto non  funziona qualcosa in tikz}