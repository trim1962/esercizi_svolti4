\chapter{Campo di esistenza e positività}
\tcbstartrecording
\section{Funzioni razionali}
\begin{exercise}
	Trovare il $C.E.$ della funzione $y=\dfrac{3x^2+2+3x}{7x-2-5x^2} $
	\tcblower
	Trovare il $C.E.$ della funzione $y=\dfrac{3x^2+2+3x}{7x-2-5x^2} $
	
	La funzione è razionale fratta quindi
		\begin{enumerate}
		\item Il denominatore non può essere zero.
	\end{enumerate}
\begin{align*}
\intertext{Quindi per la condizione}
7x-2-5x^2=&0\\
x_{1,2}=&\xunodue{-5}{7}{-2}\\
=&\dfrac{-7\pm\sqrt{49-40}}{-10}\\
=&\dfrac{-7\pm \sqrt{9}}{-10}\\
=&\dfrac{-7\pm 3}{-10}\\
=&\begin{cases}
x_1=\dfrac{-10}{-10}=1\\
x_2=\dfrac{-4}{-10}=\dfrac{2}{5}
\end{cases}\\
\end{align*}
Quindi il $C.E.$ è 
\begin{align*}
\forall x\in\R-\lbrace 1,&\dfrac{2}{5}\rbrace\\
\end{align*}
\end{exercise}
\section{Funzioni irrazionali}
\begin{exercise}
	Trovare il $C.E.$ della funzione $y=\sqrt{\dfrac{x+5}{x+3}}$%\tipo{PP}
	\tcblower
	Trovare il $C.E.$ della funzione $y=\sqrt{\dfrac{x+5}{x+3}}$
	
	La funzione è irrazionale pari fratta quindi:
	\begin{enumerate}
		\item Il radicando deve essere positivo o uguale a zero.
		\item Il denominatore non può essere zero.
	\end{enumerate}
\begin{align*}
\intertext{La prima condizione impone che:}
\dfrac{x+5}{x+3}\geq&0
\intertext{che equivale a:}
x+5\geq&0\\
x\geq&-5
\intertext{La seconda condizione impone che}
x+3>&0\\
x>&-3
\end{align*}
Otteniamo il grafico 
\begin{center}
	\includestandalone{quarto/grafici/disfrazIrPaFr1}
\end{center}
Quindi il $C.E.$ è 
\begin{align*}
x\leq&5\\
-3<& x
\end{align*}
\end{exercise}
\begin{exercise}
	Trovare il $C.E.$ della funzione $y=\sqrt[5]{\dfrac{x+5}{x+3}}$
	\tcblower
	Trovare il $C.E.$ della funzione $y=\sqrt[5]{\dfrac{x+5}{x+3}}$
	
	La funzione è irrazionale dispari fratta quindi:
	\begin{enumerate}
		\item Il radicando può essere negativo o positivo o nullo.
		\item Il denominatore non può essere zero.
	\end{enumerate}
	\begin{align*}
	\intertext{La seconda condizione impone che}
	x+3=&0\\
	x=&-3
	\end{align*}
	Quindi il $C.E.$ è 
	\begin{align*}
	\forall x\in\R-\lbrace-3&\rbrace\\
	\end{align*}
\end{exercise}
\begin{exercise}
	Trovare il $C.E.$ della funzione $y=\dfrac{\sqrt{x+5}}{x+3}$
	\tcblower
	Trovare il $C.E.$ della funzione $y=\dfrac{\sqrt{x+5}}{x+3}$
	
	La funzione è irrazionale pari fratta quindi:
	\begin{enumerate}
		\item Il radicando deve essere positivo o uguale a zero.
		\item Il denominatore non può essere zero.
	\end{enumerate}
	\begin{align*}
	\intertext{La prima condizione impone che:}
	x+5\geq&0
	\intertext{che equivale a:}
	x+5\geq&0\\
	x\geq&-5
	\intertext{La seconda condizione impone che}
	x+3=&0\\
	x=&-3
	\end{align*}
	Otteniamo il grafico 
	\begin{center}
		\includestandalone{quarto/grafici/disfrazIrPaFr2}
	\end{center}
	Quindi il $C.E.$ è 
	\begin{align*}
	-5\leq x<-3\\
	-3<& x
	\end{align*}
	o in maniera equivalente
	\begin{align*}
	-5\leq& x\\
     x\neq&-3
	\end{align*}
\end{exercise}
\begin{exercise}
	Trovare il $C.E.$ della funzione $y=\dfrac{x+5}{\sqrt{x+3}}$
	\tcblower
		Trovare il $C.E.$ della funzione $y=\dfrac{x+5}{\sqrt{x+3}}$\tipo{P}
	
	La funzione è irrazionale pari fratta quindi:
	\begin{enumerate}
		\item Il radicando deve essere positivo.
		\item Il denominatore non può essere zero.
	\end{enumerate}
	\begin{align*}
	\intertext{Le condizioni equivalgono a}
	x+3>&0\\
	x>&-3
	\end{align*}
	Otteniamo il grafico 
	\begin{center}
		\includestandalone{quarto/grafici/disfrazIrPaFr3}
	\end{center}
	Quindi il $C.E.$ è 
	\begin{align*}
	-3<& x
	\end{align*}
	
\end{exercise}
\begin{exercise}
	Trovare il $C.E.$ della funzione $y=\dfrac{x+5}{\sqrt[3]{x+3}}$
	\tcblower
	Trovare il $C.E.$ della funzione $y=\dfrac{x+5}{\sqrt[3]{x+3}}$\tipo{P}
	
	La funzione è irrazionale dispari fratta quindi:
	\begin{enumerate}
	\item Il radicando deve essere positivo o nullo.
	\item Il denominatore non può essere zero.
\end{enumerate}
	\begin{align*}
	\intertext{Le condizioni equivalgono a}
1-x^2\geq&0\\
x+4
	\end{align*}
	Quindi il $C.E.$ è 
\begin{align*}
\forall x\in\R-\lbrace-3&\rbrace\\
\end{align*}
	
\end{exercise}

\begin{exercise}
	Trovare il $C.E.$ della funzione $y=\sqrt{\frac{(1-x^{2})\cdot (x+4)}{(x-2)}}$
	\tcblower
	Trovare il $C.E.$ della funzione $y=\sqrt{\dfrac{(1-x^{2})\cdot (x+4)}{(x-2)}}$\tipo{AP}
	
	La funzione è irrazionale pari fratta quindi:
	\begin{enumerate}
	\item Il radicando deve essere positivo o nullo.
	\item Il denominatore non può essere zero.
	\end{enumerate}
	\begin{align*}
	\intertext{La prima condizione equivale a:}
	\frac{(1-x^{2})\cdot (x+4)}{(x-2)}\geq &0
	\intertext{quindi}
	1-x^2\geq&0\\
x_{1,2}=&\xunodue{-1}{0}{1}\\
=&\dfrac{0\pm\sqrt{0-4(-1)(-1)}}{-2}\\
=&\begin{cases}
x_1=\dfrac{2}{-1}=-1\\
x_2=\dfrac{-2}{-2}=+1
\end{cases}\\
x+4\geq&0\\
\intertext{Quindi per la seconda condizione}
x-2>&0
	\end{align*}
		Otteniamo il grafico 
	\begin{center}
		\includestandalone{quarto/grafici/disfrazIrPaFr4}
	\end{center}
	Quindi il $C.E.$ è 
	\begin{align*}
	-4\leq x&\leq-1\\1\leq x&<2
	\end{align*}
	
\end{exercise}

\begin{exercise}
	Trovare il $C.E.$  della funzione  $y=\dfrac{\sqrt{x^2-1}}{x+1}$ e quando è positiva.
	\tcblower
	Trovare il $C.E.$  della funzione  $y=\dfrac{\sqrt{x^2-1}}{x+1}$ e quando è positiva. \tipo{AP}
	
	La funzione è irrazionale pari fratta quindi:
	\begin{enumerate}
		\item Il radicando deve essere positivo o nullo.
		\item Il denominatore non può essere zero.
	\end{enumerate}
	\begin{align*}
	\intertext{La prima condizione equivale a:}
	x^2-1\geq &0
	\intertext{quindi}
	x_{1,2}=&\xunodue{1}{0}{-1}\\
	=&\dfrac{0\pm\sqrt{0-4(1)(-1)}}{2}\\
	=&\begin{cases}
	x_1=\dfrac{2}{-1}=-1\\
	x_2=\dfrac{-2}{-2}=+1
	\end{cases}\\
	\intertext{Quindi per la seconda condizione}
	x+1=&0\\
	x=&-1\\
	\end{align*}
	Otteniamo il grafico 
	\begin{center}
		\includestandalone{quarto/grafici/disfrazIrPaFr5}
	\end{center}
	Quindi il $C.E.$ è 
	\begin{align*}
	-1< x&\leq 1\\
	\end{align*}
	
\end{exercise}