\section{Funzioni razionali}
\begin{esempiot}{Razionale **}{}
Consideriamo la funzione\[f(x)=x^4-x^2 \]
\end{esempiot}
\begin{enumerate}[noitemsep]
	\litem{Classificazione}la funzione è una funzione razionale. 
	 \litem{Dominio}il dominio della funzione è $\R$.
	\litem{Positività}risolviamo la disequazione
	\begin{align*}
	&x^4-x^2>0
	\intertext{Raccolgo $x^2$}
	&x^2(x^2-1)>0\\
	\intertext{ottengo due disequazioni}
	&x^2>0\\
	&x^2-1>0\\
	\end{align*}
	Ottengo il~\cref{fig:Drazionale1} quindi la funzione è positiva per $-1<x$ e $x>1$
	\litem{Intersezioni assi}l'intersezione con l'asse $x$ si ottiene
	\begin{align*}
	&\begin{cases}
	y=0\\
	y=x^4-x^2
	\end{cases}&\begin{cases}
	y=0\\
	y=x^2(x^2-1)
	\end{cases}
	\intertext{risolviamo il sistema in due parti}
	&\begin{cases}
	y=0\\
	x^2=0
	\end{cases}
	&\begin{cases}
	y=0\\
	x=0
	\end{cases}
	\intertext{e ottengo}
	&\begin{cases}
	y=0\\
	x^2-1=0
	\end{cases}&\begin{cases}
	y=0\\
	y=1
	\end{cases}
	&\begin{cases}
	y=0\\
	y=-1
	\end{cases}\\
	\end{align*}
	L'intersezione asse $y$ si ottiene con
\begin{align*}
&\begin{cases}
x=0\\
y=x^4-x^2
\end{cases}
&\begin{cases}
x=0\\
y=0
\end{cases}
\end{align*}
Quindi le intersezioni sono $B(0,0)$, $A(-1,0)$, $C(1,0)$
\litem{Pari e dispari} $f(-x)=(-x)^4-(-x)^2=x^4-x^2=f(x)$ la funzione è pari
\end{enumerate}
La~\cref{exa:razio6} riassume quanto detto.
\begin{figure}
	\captionsetup{name=Grafico}
	\centering
	\includestandalone[width=8.5cm]{quarto/dominio/disequazione1}
	\caption{Segno funzione}
	\label[graf]{fig:Drazionale1}
\end{figure}
\begin{funzionet}{Razionale}{razio6}
	\includestandalone[width=\linewidth]{quarto/dominio/razio6}
	\tcblower
	\begin{itemize}
		\item $y=x^4-x^2$
		\item Dominio $\R$
		\item Codominio $\R$
		\item Positività $-1<x$ e $x>1$
		\item Intersezione asse $x$ $A(-1,0)$, $C(1,0)$,
		\item Intersezione asse $y$ $B(0,0)$
		%		\item Crescente
		\item Asintoti verticali nessuno
		\item Asintoti orizzontali nessuno
		\item Funzione pari
	\end{itemize}
\end{funzionet}
\begin{esempiot}{Razionale *}{}
Consideriamo la funzione\[f(x)=2x+1\]
\end{esempiot}
\begin{enumerate}[noitemsep]
\litem{Classificazione}la funzione è una funzione razionale. 
 \litem{Dominio}il dominio della funzione è $\R$.
\litem{Positività}risolviamo la disequazione
\begin{align*}
&2x+1>0
\intertext{separo le variabili}
&x>-\dfrac{1}{2}\\
\end{align*}
Ottengo il~\cref{fig:Drazionale2} quindi la funzione è positiva per $x>-\dfrac{1}{2}$
\litem{Intersezioni assi}l'intersezione con l'asse $x$ si ottiene con
\begin{align*}
&\begin{cases}
y=0\\
y=2x+1
\end{cases}&\begin{cases}
y=0\\
2x+1=0
\end{cases}
\intertext{ottengo}
&\begin{cases}
y=0\\
x=-\dfrac{1}{2}
\end{cases}
\end{align*}
L'intersezione asse $y$ si ottiene con
\begin{align*}
&\begin{cases}
x=0\\
y=2x+1
\end{cases}
&\begin{cases}
x=0\\
y=1
\end{cases}
\end{align*}
Quindi le intersezioni sono $A(-\dfrac{1}{2},0)$, $B(0,1)$
\litem{Pari e dispari} $f(-x)=-2x+1\neq f(x)$ la funzione non è pari ne dispari
\end{enumerate}
La~\cref{exa:razio1} riassume quanto detto.
\begin{figure}
	\captionsetup{name=Grafico}
	\centering
	\includestandalone[width=8.5cm]{quarto/dominio/disequazione2}
	\caption{Segno funzione}
	\label[graf]{fig:Drazionale2}
\end{figure}
\begin{funzionet}{Razionale}{razio1}
\includestandalone[width=\linewidth]{quarto/dominio/razio1}
	\tcblower
	\begin{itemize}
		\item $y=2x+1$
		\item Dominio $\R$
		\item Codominio $\R$
		\item Positività $x\geq-\dfrac{1}{2}$
		\item Intersezione asse $x$ $(-\dfrac{1}{2},0)$
		\item Intersezione asse $y$ $(0,1)$
%		\item Crescente
		\item Asintoti verticali nessuno
		\item Asintoti orizzontali nessuno
		\item Funzione nè pari nè dispari
	\end{itemize}
	\end{funzionet}
%
\begin{esempiot}{Razionale fratta *}{}
	Consideriamo la funzione\[f(x)=\dfrac{1}{x}\]
\end{esempiot}
\begin{enumerate}[noitemsep]
	\litem{Classificazione}la funzione è una funzione razionale fratta.
	\litem{Dominio}il dominio della funzione è $\R-\lbrace 0\rbrace$.
	\litem{Positività}risolviamo la disequazione
	\begin{align*}
	&\dfrac{1}{x}>0
	\end{align*}
	Ottengo il~\cref{fig:Drazionale3}. La funzione è positiva per $x>0$
	\litem{Intersezioni assi}l'intersezione con l'asse $x$ non esiste.
	L'intersezione asse $y$ non esiste.
	\litem{Pari e dispari} $f(-x)=-\dfrac{1}{x}=-f(x)$ la funzione è dispari
\end{enumerate}
La~\cref{exa:ratio2} riassume quanto detto.
\begin{figure}
	\captionsetup{name=Grafico}
	\centering
	\includestandalone[width=8.5cm]{quarto/dominio/disequazione3}
	\caption{Segno funzione}
	\label[graf]{fig:Drazionale3}
\end{figure}
\begin{funzionet}{Razionale fratta *}{ratio2}
	\includestandalone[width=\textwidth]{quarto/dominio/razio2}
	\tcblower
	\begin{itemize}
		\item $y=\dfrac{1}{x}$
		\item Dominio $\R-\lbrace0\rbrace$
		\item Codominio $\R$
		\item Positività $x\geq0$
		\item Intersezione asse $x$ nessuna
		\item Intersezione asse $y$ nessuna
%		\item Decrescente
		\item Asintoti verticali uno
		\item Asintoti orizzontali uno
		\item Funzione è dispari
	\end{itemize}
\end{funzionet}
\begin{esempiot}{Razionale *}{}
	Consideriamo la funzione\[f(x)=\dfrac{x}{x^2-1} \]
\end{esempiot}
\begin{enumerate}[noitemsep]
	\litem{Classificazione}la funzione è una funzione razionale fratta. 
	\litem{Dominio}dato che è una razionale fratta, determiniamo il dominio ponendo il denominatore uguale a zero. \[x^2-1=0\] 
	\[x_{1,2}=\dfrac{0\pm\sqrt{4}}{2}=\begin{cases}
	x_1=1\\x_2=-1
	\end{cases} \]
	Il dominio della funzione è $\R-\lbrace 1,-1\rbrace$.
	\litem{Positività}risolviamo la disequazione
	\begin{align*}
	&\dfrac{x}{x^2-1}\geq 0
	\intertext{ripartiamo da quello che abbiamo trovato prima}
	&x^2-1>0
	\intertext{risolviamo il numeratore}
	&x>0\\
	\end{align*}
	Ottengo il~\cref{fig:Drazionale4}. La funzione è positiva per $-1<x<0$ e $x>1$
	\litem{Intersezioni assi}l'intersezione con l'asse $x$ si ottiene
	\begin{align*}
	&\begin{cases}
	y=0\\
	y=\dfrac{x}{x^2-1}
	\end{cases}&\begin{cases}
	y=0\\
	x=0
	\end{cases}
	\end{align*}
	L'intersezione asse $y$ si ottiene con
	\begin{align*}
	&\begin{cases}
	x=0\\
	y=\dfrac{x}{x^2-1}
	\end{cases}
	&\begin{cases}
	x=0\\
	y=0
	\end{cases}
	\end{align*}
	Quindi l'intersezione è $A(0,0)$, 
	\litem{Pari e dispari} $f(-x)=\dfrac{-x}{(-x)^2-1}=-\dfrac{x}{x^2-1}=-f(x)$ la funzione è dispari
\end{enumerate}
La~\cref{exa:razio3} riassume quanto detto.
\begin{figure}
	\captionsetup{name=Grafico}
	\centering
	\includestandalone[width=8.5cm]{quarto/dominio/disequazione4}
	\caption{Segno funzione}
	\label[graf]{fig:Drazionale4}
\end{figure}
\begin{funzionet}{Razionale fratta}{razio3}
	\includestandalone[width=\textwidth]{quarto/dominio/razio3}
	\tcblower
	\begin{itemize}
		\item $y=\dfrac{x}{x^2-1}$
		\item Dominio $\R-\lbrace-1,1\rbrace$
		\item Codominio $\R$
		\item Positività $-1<x<0\vee x>1$
		\item Intersezione asse $x$ $(0,0)$ 
		\item Intersezione asse $y$ $(0,0)$ 
%		\item Decrescente 
	%	\item Crescente per $x>2$
		\item Asintoti verticali due
		\item Asintoti orizzontali uno
		\item La funzione è dispari
	\end{itemize}
\end{funzionet}
\begin{esempiot}{Razionale *}{}
	Consideriamo la funzione\[f(x)=\dfrac{x^2-1}{x^2-4} \]
\end{esempiot}
\begin{enumerate}[noitemsep]
	\litem{Classificazione} la funzione è una funzione razionale fratta. 
	\litem{Dominio} dato che è una razionale fratta determini il dominio ponendo il denominatore uguale a zero. \[x^2-4=0\] 
	\[x_{1,2}=\dfrac{0\pm\sqrt{16}}{2}=\begin{cases}
	x_1=2\\x_2=-2
	\end{cases} \]
	Il dominio della funzione è $\R-\lbrace 2,-2\rbrace$.
	\litem{Positività}risolviamo la disequazione
	\begin{align*}
	&\dfrac{x^2-1}{x^2-4}>0
	\intertext{ripartiamo da quello che stato trovato prima}
	&x^2-4>0
	\intertext{risolviamo il numeratore}
	&x^2-1>0\\
	\end{align*}
	Ottengo il~\cref{fig:Drazionale7}. La funzione è positiva per $-1<x<1\vee x>2\vee x<-2 $
	\litem{Intersezioni assi}l'intersezione con l'asse $x$ si ottiene
	\begin{align*}
	&\begin{cases}
	y=0\\
	y=\dfrac{x^2-1}{x^2-4}
	\end{cases}&\begin{cases}
	y=0\\
	x^2-1=0
	\end{cases}\\
	&\begin{cases}
	y=0\\
	x=1
	\end{cases}
		&\begin{cases}
	y=0\\
	x=-1
	\end{cases}\\
	\end{align*}
	L'intersezione asse $y$ si ottiene con
	\begin{align*}
	&\begin{cases}
	x=0\\
	y=\dfrac{x^2-1}{x^2-4}
	\end{cases}
	&\begin{cases}
	x=0\\
	y=\dfrac{1}{4}
	\end{cases}
	\end{align*}
	Quindi l'intersezione è $A(-1,0)$, $B(1,0)$, $C(0,\dfrac{1}{4})$
	\litem{Pari e dispari} $f(-x)=\dfrac{(-x)^2-1}{(-x)^2-4}=\dfrac{x^2-1}{x^2-2}=f(x)$ la funzione è pari
\end{enumerate}
La~\cref{exa:razio7} riassume quanto detto.
\begin{figure}
	\captionsetup{name=Grafico}
	\centering
	\includestandalone[width=8.5cm]{quarto/dominio/disequazione7}
	\caption{Segno funzione}
	\label[graf]{fig:Drazionale7}
\end{figure}
\begin{funzionet}{Razionale fratta}{razio7}
	\includestandalone[width=\textwidth]{quarto/dominio/razio7}
	\tcblower
	\begin{itemize}
		\item $y=\dfrac{x^2-1}{x^2-4}$
		\item Dominio $\R-\lbrace-2,2\rbrace$
		\item Codominio $\R$
		\item Positività $-1<x<1\vee x>2\vee x<-2 $
		\item Intersezione asse $x$ $(-1,0)$, $(1,0)$, 
		\item Intersezione asse $y$ $(0,\dfrac{1}{4})$
		%		\item Decrescente 
		%	\item Crescente per $x>2$
		\item Asintoti verticali due
		\item Asintoti orizzontali uno
		\item La funzione è pari
	\end{itemize}
\end{funzionet}
\begin{esempiot}{Razionale **}{}
	Consideriamo la funzione\[f(x)=\dfrac{x^2-2x-3}{x^2+6x+5} \]
\end{esempiot}
\begin{enumerate}[noitemsep]
	\litem{Classificazione}la funzione è una funzione razionale fratta.
	\litem{Dominio}dato che è una razionale fratta determini il dominio ponendo il denominatore uguale a zero. \[x^2+6x+5=0\] 
	\[x_{1,2}=\dfrac{-6\pm\sqrt{36-20}}{2}=\begin{cases}
	x_1=-5\\x_2=-1
	\end{cases} \] La funzione per $f(-1)=\dfrac{(-1)^2-2(-1)-3}{(-1)^2-6+5}=\dfrac{0}{0}$
	in questo caso risolvo \[x^2-2x-3=0\] 
	\[x_{1,2}=\dfrac{2\pm\sqrt{4+12}}{2}=\begin{cases}
	x_1=+3\\x_2=-1	\end{cases} \]
	Dato che 
	\begin{align*}
x^2-2x-3=&(x-3)(x+1)\\
x^2+6x+5=&(x+5)(x+1)\\
y=\dfrac{x^2-2x-3}{x^2+6x+5}=&\dfrac{(x-3)(x+1)}{(x+5)(x+1)}\\
y=&\dfrac{x-3}{x+5}
	\end{align*}
	Il dominio della funzione è $\R-\lbrace -5\rbrace$.
	\litem{Positività}risolviamo la disequazione
	\begin{align*}
	&\dfrac{x-3}{x+5}>0\\
	&x-3>0\\
	&x>3
	\intertext{risolviamo il numeratore}
	&x+5>0\\
	&x>-5
	\end{align*}
	Ottengo il~\cref{fig:Drazionale8}. La funzione è positiva per $x<-5\vee x>3$
	\litem{Intersezioni assi}l'intersezione con l'asse $x$ si ottiene
	\begin{align*}
	&\begin{cases}
	y=0\\
	y=\dfrac{x-3}{x+5}
	\end{cases}&\begin{cases}
	y=0\\
	x=3
	\end{cases}\\
	&\begin{cases}
	y=0\\
	x=1
	\end{cases}
	\end{align*}
	L'intersezione asse $y$ si ottiene con
	\begin{align*}
	&\begin{cases}
	x=0\\
	y=\dfrac{x-3}{x+5}
	\end{cases}
	&\begin{cases}
	x=0\\
	y=-\dfrac{-3}{5}
	\end{cases}
	\end{align*}
	Quindi l'intersezione è $A(3,0)$, $B(0,-\dfrac{3}{5})$
	\litem{Pari e dispari} $f(-x)=\dfrac{(-x)-3}{(-x)+5}=\dfrac{-(x+3)}{-(x-5)}\neq f(x)$ la funzione non è pari.
\end{enumerate}
La~\cref{exa:razio7} riassume quanto detto.
\begin{figure}
	\captionsetup{name=Grafico}
	\centering
	\includestandalone[width=8.5cm]{quarto/dominio/disequazione8}
	\caption{Segno funzione}
	\label[graf]{fig:Drazionale8}
\end{figure}
\begin{funzionet}{Razionale fratta}{razio8}
	\includestandalone[width=\textwidth]{quarto/dominio/razio8}
	\tcblower
	\begin{itemize}
		\item $y=\dfrac{x^2-2x-3}{x^2+6x+5}=\dfrac{x-3}{x+5}$
		\item Dominio $\R-\lbrace-5\rbrace$
		\item Codominio $\R$
		\item Positività $-1<x<1\vee x>2\vee x<-2 $
		\item Intersezione asse $x$ $(-1,0)$, $(1,0)$, 
		\item Intersezione asse $y$ $(0,\dfrac{1}{4})$
		%		\item Decrescente 
		%	\item Crescente per $x>2$
		\item Asintoti verticali due
		\item Asintoti orizzontali uno
		\item La funzione è pari
	\end{itemize}
\end{funzionet}